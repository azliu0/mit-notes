\section{April 25, 2023}

\subsection{Approximation Algorithms}

Notation:
\begin{itemize}
    \item Let $\mathcal{P}$ denote a class of problems
    \item Let $P\in \mathcal{P}$ denote a specific problem 
    \item Let $\mathcal{A}$ denote an algorithm for the class of problems
    \item Let $x$ be a proposed specific solution
    \item Let $\mathcal{A}(P)$ be the quality of the solution obtained by $\mathcal{A}$
    \item Let \textsc{OPT}$(P)$ be the quality of the optimal solution
\end{itemize}

\ac{$\alpha$-approximation algorithm}: for $\mathcal{P}$, we are given some approximation ratio $\alpha \geq 1$ In minimization, $\mathcal{A}(P)\leq \alpha\cdot \textsc{OPT}(P)$. In maximization, $\textsc{OPT}(P)\leq \alpha\cdot \mathcal{A}(P)$.

\subsection{Approximation Schemes}

\begin{itemize}
    \item minimization: $\forall \varepsilon > 0$, $\forall P\in \mathcal{P}$, 
    \[S(P,\varepsilon) \leq (1+\varepsilon)\textsc{OPT}(P).\]
    \item maximization: $\forall \varepsilon > 0$, $\forall P\in \mathcal{P}$, 
    \[\textsc{OPT}(P)\leq (1+\varepsilon)S(P,\varepsilon).\]
\end{itemize}

Efficient -- polynomial time approximation scheme. 
Fully efficient -- fully polynomial time approximation scheme, i.e., polynomial in both the size of the problem and $1/\varepsilon$. 

\begin{example}
\exlabel

Maximum matching.
\end{example}

Proposed algorithm: go over edges greedily, and add an edge to the matching if possible. Let $\vert M\vert$ be the size of the matching obtained using this algorithm. Let $\vert M^*\vert$ be the size of the true maximal matching. 

Note that solving this problem exactly can be done with max flow. On the other hand, this approximation algorithm is a linear scan through the vertices, so the runtime is much more efficient. 

\begin{theorem}
\claimlabel

This algorithm is a $2$-approximation, i.e., $\vert M^*\vert \leq 2\vert M\vert$. 
\end{theorem}

\begin{proof}
$e\in M$, $e^*\in M^*$, label $e$ with $e^*$ if they share an endpoint. Each $e\in M$ has $\leq 2$ labels, since $M^*$ is a matching, so the total number of labels is $\leq 2\vert M\vert$. On the other hand, every edge in $M^*$ is included as a label. If not, then any edge that is not included as a label is not adjacent to any of the edges in $M$, meaning that the greedy algorithm would have added it. So the total number of labels is $\geq \vert M^*\vert$, which completes the proof. 
\end{proof}

\begin{example}
\exlabel

Vertex Cover. 
\end{example}

Recall that the vertex covering problem seeks to find the smallest subset $S$ of vertices such that every edge in the graph is incident to at least one vertex in $S$. 

Consider some maximal matching found by the previous greedy algorithm, $M$. Let $V(M)$ denote the set of vertices adjacent to edges in $M$. Then, $\vert V(M)\vert$ is a $2$-approximation to the size of best vertex cover $\vert V^*\vert$. 

\begin{theorem}
\claimlabel

$V(M)$ is a vertex cover, and $\vert V(M)\vert \leq 2\vert V^*\vert$. 
\end{theorem}

\begin{proof}
If $V(M)$ is not a vertex cover, there exists some edge that is not adjacent to any other vertices in $V(M)$, so it can be added to $M$ by the greedy algorithm. Further, $\vert M\vert \leq \vert V^*\vert$; since $M$ is a matching, $V^*$ must include at least one vertex in each edge. Therefore, $\vert V(M)\vert = 2\vert M\vert\leq 2\vert V^*\vert$. 
\end{proof}

\begin{example}
\exlabel

LP.
\end{example}

Consider the following integer LP: take variables $x_v\in \{0,1\}$, and compute $\min \sum x_v$ subject to the constraint $x_u+x_v\geq 1$ for all $(u,v)\in E$. 

This formulation is equivalent to the min-vertex cover problem. Since we are forcing integer values, this problem is NP-hard. But, we can relax it so that $0\leq x_v\leq 1$ and obtain a polynomial time solution. 

For this fractional LP, take $x_v$, round it to the nearest integer, and include it in the VC if it becomes $1$. This works, because $x_u+x_v\geq 1$ implies at least one of $\{x_u, x_v\}$ is $\geq 1/2$, so every edge is covered. 

\begin{theorem}
\claimlabel

The rounding LP achieves a $2$-approximation for VC.
\end{theorem}

\begin{proof}
$\sum_{v\in V}\textsc{round}(x_v)\leq$2\textsc{OPT}(fractional)$\leq$2\textsc{OPT}(LP). 
\end{proof}