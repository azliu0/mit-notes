\section{September 9, 2022}
\subsection{Groups}

Quick review of group definitions: a group $G$ with binary operator satisfies:
\begin{itemize}
    \item Associative: $(fg)h=f(gh)\quad \forall f,g,h\in G$
    \item Identities: $\exists 1\in G$ s.t. $1g=g1=g\quad \forall g\in G$
    \item Inverses: every element has an inverse.
\end{itemize}

\begin{theorem}
\proplabel

The group identity is unique. So are inverses.
\end{theorem}

\begin{proof}
If $1$, $1'$ are both identities, then $1=11'=1'$. Since the identity is unique, inverses are also necessarily unique.
\end{proof}

\begin{definition}
\deflabel

There is one group with one element (only the identity). We call this group the \ac{trivial group}. 
\end{definition}

\begin{definition}
\deflabel

$C_n = \{1,g,\hdots, g^{n-1}\}$, with $g^n=1$, is the \ac{cyclic group} of order $n$. 
\end{definition}

In general, there are lots of ways to represent different groups. For example, here are two multiplication tables for $C_3$:
\[\begin{array}{c|ccc}
     \cdot &1&g&g^2\\
     \hline
     1&1&g&g^2\\
     g&g&g^2&g\\
     g^2&g^2&1&g\\
\end{array}\qquad 
\begin{array}{c|ccc}
     + &0&1&2\\
     \hline
     0&0&1&2\\
     1&1&2&0\\
     2&2&1&0\\
\end{array}
\]
These different representations are \ac{isomorphic}. Also, note the symmetry along the diagonal of each multiplication table. Each element in $C_3$ commutes, so we say that it is \ac{abelian} (commutative). This property cannot be assumed to hold for general groups.

\subsection{Classifying small groups}

Let $G$ be a group with $\vert G\vert = n$, that is, the \ac{order} of $G$ is $n$. Let's classify groups when $n$ is small. 

\begin{example}
\exlabel

$n=1$.
\end{example}

There is only one possibility here, the trivial group. 

\begin{example}
\exlabel

$n=2$.
\end{example}

Let's try filling in a multiplication table and see what possibilities there are. 
\[\begin{array}{c|cc}
     \cdot & 1&g \\
     \hline
     1&1&g \\
     g&g&? 
\end{array}\]

The last element must be $1$. If not, then $g$ does not have an inverse, which violates the group laws. So, there is only one group with order $2$, which is $C_2$. 

\begin{example}
\exlabel

$n=3$. 
\end{example}

Let's use the same approach. 
\[\begin{array}{c|ccc}
     \cdot & 1&g&h \\
     \hline
     1&1&g&h \\
     g&g&& \\
     h&h&&
\end{array}\]
Those are our freebies. We can use a bit of logic to deduce the rest of the table. We know $g^2\in \{1,g,h\}$. We can't have $g^2=1$, otherwise $gh=h\implies g=1$. We also can't have $g^2=g\implies g=1$. So $g^2=h$, and the rest of the table falls through. In this case, there is also only one group of order $3$, which is $C_3$. 

\begin{example}
\exlabel

$n=4$. 
\end{example}

This time, it turns out there are two groups. $C_4$ works as usual, but we also get another group, $C_2\times C_2$. 

\begin{definition}
\deflabel

Given two groups $(G,\cdot)$ and $(H, \star)$, their \ac{group product} is defined as the set
\[G\times H = \{(g,h) : g\in G, h\in H\},\]
with the operation $(g,h)(g',h') = (g\cdot g', h\star h')$. 
\end{definition}

Since all multiplications are distinct, its obvious why group products of two groups must itself also be a group.

Instead of the traditional ``multiplication'', sometimes we'll write abelian groups additively; for example, $C_n = \{0,1,\hdots, n-1\}$ with addition modulo $n$. In this notation, here's the multiplication table for $C_2\times C_2$: 

\[\begin{array}{c|cccc}
     + & (0,0) & (0,1) & (1,0) & (1,1) \\
     \hline
     (0,0) & (0,0) & (0,1) & (1,0) & (1,1) \\
     (0,1) & (0,1) & (0,0) & (1,1) & (1,0) \\
     (1,0) & (1,0) & (1,1) & (0,0) & (0,1) \\
     (1,1) & (1,1) & (1,0) & (0,1) & (0,0) 
\end{array}\]

We know that $C_2\times C_2$ and $C_4$ aren't isomorphic because all elements in $C_2\times C_2$ square to the identity, which naturally follows from the definition of group product. 

\begin{example}
\exlabel

$n=5$.
\end{example}

As before, the only possible group is $C_5$. In general, for any prime $p$, there is only one group with order $p$, which is $C_p$. 

\begin{example}
\exlabel

$n=6$.
\end{example}

There are also two groups for $n=6$, $C_6$ and $S_3$. This is the first time that we have a non-abelian group. 

Recall that $S_n$ is the symmetric group of permutations. A permutation is a one-to-one correspondence from $\{1,\hdots, n\}$ to $\{1,\hdots, n\}$. The group operation for $S_n$ is composition. Here's a concrete example for $S_5$:

\[\begin{array}{c|ccccc}
     i&1&2&3&4&5  \\
     \hline
     \pi(i)&3&5&1&2&4\\
     \pi'(i)&1&2&5&3&4\\
     \pi\pi'(i)&3&5&4&1&2 \\
\end{array}\]

\subsection{Infinite Groups}

Some groups have infinite order. For example, we introduced last lecture the general linear group:
\[\GL_n(\RR) = \{A\in \RR^{n\times n} : A\text{ invertible}\}\]

$\GL_n(\RR)$ is closed under matrix multiplication. Given any $A,B\in \GL_n(\RR)$, $(AB)^{-1} = B^{-1}A^{-1}$, since $(AB)(B^{-1}A^{-1}) = AA^{-1} = I_n$, so $AB$ is also invertible and in the group. You can also argue this using the fact that a matrix is invertible if and only if its determinant is non-zero. 

\begin{definition}
\deflabel

$H$ is a \ac{subgroup} of $G$ if $H\subseteq G$ and $H$ forms a group itself under $G$'s operation with the same identity, inverses, etc. 
\end{definition}

\begin{example}
\exlabel

Here are some examples of common subgroups. 
\end{example}

\begin{itemize}
    \item The trivial group is a subgroup of any group. 
    \item $(\ZZ, +)$ is a subgroup of $(\RR, +)$
    \item $\SL_n(\RR) = \{A\in \RR^{n\times n}: \det A=1\}$ is a subgroup of $\GL_n(\RR)$. 
    \item The \ac{special orthogonal group} of order $2$ is defined as the set of two-dimensional rotation matrices:
    \[SO_2(\RR) = \left\{\twotwo{\cos{\theta}}{\sin{\theta}}{-\sin{\theta}}{\cos{\theta}}: \theta \in \RR\right\}\]
    Since these matrices always have determinant $1$, we have the chain of subgroups
    \[SO_2(\RR)\subseteq SL_2(\RR)\subseteq GL_2(\RR).\]
\end{itemize}

\subsection{Symmetry Groups}

\begin{definition}
\deflabel

Let $S\subseteq \RR^n$. Then, the \ac{symmetry group} of $S$ is equal to the set of all rigid motions $T$ of $\RR^n$ such that $\{Tx: x\in S\}=S$. 
\end{definition}

For instance, an equilateral triangle has symmetries isomorphic to $S_3$. The isomorphism arises when you label the vertices of the triangle $1,2,3$; then, each rigid motions directly maps to a different permutation. It turns out that this correspondence is only true for $n=3$, and not true in general. 

\begin{definition}
\deflabel

An \ac{isomorphism} $f: G\rightarrow H$ between the groups $(G, \cdot)$ and $(H, \star)$ is a one-to-one correspondence between the elements of $G$ and $H$ such that $f(g_1\cdot g_2) = f(g_1)\star f(g_2)$ for all $g_1,g_2\in G$. 
\end{definition}

When $G$ and $H$ are isomorphic, we write $G\cong H$. 