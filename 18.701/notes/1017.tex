\section{October 17, 2022}
\subsection{Discrete Subgroups of $M_2$}
\begin{definition}
\deflabel

$G\subseteq M_2$ is \ac{discrete} if (1) $\exists \epsilon > 0$ such that no two distinct translation elements in $G$ are within $\epsilon$ of each other and (2) $\exists \epsilon_{\theta} > 0$ such that no two distinct rotation elements in $G$ are within $\epsilon_{\theta}$ of each other.

\end{definition}

\begin{example}
\exlabel

Discrete subgroups of $\RR^2$.
\end{example}

\begin{itemize}
    \item $\{0\}$. This is the trivial group, so it is discrete.
    \item $\ZZ\alpha$, where $\alpha\neq 0$.
    \item $\ZZ\alpha + \ZZ\beta$, where $\alpha,\beta$ linearly independent
\end{itemize}

\begin{definition}
\deflabel

Let $G$ be a discrete subgroup of $M_2$. Define $\pi: G\rightarrow O_2(\RR)$ such that $\pi(Ax+b) = A$. Then, the $\ac{point group}$ $\overline{G} = \im{\pi}$, and the $\ac{lattice}$ $L = \ker{\pi}$. 
\end{definition}

$L$ is a normal subgroup in $G$, so we also have $\overline{G} = G/L$. These definitions should feel somewhat intuitive. The kernel of $\pi$ is the set of translations, since $A=I$, and this corresponds to our lattice. The image of $\pi$ is the set of all possible rotations / reflections of a single point in the plane, which corresponds to the point group. 

\begin{theorem}
\thmlabel

$\forall A\in \overline{G}$, $b_0\in L$, we have $Ab_0\in L$. In other words, $\overline{G}$ preserves the lattice. 
\end{theorem}

Intuitively, if this wasn't true, this would be pretty catastrophic. For example, if your lattice was a square grid, and your point group somehow did not preserve the symmetries of a square, you would generate points outside of your lattice, and therefore your lattice would not be a square grid. 

\begin{proof}
Since $A\in \overline{G}$, there exists some map $\varphi\in G$ taking $x\mapsto Ax+b$. Note that $\varphi^{-1} = A^{-1}x-A^{-1}b$. Conjugating the map $x\mapsto x+b_0$ (which is in $G$, since it is in $L$) by $\varphi$ gives
\[x\mapsto A(A^{-1}x-A^{-1}b+b_0)+b = x+Ab_0,\]
so $Ab_0\in L$. 
\end{proof}

\begin{theorem}
\thmlabelname{Crystallographic Restriction}

If $L\neq \{0\}$, then $\overline{G} = C_n$ or $D_n$ with $n\in \{1,2,3,4,6\}$. 
\end{theorem}

\begin{proof}
Pick some $\alpha \in L-\{0\}$ with $\alpha$ minimal. Suppose $\rho\in G$ is a rotation by $2\pi/n$. 
\begin{center}
\begin{asy}
import graph; size(3cm); 
pen dps = linewidth(0.7) + fontsize(10); defaultpen(dps);
pen dotstyle = black;
real scale = 1.75;

pair O = (0,0);
pair A = dir(20);
pair pA = dir(70);

draw(O--A);
draw(A--pA);
draw(O--pA);

dot(O,dotstyle); 
label("$O$", O, S*scale);
dot(A,dotstyle); 
label("$\alpha$", A, E*scale);
dot(pA,dotstyle); 
label("$\rho(\alpha)$", pA, NE*scale);
\end{asy}
\end{center}

Whenever $n > 6$, like in the picture above, $\vert \rho(\alpha) - \alpha\vert < \vert \alpha\vert$. By our last theorem $\rho(\alpha)\in L\implies \rho(\alpha)-\alpha\in L$, so this contradicts the minimality of $\vert \alpha\vert$. Therefore, $n \leq 6$. 

\begin{center}
\begin{asy}
import graph; size(4.5cm); 
pen dps = linewidth(0.7) + fontsize(10); defaultpen(dps);
pen dotstyle = black;
real scale = 2.5;

pair O = (0,0);
pair A = dir(0);
pair pA = dir(72);
pair ppA = dir(144);
pair X = ppA+A;

draw(O--A);
draw(O--pA);
draw(O--ppA);
draw(ppA--X);
draw(A--X);

dot(O,dotstyle); 
label("$O$", O, S*scale);
dot(A,dotstyle); 
label("$\alpha$", A, S*scale);
dot(pA,dotstyle); 
label("$\rho(\alpha)$", pA, NE*scale);
dot(ppA,dotstyle); 
label("$\rho(\rho(\alpha))$", ppA, N*scale);
dot(X,dotstyle); 
label("$\alpha + \rho(\rho(\alpha))$", X, E*scale);
\end{asy}
\end{center}

If $\rho$ is a rotation by $2\pi/5$, $0 < \vert \alpha + \rho(\rho(\alpha))\vert < \vert \alpha\vert$, so $n\neq 5$. 
\end{proof}

\comment{add something about frieze groups here}