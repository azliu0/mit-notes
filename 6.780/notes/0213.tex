\section{February 13, 2023}

Bad weather day today. Prof. Wornell tells us about how this is the first time he's given a lecture over zoom since the pandemic. He was hoping that he would never have had to give a zoom lecture again, but here we are. 

\subsection{Review: Bayesian Hypothesis Testing}

Overarching goal: optimize the expected cost of choosing one hypothesis over another. This can be accomplished with the Likelihood Ratio Test: 

\[L(y) \overunderset{\hat{H}_1}{\hat{H}_0}{\gtreqless} \eta \] 

Some issues with this solution: we require some notion of costs, as well as priors $P_i = P_{H}(H_i)$. It can be difficult to assign probabilities to abstract concepts, like $\PP[\text{patient has disease}]$.

\subsection{Non-Bayesian Hypothesis Testing}

The ``folk theorem'' that we will be proving today is that all optimum decision rules takes the form of a Likelihood Ratio Test:

\[\frac{p_{Y|H}(y|H_1)}{p_{Y|H}(y_{H_0})}\overunderset{\hat{H}_1}{\hat{H}_0}{\gtreqless} \eta,\]
for some $\eta$. This is largely true, but not always.

\subsection{General Performance Measures}
Let $\hat{H}(\cdot)$ be any rule. An equivalent characterization of this rule is some partition of the observation space $Y$: 
\begin{align*}
	y_0 &= \{y\in Y : \hat{H}(y) = H_0\} \\
	y_1 &= y\backslash y_0
\end{align*}

Then $P_D = \PP[\hat{H}=H_1|H=H_1] = \int_{y_1}p_{Y|H}(y|H_1)\ddd y$ is called the \ac{detection probability}, and $P_F = \PP[\hat{H}=H_1|H=H_0] = \int_{y_1}p_{Y|H}(y|H_0)\ddd y$ is called the \ac{``false alarm'' probability}. Some related terminology: $P_M = 1-P_D$ is called the ``miss'' probability. 

\begin{itemize}
	\item Statistics terminology: $P_E^1 = P_F$, $P_E^2 = P_M$, ``probability of error of each kind''. The probability of error of the first kind is called the ``size'' of the test, while the probability of error the second kind is called the ``power'' of the test. 
	\item Medical terminology: $P_F$ is the false positive rate, while $P_M$ is the false negative rate. 
	\item Learning / pattern classification: $P_R = P_D$ is the ``recall'' or ``sensitivity''. The ``precision'' is defined as $P_P = \PP[H=H_1|\hat{H}=H_1] = 1 / (1 + P_F/P_D\cdot P_0/P_1)$. 
\end{itemize}

In general, $P_D$ and $P_F$ are conflicting objectives. We seek large $P_D$ and small $P_F$. The bayesian approach to this ``multi-objective'' optimization is to choose the rule that satisfies: 
\[\min_{\hat{H}(\cdot)}(\alpha P_F - \beta P_D).\]

\subsection{Operating Characteristic of the LRT} 
Other tradeoffs are possible. Consider the family of ratio tests: 
\[\{\hat{H}(\cdot) = \text{LRT}, \text{for some }\eta\}.\] 

\begin{example}
\exlabel

Consider two hypotheses
\begin{align*}
	H_0 &: y\sim \Norm(0,\sigma^2) \\
	H_1 &: y\sim \Norm(m,\sigma^2)
\end{align*} 
\end{example}

The LRT is given by 
\[y \overunderset{\hat{H}_1}{\hat{H}_0}{\gtreqless} \frac{m}{2} + \frac{\sigma^2\ln \eta}{m} \overset{\Delta}{=}\gamma.\] 
The false positive rate
\[P_F = \int_{\gamma}^{\infty}p_{Y|H}(y|H_0)\ddd y = Q\left(\frac{\gamma}{\sigma}\right),\] 
and the detection probability 
\[P_D = \int_{\gamma}^{\infty}p_{Y|H}(y|H_1)\ddd y = Q\left(\frac{\gamma - m}{\sigma}\right).\]
Graphing a plot of $(P_F, P_D)$ over different values of $\mu = m/\sigma$ gives 

\input{figures/oclrt_mu.txt}
This curve is called the \ac{OC-LRT}: Operating Characteristic of the Likelihood Ratio Test. 

\[\text{OC-LRT}: \{(P_F, P_D): \hat{H}(\cdot)\text{ is LRT for some }\eta\}.\] 

\begin{theorem}
\claimlabel

OC-LRT is monotonic and non-decreasing. 
\end{theorem}

\begin{proof}
If $\eta_2 > \eta_1$, then $P_D(\eta_2)\leq P_D(\eta_1)$ and $P_F(\eta_2)\leq P_F(\eta_1)$.
\end{proof}
\noindent A few other key properties here: 
\begin{itemize}
	\item All of the frontiers are concave-down, i.e., we should always be ``better'' at maximizing $P_D$ than minimizing $P_F$
	\item As $\mu\rightarrow \infty$, we approach the optimal curve, which is when $P_D$ is $1$ everywhere and $P_F$ is $0$ everywhere. This makes sense intuitively, since $\mu$ large separates the two hypotheses. 
	\item Similarly, when $\mu = 0$, the two hypotheses are indistinguishable, so the frontier is as good as random guessing. 
	\item When $P_F = P_D = 1$ when $\gamma = 0$ and $P_F = P_D = 0$ when $\gamma\rightarrow \infty$. Therefore, as $\gamma$ increases to infinity, we travel from top-right to bottom-left along each of the curves (this will always be the case for any OC-LRT, for this reason). 
\end{itemize}
We will prove some of these next lecture.

\subsection{Neyman-Pearson Criterion}

To avoid the problem of costs and priors, a common alternate criteria choose a rule subject is the \ac{Neyman-Pearson Criterion}: 
\[\max_{\hat{H}(\cdot)}P_D \text{ s.t. }P_F\leq \alpha.\] 
In words, choose the hypothesis with largest detection power given a fixed upper bound on the false alarm size.
\begin{theorem}
\thmlabelname{Neyman-Pearson Lemma, Specialized}

For deterministic $\hat{H}(\cdot)$, a solution to the Neyman-Pearson Criterion is an LRT when the LRT is continuous. In other words, 
\[\hat{H}(y) = H_{\mathbbm{1}_{L(y)\geq \eta}},\] 
where $\eta$ is the smallest threshold s.t. 
\[P_F = \PP(L(y)\geq \eta | H=H_0)\leq \alpha.\] 
\end{theorem}

This statement of NP is considered `specialized' because we are not employing randomization in our hypotheses. We will see and prove the full version next lecture. 

\begin{proof}
We can prove this with lagrange multipliers. Fix $P_F = \alpha' \leq \alpha$. Then, we want to optimize
\begin{align*}
	min_{\hat{H}(\cdot)} \varphi(\hat{H}) &= (1-P_D) + \lambda(P_F-\alpha') \\
	&= \int_{y_0}p_{Y|H}(y|H_1)\ddd y + \lambda \left( \int_{y_1}p_{Y|H}(y|H_0)\ddd y - \alpha' \right) \\
	&= \lambda(1-\alpha') + \int_{y_0}\left(p_{Y|H}(y|H_1) - \lambda p_{Y|H}(y|H_0)\right) \ddd y.
\end{align*}
The min $\varphi$ occurs when we assign $y$ to $y_0$ whenever the the integrand is $\leq 0$, as this minimizes the cost. For the same reason, we want $\alpha'$ to be as large as possible. Therefore, 
\[\frac{p_{Y|H}(y|H_1)}{p_{Y|H}(y|H_0)}\overunderset{\hat{H_1}}{\hat{H_0}}{\gtreqless}\lambda,\]
where $\alpha'$ is chosen to be the largest threshold achievable by LRT. 
\end{proof}

\noindent We also present an alternate proof below.

\begin{proof}
	We compare the LRT decision region with an arbitrary decision region, and show that we cannot do better than the LRT decision region. Let $\mathcal{Y}_1^{\eta}$ be the set of points for which $\hat{H}^{\eta}(y)=H_1$, and $\mathcal{Y}_1$ be the set of points for which $\hat{H}(y)=H_1$ for an arbitrary decision rule. We have 
	\[(\mathbbm{1}(L(y)\geq \eta) - \mathbbm{1}(y\in \mathcal{Y}_1))(L(y)-\eta)\geq 0,\] 
which can be verified with casework. Then, 
\begin{align*}
	&\int_{\mathcal{Y}} (\mathbbm{1}(L(y)\geq \eta) - \mathbbm{1}(y\in \mathcal{Y}_1))(L(y)-\eta)p_{Y|H}(y|H_0) \ddd y \\
	&= \int_{\mathcal{Y}} (\mathbbm{1}(L(y)\geq \eta) - \mathbbm{1}(y\in \mathcal{Y}_1))(p_{Y|H}(y|H_1) - \eta p_{Y|H}(y|H_0))\ddd y\geq 0.
\end{align*}
Some expansion and rearranging gives 
\[\int_{\mathcal{Y}_1^{\eta}}p_{Y|H}(y|H_1)\ddd y - \int_{\mathcal{Y}_1}p_{Y|H}(y|H_1)\ddd y\geq \eta\left(\int_{\mathcal{Y}_1^{\eta}}p_{Y|H}(y|H_0)\ddd y - \int_{\mathcal{Y}_1}p_{Y|H}(y|H_0)\ddd y\right),\]
which (term-by-term) is just 
\[P^{\eta}_D - P_D \geq \eta(P_F^{\eta} - P_F).\] 
When $P_F^{\eta}$ is fixed at $\alpha$, the constraints of the problem force us to pick $P_F\leq P_F^{\eta}$, so $P^{\eta}_D-P_D\geq 0$. But this implies that any hypothesis that is not the LRT can only have worse detection power, hence $\hat{H}^{\eta}_1$ is a valid solution.
\end{proof}

\subsection{Hypothesis Testing in the LRT framework}

The typical LRT criteria:
\[L(y)\overunderset{\hat{H}=H_1}{\hat{H}=H_0}{\gtreqless} \eta.\] 
When we apply a monotonic function to both sides, the criteria remains the same. In particular, when we apply $P_F$, we get
\[p_*(y)\overset{\Delta}{=}P_F(L(y)) \overunderset{\hat{H}=H_0}{\hat{H}=H_1}{\gtreqless} P_F(\eta) = \alpha.\] 
Note that the direction of the hypothesis changed, since $P_F$ is monotonically decreasing in $\eta$. The RHS is the significance level of the test, while $p_*$ is a function that maps each data point to a $p$-value. Roughly speaking, if $p_*$ is large, this means $L$ was small, so the data was not very significant. If $p_*$ is small, this means $L$ was large, so the data was significant. The threshold of "significant" is determined by our threshold $\alpha$. 

