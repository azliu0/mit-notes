\section{September 19, 2023}

\subsection{Ergodic Theorem}

\begin{definition}
\deflabelname{Stopping Time}

A \ac{stopping time} for stochastic process $X_i$ is a random variable $T$ on $\NN\cup \{\infty\}$ such that $T=i$ can be determined by $X_1, \hdots, X_i$. 
\end{definition}

For example, $\tau_z^+$ is a stopping time, since the event $\tau_z^+=i$ occurs only when $X_1, \hdots, X_{i-1}\neq z$ and $X_i = z$. 

\begin{theorem}
\proplabelname{Strong Markov property}

Let $X_i$ be a Markov chain and $T$ be a stopping time for $X_i$. Given $T<\infty$ and $X_T=x$, $(X_{T+i})_{i\in \NN}$ is distributed as $(X_i)_{i\in \NN}$ starting from $X_0=x$.
\end{theorem}

\begin{proof}
For $T$ fixed, this is a restatement of the usual Markov property. Also, since $T$ is a stopping time, fixing $T=n$ depends only on $X_0, \hdots, X_{n}$; therefore, by time homogeneity, this statement is also true conditioned on $T=n$ and $X_n=x$. Since this is true for all $n$, we are done. 
\end{proof}

For Markov chain $X_i$, let $V_x(n)$ be the number of visits to $x$ before time step $n$. 

\begin{theorem}
\thmlabelname{Ergodic Theorem}

Let $P$ be irreducible with $\vert \mathcal{X}\vert < \infty$. Then, 
\[\frac{V_x(n)}{n}\cas \frac{1}{\EE[\tau_x^+]},\]
and for any function $f: \mathcal{X}\rightarrow \RR$, 
\[\frac{\sum_{i=0}^{n-1}f(X_i)}{n}\cas \ov{f},\]
where $\ov{f} = \sum_x \pi(x)f(x)$.
\end{theorem}

\begin{proof}
	Fix $z\in \mathcal{X}$. Let $T_i$ be the $i$th time that $z$ is visited. Then, $T_{i+1}-T_i$ are independent and identically distributed by the Strong Markov property. Since $T_{V_z(n)}\leq n$ and $T_{V_z(n)+1}\geq n$, 

\[\frac{T_{V_z(n)}}{V_z(n)}\leq \frac{n}{V_z(n)} \leq \frac{T_{V_z(n)+1}}{V_z(n)}.\]

Let $S_n = 1/(n-1) \cdot \sum_{i=1}^{n-1}(T_{i+1} - T_i)$. By SLLN, $S_n\cas \EE[\tau_z^+]$. But note that 
\[S_{V_z(n)+1} = \frac{(T_{V_z(n) + 1}-T_1)}{V_z(n)} \rightarrow \frac{T_{V_z(n) + 1}}{V_z(n)} \geq \frac{n}{V_z(n)},\] 
while 
\[S_{V_z(n)} = \frac{(T_{V_z(n)} - T_1)}{(V_z(n) - 1)}\rightarrow \frac{T_{V_z(n)}}{V_z(n)} \leq \frac{n}{V_z(n)},\] 
so $n / V_z(n)\cas \EE[\tau_z^+]$, as desired.

For the second part,
\begin{align*}
	\frac{1}{n}\sum_{i=0}^{n-1}f(X_i) &= \frac{1}{n}\sum_{x\in \mathcal{X}}V_x(n)f(x) \\
																		&= \sum_{x\in \mathcal{X}}\left(\frac{V_x(n)}{n} - \frac{1}{\EE[\tau_x^+]}\right)f(x)  + \sum_{x\in \mathcal{X}} \frac{f(x)}{\EE[\tau_x^+]} \\
																		&\cas \ov{f},
\end{align*}
by the first part. 
\end{proof}



