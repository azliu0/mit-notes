\section{February 15, 2024}

\subsection{Randomizing Neyman-Pearson}

First, some intuition on why randomizing can be beneficial. Imagine any scenario involving a discrete process such as a Poisson process. In this case, the points on the OC-LRT are discontinuous, such as in the following diagram: 

\input{figures/oclrt_1.txt}

\begin{theorem}
\thmlabelname{Neyman-Pearson Lemma}

Given hypotheses $p_{Y|H}(y|H_0)$,$p_{Y|H}(y|H_1)$ and $\alpha\in [0,1]$, for some $\eta$ and $p\in [0,1]$ there exists rule of the form 

\[
q(y) = \begin{cases}
	0 & L(y) < \eta \\
	p & L(y) = \eta \\
	1 & L(y) > \eta
\end{cases},
\] 
where $P_F(q_*) = \alpha$ and $P_D(q_*)\geq P_D(q)$ for any decision rule $q$ satisfying $P_F(q)\leq \alpha$.
\end{theorem}
Note that this is essentialy the same as the deterministic statement, except that we have generalized to randomize the decision in case of ties. 
