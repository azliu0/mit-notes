\section{November 4, 2022}

\subsection{Proving Sylow's Theorem}

Recap from last time:

\begin{definition}
\deflabel

Let $G$ be finite group and $p$ a prime number such that $p^k\mid \vert G\vert$ but $p^{k+1}\nmid \vert G\vert$. Then, a \ac{Sylow $p$-subgroup} of $G$ is a subgroup of order $p^k$.

\end{definition}

\begin{theorem}
\thmlabelname{Sylow's Theorem}

For any finite group $G$ and prime $p$, 
\begin{enumerate}
    \item [(1)] $G$ has a Sylow $p$-subgroup.
    \item [(2)] All the Sylow $p$-subgroups in $G$ are conjugate.
    \item [(3)] Every $p$-subgroup (any subgroup whose order is a power of $p$) is contained in some Sylow $p$-subgroup. 
    \item [(4)] The number of Sylow $p$-subgroups is $1\mod{p}$ and divides $\vert G\vert / p^k$.
\end{enumerate}
\end{theorem}

Today, we'll prove all four parts of this theorem. Proving this theorem is good culmination of everything that we have learned so far about group actions in this course.

Let's start with (1), which we proved on the problem set. We'll take a different approach here.
\begin{proof}
Consider the action of $G$ by left multiplication on subsets (not necessarily subgroups) of $G$ of size $p^k$. Suppose we find some subset $U\subseteq G$ with $\vert U\vert = p^k$ such that the orbit size is not divisible by $p$.  

If so, let $H = \stab_{G}(U) = \{h\in G: hU = U\}$. Then $H$ is a p-group (i.e., its order is a power of $p$). We have that $Hu\in U\forall u\in U$, so $U$ is partitioned into right cosets of $H$. Since every right coset has the same size ($\vert H\vert$), the size of $H$ must divide the size of $U$, so $H$ is a p-group. 
We also have $\vert G\vert = \underbrace{\vert GU\vert}_{v_p=0} \vert H\vert\implies p^k\mid \vert H\vert$, so $H$ is a Sylow $p$-subgroup. 

Now we are left to show that $U$ exists. The number of subsets of size $p^k$ is equal to $\binom{mp^k}{p^k}$. Since the orbits partition this set of subsets, if this quantity isn't divisible by $p$, some orbit size isn't, so we are left to show that this is not divisible by $p$.

\begin{theorem}
\lemlabel

\[\binom{pa}{pa}\equiv \binom{a}{b}\pmod{p}.\]
\end{theorem}

\begin{proof}
By binomial expansion, 
\[\binom{pa}{pb} = [x^{pb}] (1+x)^{pa}\]

We know that $(1+x)^p = 1+x^p\pmod{p}$, since $\binom{p}{i}\equiv 0\pmod{p}$ for all $1\leq i\leq p-1$. 

Therefore, 
\[[x^{pb}] (1+x)^{pa} \equiv [x^{pb}] (1+x^p)^a = \binom{a}{b}.\]
\end{proof}

Applying the lemma $k$ times, 
\[\binom{mp^k}{p^k}\equiv m\pmod{p},\]
but $(p,m)=1$, so we're done.
\end{proof}

Now we'll prove (2) and (3) together, by showing the following the following reformulation:
\begin{theorem}
\lemlabel 

Let $H$ be a Sylow $p$-subgroup of $G$. Let $K$ be any $p$-subgroup. Then $K\subseteq gHg^{-1}$ for some $g\in G$.  
\end{theorem}

This is enough to show (2), because $K$ can be any Sylow $p$-subgroup. This is also enough to show (3), because $gHg^{-1}$ has the same size as $H$, so $gHg^{-1}$ is also a Sylow $p$-subgroup. Note that we know $H$ exists by (1). 

\begin{proof}
Consider the action of $K$ by left multiplication on the left cosets of $H$, i.e., $k\cdot gH = kgH$. It turns out that proving this lemma amounts to proving that there exists a fixed point in $K$.

We know $p \nmid \vert G/H\vert = m$, and we also know that every orbit size divides $\vert K\vert = p^{\ell}$. Since the orbits partition $\vert G/H\vert$, there must be an orbit of size $1$, i.e., there exists a coset $gH$ such that $kgH = gH$ for all $k\in K$. 

$kgH = gH\forall k\in K \iff g^{-1}kgH = H\forall k\in K \iff g^{-1}kg\in H\forall k\in K \iff g^{-1}Kg\subseteq H \iff K\subseteq gHg^{-1}$, so we're done.
\end{proof}

Finally, let's prove (4). 

\begin{proof}
Let $X$ be the set of Sylow $p$-subgroups. We want to show that $\vert X\vert \mid m$, and $\vert X\vert \equiv 1\pmod{p}$. By (2), $X$ is a single orbit under conjugation by $G$. 

By the orbit-stabilizer theorem,
\[p^km = \vert G\vert = \vert X\vert \vert N(H)\vert,\]
for some $H\in X$, where the normalizer of $H$
\[N(H) = \stab_G(H) = \{g\in G : gHg^{-1} = H\}.\]

But $H$ is a subgroup of $N(H)\implies p^k = \vert H\vert \mid \vert N(H)\vert$. Thus, $m = \vert X\vert \cdot \left(\vert N(H)\vert / p^k\right)$, so $\vert X\vert$ divides $m$. 

Now, consider the conjugation action of $H$ on $X$. There could be many orbits, since we are restricting our first action to elements of $H$, rather than all elements in $G$. All orbit sizes are powers of $p$, since they must divide $\vert H\vert$. Note that $H$ itself has an orbit of size $1$. We claim that this is the only orbit of size $1$, which is enough to imply $\vert X\vert \equiv 1\pmod{p}$.

\begin{theorem}
\lemlabel

$\forall H, H'\in X$, $H\subseteq N(H')\iff H'=H$ $\iff\{H'\}$ is an $H$ orbit.  
\end{theorem}
\begin{proof}
$H, H'$ are both Sylow $p$-subgroups of the normalizer of $H'$. By (2), there exists $n\in N(H')$ s.t. $H = nH'n^{-1}$. By definition, the normalizer fixes $H'$, so $H = H'$. 
\end{proof}

For any $H'\in X$ that has an orbit of size $1$, $H\subseteq N(H')$, since all elements in $H$ are fixed points for $H'$. Thus, the lemma implies $H'=H$, so we're done.
\end{proof}

\subsection{Addendum: simple groups with order $60$}

Addendum: the following was not covered in lecture, but I wanted to include it anyways, because it feels like a solid way to wrap up our discussions on orbits, stabilizers, and the Sylow's Theorem. 

\begin{theorem}
\thmlabel 

The only simple group with order $60$ is $A_5$. 
\end{theorem}

This proof extends the logic that we used in Theorem 20.7, where we proved that $A_5$ was simple.

\begin{proof}
Let $G$ be a simple group with order $60$. Our goal is to show that $G\cong A_5$. 

First, suppose that $G$ acts non-trivially on some set $S$ with order less than order to $5$. By the same logic that we used in our proof of Theorem $20.7$, this implies an injective homomorphism from $G$ to the permutations of $S$, which is possible if and only if $\vert S\vert = 5$, in which case $G\cong A_5$. 

Now, we go about constructing $S$. Assume for the sake of contradiction that $G\not\cong A_5$. By Sylow's Theorem, $n_2\mid 15\implies n_2\in\{1,3,5,15\}$. If $n_2=1$, then the unique Sylow $2$-subgroup is normal, which breaks the assumption that $G$ is simple. If $n_2=3$, or $n_2=5$, then the conjugation action from $G$ onto the set of Sylow $2$-subgroups is well-defined (Sylow groups are closed under conjugation, by Sylow's theorem), and certainly non-trivial. Thus, our argument in the previous paragraph breaks the assumption that $G\cong A_5$. We can use a similar line of reasoning to show that $n_3\equiv 1\pmod{3}$ and $n_3\mid 20 \implies n_3=10$. 

Since all Sylow $3$-subgroups are cyclic, they all intersect trivially. Now, suppose all Sylow $2$-subgroups intersected trivially. If so, then the number of non-identity elements contained in the union of these subgroups is $3\cdot 15=45$. But the number of non-identity elements contained in the unions of the Sylow $3$-subgroups is $2\cdot 10 = 20$, in which case the total number of non-identity elements exceeds the size of $G$. 

Therefore, there exists distinct $H$, $H'\in G$ which are Sylow $2$-subgroups and intersect non-trivially. Denote $G^*$ the subgroup of $G$ generated by the elements in $H\cup H'$. Since $H$ is a strict subgroup of $G^*$, $\vert G^*\vert = 4k$ for some $k > 1$. Since we also have $\vert G^*\vert \mid \vert G\vert$, $k\in \{3,5,15\}$. 

Finally, let $x\in H\cap H'-\{1\}$. Since $\vert H\vert = \vert H'\vert = 4$, they are abelian, and $x$ commutes with everything in both groups. This implies that $x$ commutes with everything in $G^*$, so $\gen{x}$ is a normal subgroup in $G^*$. If $k=15$, then $G^*=G$, so this is not possible. Therefore, $k\in\{3,5\}$. In either case, the action from $G$ onto $G/G^*$ defined by $(g_1, g_2G^*)\mapsto g_1g_2G^*$ is well-defined and non-trivial, so we are done by our first paragraph.  
\end{proof}