\section{October 5, 2022}

\subsection{Cauchy-Schwarz}

\begin{definition}
\deflabel

The \ac{inner product} or \ac{dot product} of any two vectors $x,y\in \RR^n$ is given by 
\[\gen{x,y} = x\cdot y = x^ty = x_1y_1+\hdots+x_ny_n\]
\end{definition}

Using our definition of inner product (does not necessarily have to be a dot product), we can define a few more quantities. 

\begin{definition}
\deflabel

The \ac{length} of a vector $x$ is given by $\vert x\vert = \sqrt{\gen{x,x}}$.
\end{definition}

\begin{definition}
\deflabel

The \ac{distance} between two vectors $x,y$ is given by $\vert x-y\vert$. 
\end{definition}

Like the usual dot product, we would like if $\gen{x,y} = \vert x\vert \vert y\vert \cos{\theta}$, for some meaning of $\theta$ in our vector space. What happens if we view this definition as an angular measurement in and of itself?

Well, this should work as long as we don't have any domain errors. If $x,y\neq 0$, is it true that 
\[\left\vert\frac{\gen{x,y}}{\vert x\vert\vert y\vert}\right\vert \leq 1?\]
The answer is yes!

\begin{theorem}
\thmlabelname{Cauchy-Schwarz Inequality}

\[\vert \gen{x,y}\vert \leq \vert x\vert \vert y\vert \quad \forall x,y\in \RR^n\]
\end{theorem}

\begin{proof}
We start with 
\[\vert x-\lambda y\vert^2 \geq 0\quad \forall \lambda \in \RR.\]

Expanding, 
\begin{align*}
    \vert x-\lambda y\vert^2 &= \gen{x-\lambda y, x-\lambda y} \\
    &= \vert x\vert^2 - 2\lambda \gen{x,y} + \lambda^2\vert y\vert^2 \geq 0.
\end{align*}

Taking the discriminant implies the result. 
\end{proof}

\subsection{Orthogonal Matrices}

\begin{definition}
\deflabel

A basis $x_1, x_2, \hdots, x_n\in \RR^n$ is \ac{orthogonal} if $\gen{x_i, y_j} = 0$ for $i\neq j$, and \ac{orthonormal} if it additionally satisfies $\vert x_i\vert = 1\quad \forall i$. 
\end{definition}

\begin{definition}
\deflabel

A matrix $A\in \RR^{n\times n}$ is \ac{orthogonal} if 
\[\gen{Ax, Ay} = \gen{x,y}\quad \forall x,y\in \RR^n\]
\end{definition}

\begin{theorem}
\thmlabel

For $A\in \RR^{n\times n}$, the following are equivalent: 
\begin{itemize}
    \item [(1)] $A$ is orthogonal
    \item [(2)] $\vert Ax\vert = \vert x\vert\quad \forall x\in \RR^n$ 
    \item [(3)] $A^tA = I_n$
    \item [(4)] The columns of $A$ are orthonormal. 
    \item [(5)] The rows of $A$ are orthonormal. 
\end{itemize}
\end{theorem}

Fun fact: IOAA 2022 DA Q2 was basically just (3), Andrew didn't know linear algebra so he mindlessly bashed out inverses and wasted all his time, sad. 

\begin{proof}
$(1)\implies (2)$: When we set $x=y$ in the definition of orthogonality, $\gen{Ax, Ax} = \gen{x,x}\iff \vert Ax\vert^2 = \vert x\vert^2$. 

$(2)\implies (1)$: We can express the inner product in terms of length (this technique is called \ac{polarization}), by 
\[\gen{x,y} = \frac{\vert x+y\vert^2 - \vert x\vert^2 - \vert y\vert^2}{2}.\]
Therefore, if lengths are preserved, then so are inner products. 

$(3)\implies (1)$: We have $\gen{x,y} = x^ty$. On the other hand, $\gen{Ax, Ay} = (Ax)^tAy = x^tA^tAy = x^ty\iff A^tA = I_n$. 

$(1)\implies (3)$: Let $e_1, \hdots, e_n$ be the standard basis. Then \[\gen{e_i, e_j} = \delta_{ij} = \begin{cases}1 & i=j \\ 0 & i\neq j\end{cases}.\]

($\delta_{ij}$ is called the \ac{kronecker delta}.) So we have 
\[\delta_{ij} = \gen{Ae_i, Ae_j} = e_i^t(A^tA)e_j = (A^tA)_{ij},\]
thus $A^tA$ is the identity.

$(3)\iff (4)$: 
\[A^t = \begin{pmatrix}
    \text{---} & a_1 & \text{---} \\
     & \vdots & \\
    \text{---} & a_2 & \text{---}
\end{pmatrix}, \quad A = \begin{pmatrix}
    \vert & & \vert \\
    a_1   & \hdots & a_n   \\
    \vert & & \vert
\end{pmatrix}\]
$(A^tA)_{ij} = a_i^ta_j$, so $A^tA = I_n\iff a_i^ta_j = \delta_{ij}\iff a_1, a_2, \hdots, a_n$ are orthonormal. 

The proof for $(3)\iff (5)$ is analogous, so we are done. 
\end{proof}

\begin{definition}
\deflabel

The \ac{orthogonal group}
\[O_n(\RR) = \{A\in \RR^{n\times n} : A^tA = I_n\}\subset GL_n(\RR)\]
\end{definition}

\begin{definition}
\deflabel

The \ac{special orthogonal group}
\[SO_n(\RR) = \{A \in O_n(\RR) : \det A = 1\}\]
\end{definition}

Note: $\det (A^t)\det (A) = 1 \implies \det A = \pm 1$, so $SO_n(\RR)$ has index $2$ in $O_n(\RR)$. 




