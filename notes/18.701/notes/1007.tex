\section{October 7, 2022}

\subsection{Characterizing $O_n(\RR)$}

Last lecture, we defined the orthogonal group of matrices over $\RR$ as the set of matrices that preserves length and inner product. Today, we're going to look at some examples of $O_n(\RR)$, and classify the isometries of $\RR^n$. 

\begin{example}
\exlabel 

$n=1$.
\end{example}

In this case, $O_1(\RR) = \{\pm I_1\}$. 

\begin{example}
\exlabel

$n=2$. 
\end{example}

We have 
\[O_2(\RR) = \left\{\twotwo{a}{b}{c}{d} : \vtwo{a}{c}, \vtwo{b}{d}\text{ orthonormal}\right\}.\]

Since $a^2+c^2=1$, we may paramaterize $a = \cos{\theta}$ and $c = \sin{\theta}$. By the orthogonal property, this implies $b = \pm \sin{\theta}$ and $d = \mp \cos{\theta}$. 

This gives us 
\[SO_2(\RR) = \left\{\twotwo{\cos{\theta}}{-\sin{\theta}}{\sin{\theta}}{\cos{\theta}}: \theta\in \RR\right\},\]
which is the set of rotations. The other coset is
\[\left\{\twotwo{\cos{\theta}}{\sin{\theta}}{\sin{\theta}}{-\cos{\theta}}: \theta\in \RR\right\}.\]
We claim these are the set of reflections. The characteristic equation here is $\lambda^2=1\implies \lambda = \pm 1$. So, there exists $v,w\in \RR^2$ with $Av=v$ and $Aw=-w$, in which case 
\[\gen{v,w} = \gen{Av, Aw} = \gen{v,-w}\implies \gen{v,w}=0.\]

This implies that $A$ is a reflection through the line through $v$. We say that matrices in $O_2(\RR)$ are \ac{orientation-preserving} if their determinant is $1$, and \ac{orientation-reversing} if their determinant is $-1$. 

A key fact is that $SO_2(\RR)$ consists only of rotations. When we try to multiply two elements in the other coset, we get a composition of two reflections. Since reflections preserve distance to the origin, this is equivalent to a rotation, so this agrees with our analysis.

We'll upgrade our next example to a theorem, since its an important (and not necessarily intuitive) result. 

\begin{theorem}
\thmlabel

Every element of $SO_3(\RR)$ is a rotation about some axis in $\RR^3$.
\end{theorem}

First, we prove a lemma. 

\begin{theorem}
\lemlabel

Every matrix $A\in SO_3(\RR)$ has an eigenvalue of $1$. 
\end{theorem}

\begin{proof}
We want to show that $\det(A-I_3) = 0$. We know $AA^t=I_3$, so $\det(A) = \det(A^t) = 1$. Then, 
\begin{align*}
    \det(A-I_3) &= \det(A-AA^t) \\
    &= \det(A)\det(I_3-A^t) \\
    &= \det((I_3-A)^t) \\
    &= \det(A-I_3)\cdot (-1)^3,
\end{align*}
which is enough to imply our result.
\end{proof}

Now, we are ready to prove Theorem 13.3. 

\begin{proof}
By our lemma, there exists $x\neq 0$ such that $Ax=x$; in other words, $A$ fixes some pole in $\RR^3$. Without loss of generality, let $\vert x\vert = 1$. Let $B$ be the basis matrix formed when we extend $x$ to an orthonormal basis $x,y,z$ of $\RR^3$. Then, under our new basis,

\[A\mapsto B^{-1}AB = \begin{pmatrix}
1 & 0 & 0 \\
0 & a & b \\
0 & c & d
\end{pmatrix}.\]

Note that changing our basis does not disturb the determinant, since $\det B\det B^{-1}=1$. To preserve orthonormality, $\twotwo{a}{b}{c}{d}\in SO_2(\RR)$ is a rotation about our fixed pole, so we're done. 
\end{proof}

Interesting things to note: Lemma 13.4 holds for any $n$ odd, since all logic holds as long as we are flipping the sign for an odd number of rows in the last equality. Also, nicer characterizations of $SO_n(\RR)$ when $n\geq 4$ are more difficult. For example, we cannot necessarily guarantee single rotations, because 
\[\begin{pmatrix}
\cos{\alpha} & -\sin{\alpha} & 0 & 0\\
\sin{\alpha} & \cos{\alpha} & 0 & 0\\
0 & 0 & \cos{\beta} & -\sin{\beta}\\
0 & 0 & \sin{\beta} & \cos{\beta}\\
\end{pmatrix}\in SO_4(\RR),\]
which performs two rotations simultaneously in orthogonal subspaces at the same time.

\subsection{Isometries of $\RR^n$}

\begin{definition}
\deflabel

$f: \RR^n\rightarrow \RR^n$ is an \ac{isometry} if $f$ preserves distances. In other words, 
\[\vert f(x)-f(y)\vert = \vert x-y\vert\quad \forall x,y.\]
\end{definition}

For example, $f(x) = Ax$ for $A\in O_n(\RR)$ is an isometry. Also, $f(x) = x+b$ for $b\in \RR^n$ is an isometry. It turns out that the composition of these two generates all possible isometries. 

\begin{theorem}
\thmlabel

Every isometry of $\RR^n$ is of the form $x\mapsto Ax+b$ where $A\in O_n(\RR)$, $b\in \RR^n$. 
\end{theorem}

\begin{proof}
We may assume $f(0)=0$, since $x\mapsto x-f(0)$ is an isometry.

First, we show that $f$ necessarily preserves inner products. We can do this by using the same polarization technique that we used last lecture: 
\begin{align*}
    \gen{x,y} &= \frac{\vert x-0\vert ^2 + \vert y-0\vert ^2 - \vert x-y\vert^2}{2} \\
    &= \frac{\vert f(x)-f(0)\vert ^2 + \vert f(y)-f(0)\vert ^2 + \vert f(x) - f(y)\vert ^2}{2} = \gen{f(x), f(y)}.
\end{align*}

Next, we show that $f$ is linear, i.e., $f(x+y) = f(x)+f(y)\quad \forall x,y\in \RR^n$. 
\begin{align*}
    &\qquad\qquad  z = x+y\\
    &\iff \vert z-x-y\vert^2 = 0 \\
    &\iff \gen{z,z} + \gen{x,x} + \gen{y,y} - 2\gen{x,z} - 2\gen{y,z} + 2\gen{x,y} = 0 \\
    &\iff \vert f(z) - f(x) - f(y)\vert ^2 = 0\\
    &\iff f(x+y) = f(x) + f(y),
\end{align*}
where the transition from the third to fourth line is justified by the fact that $f$ preserves inner products (so we can replace $a\mapsto f(a)$ everywhere). 

Finally, we show that $f$ preserves scalar multiplication, i.e., $f(\lambda x) = \lambda f(x)$. The proof here is exactly the same as the proof for showing linearity. 
\begin{align*}
    &\qquad\qquad  y = \lambda x\\
    &\iff \vert y-\lambda x\vert^2 = 0 \\
    &\iff \gen{y,y} - 2\lambda\gen{y, x} + \gen{y,y} = 0 \\
    &\iff \vert f(y) - \lambda f(x)\vert ^2 = 0\\
    &\iff f(y) = \lambda f(x).
\end{align*}

Since $f$ is linear and preserves scalar multiplication, $f$ is a linear operator. Moreover, $f$ preserves inner products, so we must have $f\in O_n(\RR)$, and we're done. 
\end{proof}

\begin{definition}
\deflabel

\ac{$\mathbf{M_n}$} is the group of isometries of $\RR^n$.
\end{definition}

As a caveat, Prof. Cohn notes that this notation technically is not standardized, its just what Artin uses. 