\section{September 14, 2022}

\subsection{Kernels and images}

Every homomorphism $f: G\rightarrow H$ has two subgroups associated with it: 

\begin{definition}
\deflabel

The \ac{kernel} of $f$ is 
\[\ker(f) = \{g\in G : f(g)=1\}.\]
\end{definition}

\begin{definition}
\deflabel

The \ac{image} of $f$ is 
\[\im(f) = \{h\in H : \exists g\in G \text{ s.t. } f(g)=h\}\]
\end{definition}

Unlike isomorphisms, homomorphisms lose information. The kernel can roughly be thought of as what information we lost under the mapping. The image is how much of $H$ we're actually hitting. 

\begin{theorem}
\proplabel

$f$ is injective if and only if $\ker(f)=\{1\}$. 
\end{theorem}
\begin{proof}
\[f(g)=f(g')\iff f(gg'^{-1})=1\iff gg'^{-1}\in \ker(f).\]
If $f$ is injective, then $g=g'$, so $\ker(f)=\{1\}$. On the other hand, if the kernel was trivial, then $g=g'$, so $f$ is injective. 
\end{proof}

The above proposition shows that $f$ is an isomorphism if and only if $\ker(f)=\{1\}$ and $\im(f)=H$. Let's look at some examples of kernels and images for specific groups. 

\begin{example}
\exlabel
\[\text{exp}: \CC\rightarrow \CC^{x}.\]
\end{example}

$\im(\text{exp})=\CC^{x}$, and $\ker(\text{exp}) = 2\pi i\ZZ$. 

\begin{example}
\exlabel
\[\det: \GL_n(\RR)\rightarrow \RR^x.\]
\end{example}

$\im(\det) = \RR^x$ (recall the general linear group only contains invertible matrices, which is what allows it to be a group in the first place). $\ker(\det) = \SL_n(\RR)$.

\begin{example}
\exlabel
\[\text{sgn}:S_n\rightarrow{\pm 1}.\]
\end{example}

$\im{(\text{sgn})}=\{\pm 1\}$, and $\ker(\text{sgn})=A_n$. This is called the \ac{alternating group}, and contains all permutations with an even number of swaps (transpositions). We'll deal more with this group later in the semester. 

\begin{example}
\exlabel

Given any group $G$ and $g\in G$, define the mapping $f:\ZZ\rightarrow G$ by $f(n)=g^n$. 
\end{example}

It is easy to show that $f$ is a homomorphism. $\im(f)=\gen{g}$, or the subgroup generated by $g$. $\ker(f)=k\ZZ$ if $g$ has finite order $k$. Otherwise, $\ker(f) = \{0\}$. 

\subsection{Cosets}

Consider $G=\ZZ$. For any $k\ZZ\subseteq \ZZ$, there is a nice way to partition $G$ with different translations $k\ZZ$: 
\begin{align*}
    \ZZ &= (2\ZZ)\cup (1+2\ZZ), \\ 
    \ZZ &= (3\ZZ)\cup (1+3\ZZ)\cup (2+3\ZZ), \\
    &\text{etc.}
\end{align*}
This generalizes for non-abelian groups. 
\begin{definition}
\deflabel

For any group $G$ and subgroup $H$, a \ac{left coset} of $H$ in $G$ is a set 
\[gH = \{gh : h\in H\}\] 
with $g\in G$. 
\end{definition}

Right cosets are defined analogously. \textbf{In general, cosets are not subgroups}. This is because $gH$ does not contain the identity unless $g\in H$ (in which case $g^{-1}\in H$, so $gg^{-1}=1$). 

We'll work with left cosets, but all of the following results also hold for right cosets. It turns out the cosets partition groups in exactly the way that we were looking for earlier. 

\begin{theorem}
\proplabel

All cosets have the same size.
\end{theorem}

\begin{proof}
There is a one-to-one correspondence between $H$ and $gH$; namely, multiply all elements in $H$ by $g$ to get $gH$, and multiply all elements in $gH$ by $g^{-1}$ to get $H$, so every coset has the same size as $H$ itself. 
\end{proof}

\begin{theorem}
\proplabel

Every element of $G$ is in some coset.
\end{theorem}

\begin{proof}
$g\in gH$, since $1\in H$. 
\end{proof}

\begin{theorem}
\proplabel

If $gH\cap g'H\neq \emptyset$, then $gH=g'H$. 
\end{theorem}

\begin{proof}
If $gh=g'h'$ for some $h,h'\in H$, then $g=g'h'h^{-1}\implies gH = g'h'h^{-1}H=g'H$. 
\end{proof}

Together, the above three propositions show that we can partition $G$ into cosets of equal size. 

\subsection{Lagrange's Theorem}

\begin{definition}
\deflabel

For any group $G$ and subgroup $H$, we define the \ac{index} of $H$ in $G$ as
\[[G:H] = \text{left cosets of }H\text{ in }G.\]
\end{definition}

The index can be infinite. For example $[2\ZZ:6\ZZ]=3$, while $[\ZZ:\{0\}]=\infty$.

\begin{theorem}
\thmlabelname{Lagrange's Theorem}
\[\vert G\vert = \vert H\vert\cdot [G:H].\]
\end{theorem}
This encapsulates the partition idea that we uncovered with respect to the cosets of $H$ in $G$. An important thing to note here is that whenever $H$ is a subgroup of $G$, $\vert H\vert$ divides $\vert G\vert$, which allows us to confirm one of the observations that we made in Lecture 2: 

\begin{theorem}
\corlabel

If $\vert G\vert = p$ with $p$ prime, then $G\cong C_p$.
\end{theorem}
\begin{proof}
Let $g\in G$, with $g\neq 1$. Then $\vert \gen{g}\vert>1$, so $\vert \gen{g}\vert=p$ by Lagrange. This shows that $G$ has one generator, so it must be $C_p$.
\end{proof}

Back to the idea of the kernel representing what information is lost under any homomorphism. Why? Consider a homomorphism $f: G\rightarrow H$, with $\ker(f)=K$. Then,
\[f(g)=f(g') \iff f(g^{-1}g')=1\iff g^{-1}g'\in K\iff g^{-1}g'K=K\iff g'K=gK,\]
so elements in the same coset of $K$ in $G$ map to the same thing. When $K$ is large, lots of elements map to the same thing, so information is lost, and vice versa. In fact, when $\vert K\vert=1$, the mapping is injective (Proposition 4.3), so no information is lost. 

\comment{add diagram}

