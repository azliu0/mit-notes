\section{November 3, 2023}

\subsection{Forward Diffusion}

In diffusion, we fix a forward process that adds Gaussian noise to an image. We then use a reverse de-noising process to reverse this process and generate images from noise. 

More specifically, we start with some data $x_0$ sampled from distribution $q(x)$. Then, we define a forward diffusion process 
\[q(x_t|x_{t-1}) = \Norm(x_t; \sqrt{1-\beta_t}x_{t-1}, \beta_t I),\]
where the probability of the entire process up to time $T$ is 
\[q(x_{1:T}|x_0) = \prod_{t=1}^T q(x_t|x_{t-1}).\]
At each time step, we're injecting a bit of noise into the image. By the end of the forwards process, $x_T$ is isotropic (pure noise). Usually, $\beta_1 < \beta_2 <\hdots < \beta_T$ with some scheduling process (linear, cosine) to ensure that this is true. Using nice properties of Gaussians, we can sample any timestep directly instead of having to simulate the entire process each time. 

\begin{theorem}
\claimlabel

Let $\alpha_t = 1-\beta_t$ and $\ov{\alpha}_t = \prod_{i=1}^t \alpha_i$, and $\varepsilon\sim \Norm(0,I)$. Then, 
\[x_t = \sqrt{\ov{\alpha}_t} x_0 + \sqrt{1-\ov{\alpha}_t}\varepsilon.\] 
\end{theorem}

\begin{proof}
We can do this with induction. This clearly holds for $t=1$. Now, for arbitrary $t>1$, 
    \begin{align*}
        x_t &= \sqrt{\alpha_t}x_{t-1} + \sqrt{1-\alpha_t}\varepsilon_t \\
        &= \sqrt{\alpha_t}\sqrt{\ov{\alpha}_{t-1}}x_0 + \sqrt{\alpha_t - \alpha_t \ov{\alpha}_{t-1}}\varepsilon + \sqrt{1-\alpha_t}\varepsilon_t \\
        &= \sqrt{\ov{\alpha}_t}x_0 + \sqrt{1-\ov{\alpha}_t}\varepsilon,
    \end{align*}
    where the last equality comes from linearity of variance for independent gaussian noise. This completes the induction.   
\end{proof}

This result shows that we can think of the image at timestep $t$ as a linear combination of pure noise and the original image, where the proportion assigned to pure noise approaches $1$ as $t\rightarrow T$. The graph below visualizes linear vs cosine scheduling: 

\begin{center}
\begin{asy}
 /* Geogebra to Asymptote conversion, documentation at artofproblemsolving.com/Wiki go to User:Azjps/geogebra */
import graph; size(8cm,8cm,keepAspect=false); 
real labelscalefactor = 0.5; /* changes label-to-point distance */
pen dps = linewidth(0.7) + fontsize(15); defaultpen(dps); /* default pen style */ 
pen dotstyle = black; /* point style */ 
real xmin = -246.73728220668832, xmax = 1408.1500640297409, ymin = -0.12479539420646454, ymax = 1.214074230171313;  /* image dimensions */
pen ccqqqq = rgb(0.8,0,0); pen qqwuqq = rgb(0,0.39215686274509803,0); 
Label laxis; laxis.p = fontsize(10); 
xaxis(xmin, xmax-200, EndArrow(6), above = true); 
yaxis(ymin, ymax-0.1, EndArrow(6), above = true); /* draws axes; NoZero hides '0' label */ 
 /* draw figures */draw( (0,0.9996219999999528) -- (0,0.9996219999999528),ccqqqq+linewidth(0.7));
draw( (0,0.9996219999999528) -- (2.5000001537075365,0.9996069952097075),ccqqqq+linewidth(0.7));
draw( (2.5000001537075365,0.9996069952097075) -- (5.000000153155107,0.9995636937539768),ccqqqq+linewidth(0.7));
draw( (5.000000153155107,0.9995636937539768) -- (7.500000152602678,0.999492366063429),ccqqqq+linewidth(0.7));
draw( (7.500000152602678,0.999492366063429) -- (10.000000152050248,0.999393275058656),ccqqqq+linewidth(0.7));
draw( (10.000000152050248,0.999393275058656) -- (12.500000151497819,0.9992666763216992),ccqqqq+linewidth(0.7));
draw( (12.500000151497819,0.9992666763216992) -- (15.00000015094539,0.999112818265427),ccqqqq+linewidth(0.7));
draw( (15.00000015094539,0.999112818265427) -- (17.50000015039296,0.9989319423007809),ccqqqq+linewidth(0.7));
draw( (17.50000015039296,0.9989319423007809) -- (20.00000014984053,0.9987242830018993),ccqqqq+linewidth(0.7));
draw( (20.00000014984053,0.9987242830018993) -- (22.500000149288102,0.9984900682691308),ccqqqq+linewidth(0.7));
draw( (22.500000149288102,0.9984900682691308) -- (25.000000148735673,0.9982295194899515),ccqqqq+linewidth(0.7));
draw( (25.000000148735673,0.9982295194899515) -- (27.500000148183243,0.9979428516977971),ccqqqq+linewidth(0.7));
draw( (27.500000148183243,0.9979428516977971) -- (30.000000147630814,0.9976302737288231),ccqqqq+linewidth(0.7));
draw( (30.000000147630814,0.9976302737288231) -- (32.50000014707838,0.997291988376602),ccqqqq+linewidth(0.7));
draw( (32.50000014707838,0.997291988376602) -- (35.00000014652595,0.9969281925447768),ccqqqq+linewidth(0.7));
draw( (35.00000014652595,0.9969281925447768) -- (37.500000145973516,0.9965390773976738),ccqqqq+linewidth(0.7));
draw( (37.500000145973516,0.9965390773976738) -- (40.00000014542108,0.9961248285088954),ccqqqq+linewidth(0.7));
draw( (40.00000014542108,0.9961248285088954) -- (42.50000014486865,0.9956856260078994),ccqqqq+linewidth(0.7));
draw( (42.50000014486865,0.9956856260078994) -- (45.00000014431622,0.9952216447245802),ccqqqq+linewidth(0.7));
draw( (45.00000014431622,0.9952216447245802) -- (47.500000143763785,0.9947330543318623),ccqqqq+linewidth(0.7));
draw( (47.500000143763785,0.9947330543318623) -- (50.00000014321135,0.9942200194863203),ccqqqq+linewidth(0.7));
draw( (50.00000014321135,0.9942200194863203) -- (52.50000014265892,0.9936826999668348),ccqqqq+linewidth(0.7));
draw( (52.50000014265892,0.9936826999668348) -- (55.00000014210649,0.9931212508113005),ccqqqq+linewidth(0.7));
draw( (55.00000014210649,0.9931212508113005) -- (57.500000141554054,0.9925358224513924),ccqqqq+linewidth(0.7));
draw( (57.500000141554054,0.9925358224513924) -- (60.00000014100162,0.991926560845412),ccqqqq+linewidth(0.7));
draw( (60.00000014100162,0.991926560845412) -- (62.50000014044919,0.9912936076092146),ccqqqq+linewidth(0.7));
draw( (62.50000014044919,0.9912936076092146) -- (65.00000013989676,0.9906371001452384),ccqqqq+linewidth(0.7));
draw( (65.00000013989676,0.9906371001452384) -- (67.50000013934432,0.9899571717696448),ccqqqq+linewidth(0.7));
draw( (67.50000013934432,0.9899571717696448) -- (70.00000013879189,0.9892539518375801),ccqqqq+linewidth(0.7));
draw( (70.00000013879189,0.9892539518375801) -- (72.50000013823946,0.9885275658665741),ccqqqq+linewidth(0.7));
draw( (72.50000013823946,0.9885275658665741) -- (75.00000013768702,0.9877781356580868),ccqqqq+linewidth(0.7));
draw( (75.00000013768702,0.9877781356580868) -- (77.50000013713459,0.9870057794172139),ccqqqq+linewidth(0.7));
draw( (77.50000013713459,0.9870057794172139) -- (80.00000013658216,0.9862106118705657),ccqqqq+linewidth(0.7));
draw( (80.00000013658216,0.9862106118705657) -- (82.50000013602973,0.9853927443823304),ccqqqq+linewidth(0.7));
draw( (82.50000013602973,0.9853927443823304) -- (85.0000001354773,0.9845522850685329),ccqqqq+linewidth(0.7));
draw( (85.0000001354773,0.9845522850685329) -- (87.50000013492486,0.9836893389095042),ccqqqq+linewidth(0.7));
draw( (87.50000013492486,0.9836893389095042) -- (90.00000013437243,0.9828040078605726),ccqqqq+linewidth(0.7));
draw( (90.00000013437243,0.9828040078605726) -- (92.50000013382,0.9818963909609846),ccqqqq+linewidth(0.7));
draw( (92.50000013382,0.9818963909609846) -- (95.00000013326756,0.980966584441078),ccqqqq+linewidth(0.7));
draw( (95.00000013326756,0.980966584441078) -- (97.50000013271513,0.9800146818277052),ccqqqq+linewidth(0.7));
draw( (97.50000013271513,0.9800146818277052) -- (100.0000001321627,0.9790407740479342),ccqqqq+linewidth(0.7));
draw( (100.0000001321627,0.9790407740479342) -- (102.50000013161026,0.978044949531027),ccqqqq+linewidth(0.7));
draw( (102.50000013161026,0.978044949531027) -- (105.00000013105783,0.9770272943087154),ccqqqq+linewidth(0.7));
draw( (105.00000013105783,0.9770272943087154) -- (107.5000001305054,0.975987892113782),ccqqqq+linewidth(0.7));
draw( (107.5000001305054,0.975987892113782) -- (110.00000012995297,0.9749268244769622),ccqqqq+linewidth(0.7));
draw( (110.00000012995297,0.9749268244769622) -- (112.50000012940053,0.9738441708221767),ccqqqq+linewidth(0.7));
draw( (112.50000012940053,0.9738441708221767) -- (115.0000001288481,0.9727400085601069),ccqqqq+linewidth(0.7));
draw( (115.0000001288481,0.9727400085601069) -- (117.50000012829567,0.9716144131801291),ccqqqq+linewidth(0.7));
draw( (117.50000012829567,0.9716144131801291) -- (120.00000012774323,0.9704674583406138),ccqqqq+linewidth(0.7));
draw( (120.00000012774323,0.9704674583406138) -- (122.5000001271908,0.9692992159576067),ccqqqq+linewidth(0.7));
draw( (122.5000001271908,0.9692992159576067) -- (125.00000012663837,0.9681097562919031),ccqqqq+linewidth(0.7));
draw( (125.00000012663837,0.9681097562919031) -- (127.50000012608594,0.9668991480345264),ccqqqq+linewidth(0.7));
draw( (127.50000012608594,0.9668991480345264) -- (130.00000012553352,0.9656674583906242),ccqqqq+linewidth(0.7));
draw( (130.00000012553352,0.9656674583906242) -- (132.50000012498109,0.9644147531617938),ccqqqq+linewidth(0.7));
draw( (132.50000012498109,0.9644147531617938) -- (135.00000012442865,0.9631410968268489),ccqqqq+linewidth(0.7));
draw( (135.00000012442865,0.9631410968268489) -- (137.50000012387622,0.9618465526210411),ccqqqq+linewidth(0.7));
draw( (137.50000012387622,0.9618465526210411) -- (140.0000001233238,0.960531182613747),ccqqqq+linewidth(0.7));
draw( (140.0000001233238,0.960531182613747) -- (142.50000012277135,0.9591950477846342),ccqqqq+linewidth(0.7));
draw( (142.50000012277135,0.9591950477846342) -- (145.00000012221892,0.9578382080983178),ccqqqq+linewidth(0.7));
draw( (145.00000012221892,0.9578382080983178) -- (147.5000001216665,0.9564607225775189),ccqqqq+linewidth(0.7));
draw( (147.5000001216665,0.9564607225775189) -- (150.00000012111406,0.9550626493747405),ccqqqq+linewidth(0.7));
draw( (150.00000012111406,0.9550626493747405) -- (152.50000012056162,0.9536440458424682),ccqqqq+linewidth(0.7));
draw( (152.50000012056162,0.9536440458424682) -- (155.0000001200092,0.9522049686019131),ccqqqq+linewidth(0.7));
draw( (155.0000001200092,0.9522049686019131) -- (157.50000011945676,0.9507454736103045),ccqqqq+linewidth(0.7));
draw( (157.50000011945676,0.9507454736103045) -- (160.00000011890432,0.9492656162267474),ccqqqq+linewidth(0.7));
draw( (160.00000011890432,0.9492656162267474) -- (162.5000001183519,0.9477654512766579),ccqqqq+linewidth(0.7));
draw( (162.5000001183519,0.9477654512766579) -- (165.00000011779946,0.9462450331147854),ccqqqq+linewidth(0.7));
draw( (165.00000011779946,0.9462450331147854) -- (167.50000011724703,0.9447044156868352),ccqqqq+linewidth(0.7));
draw( (167.50000011724703,0.9447044156868352) -- (170.0000001166946,0.9431436525897077),ccqqqq+linewidth(0.7));
draw( (170.0000001166946,0.9431436525897077) -- (172.50000011614216,0.941562797130359),ccqqqq+linewidth(0.7));
draw( (172.50000011614216,0.941562797130359) -- (175.00000011558973,0.9399619023833007),ccqqqq+linewidth(0.7));
draw( (175.00000011558973,0.9399619023833007) -- (177.5000001150373,0.9383410212467514),ccqqqq+linewidth(0.7));
draw( (177.5000001150373,0.9383410212467514) -- (180.00000011448486,0.936700206497447),ccqqqq+linewidth(0.7));
draw( (180.00000011448486,0.936700206497447) -- (182.50000011393243,0.9350395108441276),ccqqqq+linewidth(0.7));
draw( (182.50000011393243,0.9350395108441276) -- (185.00000011338,0.9333589869797106),ccqqqq+linewidth(0.7));
draw( (185.00000011338,0.9333589869797106) -- (187.50000011282756,0.931658687632163),ccqqqq+linewidth(0.7));
draw( (187.50000011282756,0.931658687632163) -- (190.00000011227513,0.9299386656140832),ccqqqq+linewidth(0.7));
draw( (190.00000011227513,0.9299386656140832) -- (192.5000001117227,0.9281989738710088),ccqqqq+linewidth(0.7));
draw( (192.5000001117227,0.9281989738710088) -- (195.00000011117027,0.9264396655284574),ccqqqq+linewidth(0.7));
draw( (195.00000011117027,0.9264396655284574) -- (197.50000011061783,0.9246607939377169),ccqqqq+linewidth(0.7));
draw( (197.50000011061783,0.9246607939377169) -- (200.0000001100654,0.9228624127203955),ccqqqq+linewidth(0.7));
draw( (200.0000001100654,0.9228624127203955) -- (202.50000010951297,0.9210445758117445),ccqqqq+linewidth(0.7));
draw( (202.50000010951297,0.9210445758117445) -- (205.00000010896053,0.9192073375027637),ccqqqq+linewidth(0.7));
draw( (205.00000010896053,0.9192073375027637) -- (207.5000001084081,0.9173507524811072),ccqqqq+linewidth(0.7));
draw( (207.5000001084081,0.9173507524811072) -- (210.00000010785567,0.9154748758707942),ccqqqq+linewidth(0.7));
draw( (210.00000010785567,0.9154748758707942) -- (212.50000010730324,0.9135797632707451),ccqqqq+linewidth(0.7));
draw( (212.50000010730324,0.9135797632707451) -- (215.0000001067508,0.9116654707921473),ccqqqq+linewidth(0.7));
draw( (215.0000001067508,0.9116654707921473) -- (217.50000010619837,0.9097320550946708),ccqqqq+linewidth(0.7));
draw( (217.50000010619837,0.9097320550946708) -- (220.00000010564594,0.9077795734215386),ccqqqq+linewidth(0.7));
draw( (220.00000010564594,0.9077795734215386) -- (222.5000001050935,0.9058080836334688),ccqqqq+linewidth(0.7));
draw( (222.5000001050935,0.9058080836334688) -- (225.00000010454107,0.9038176442414998),ccqqqq+linewidth(0.7));
draw( (225.00000010454107,0.9038176442414998) -- (227.50000010398864,0.901808314438709),ccqqqq+linewidth(0.7));
draw( (227.50000010398864,0.901808314438709) -- (230.0000001034362,0.8997801541308404),ccqqqq+linewidth(0.7));
draw( (230.0000001034362,0.8997801541308404) -- (232.50000010288377,0.8977332239658483),ccqqqq+linewidth(0.7));
draw( (232.50000010288377,0.8977332239658483) -- (235.00000010233134,0.8956675853623769),ccqqqq+linewidth(0.7));
draw( (235.00000010233134,0.8956675853623769) -- (237.5000001017789,0.8935833005371805),ccqqqq+linewidth(0.7));
draw( (237.5000001017789,0.8935833005371805) -- (240.00000010122648,0.8914804325314996),ccqqqq+linewidth(0.7));
draw( (240.00000010122648,0.8914804325314996) -- (242.50000010067404,0.8893590452364087),ccqqqq+linewidth(0.7));
draw( (242.50000010067404,0.8893590452364087) -- (245.0000001001216,0.8872192034171406),ccqqqq+linewidth(0.7));
draw( (245.0000001001216,0.8872192034171406) -- (247.50000009956918,0.8850609727364047),ccqqqq+linewidth(0.7));
draw( (247.50000009956918,0.8850609727364047) -- (250.00000009901675,0.8828844197767133),ccqqqq+linewidth(0.7));
draw( (250.00000009901675,0.8828844197767133) -- (252.5000000984643,0.88068961206172),ccqqqq+linewidth(0.7));
draw( (252.5000000984643,0.88068961206172) -- (255.00000009791188,0.8784766180765924),ccqqqq+linewidth(0.7));
draw( (255.00000009791188,0.8784766180765924) -- (257.5000000973595,0.8762455072874227),ccqqqq+linewidth(0.7));
draw( (257.5000000973595,0.8762455072874227) -- (260.00000009680707,0.8739963501596963),ccqqqq+linewidth(0.7));
draw( (260.00000009680707,0.8739963501596963) -- (262.50000009625467,0.8717292181758225),ccqqqq+linewidth(0.7));
draw( (262.50000009625467,0.8717292181758225) -- (265.00000009570226,0.8694441838517465),ccqqqq+linewidth(0.7));
draw( (265.00000009570226,0.8694441838517465) -- (267.50000009514986,0.8671413207526518),ccqqqq+linewidth(0.7));
draw( (267.50000009514986,0.8671413207526518) -- (270.00000009459745,0.8648207035077645),ccqqqq+linewidth(0.7));
draw( (270.00000009459745,0.8648207035077645) -- (272.50000009404505,0.8624824078242742),ccqqqq+linewidth(0.7));
draw( (272.50000009404505,0.8624824078242742) -- (275.00000009349264,0.8601265105003806),ccqqqq+linewidth(0.7));
draw( (275.00000009349264,0.8601265105003806) -- (277.50000009294024,0.8577530894374819),ccqqqq+linewidth(0.7));
draw( (277.50000009294024,0.8577530894374819) -- (280.00000009238784,0.8553622236515145),ccqqqq+linewidth(0.7));
draw( (280.00000009238784,0.8553622236515145) -- (282.50000009183543,0.8529539932834561),ccqqqq+linewidth(0.7));
draw( (282.50000009183543,0.8529539932834561) -- (285.000000091283,0.8505284796090054),ccqqqq+linewidth(0.7));
draw( (285.000000091283,0.8505284796090054) -- (287.5000000907306,0.8480857650474521),ccqqqq+linewidth(0.7));
draw( (287.5000000907306,0.8480857650474521) -- (290.0000000901782,0.8456259331697461),ccqqqq+linewidth(0.7));
draw( (290.0000000901782,0.8456259331697461) -- (292.5000000896258,0.8431490687057811),ccqqqq+linewidth(0.7));
draw( (292.5000000896258,0.8431490687057811) -- (295.0000000890734,0.8406552575509022),ccqqqq+linewidth(0.7));
draw( (295.0000000890734,0.8406552575509022) -- (297.500000088521,0.8381445867716532),ccqqqq+linewidth(0.7));
draw( (297.500000088521,0.8381445867716532) -- (300.0000000879686,0.8356171446107717),ccqqqq+linewidth(0.7));
draw( (300.0000000879686,0.8356171446107717) -- (302.5000000874162,0.8330730204914495),ccqqqq+linewidth(0.7));
draw( (302.5000000874162,0.8330730204914495) -- (305.0000000868638,0.8305123050208629),ccqqqq+linewidth(0.7));
draw( (305.0000000868638,0.8305123050208629) -- (307.5000000863114,0.8279350899929945),ccqqqq+linewidth(0.7));
draw( (307.5000000863114,0.8279350899929945) -- (310.000000085759,0.8253414683907502),ccqqqq+linewidth(0.7));
draw( (310.000000085759,0.8253414683907502) -- (312.5000000852066,0.8227315343873897),ccqqqq+linewidth(0.7));
draw( (312.5000000852066,0.8227315343873897) -- (315.0000000846542,0.8201053833472798),ccqqqq+linewidth(0.7));
draw( (315.0000000846542,0.8201053833472798) -- (317.50000008410177,0.8174631118259822),ccqqqq+linewidth(0.7));
draw( (317.50000008410177,0.8174631118259822) -- (320.00000008354937,0.8148048175696917),ccqqqq+linewidth(0.7));
draw( (320.00000008354937,0.8148048175696917) -- (322.50000008299696,0.8121305995140324),ccqqqq+linewidth(0.7));
draw( (322.50000008299696,0.8121305995140324) -- (325.00000008244456,0.8094405577822261),ccqqqq+linewidth(0.7));
draw( (325.00000008244456,0.8094405577822261) -- (327.50000008189215,0.8067347936826479),ccqqqq+linewidth(0.7));
draw( (327.50000008189215,0.8067347936826479) -- (330.00000008133975,0.8040134097057767),ccqqqq+linewidth(0.7));
draw( (330.00000008133975,0.8040134097057767) -- (332.50000008078734,0.8012765095205551),ccqqqq+linewidth(0.7));
draw( (332.50000008078734,0.8012765095205551) -- (335.00000008023494,0.7985241979701699),ccqqqq+linewidth(0.7));
draw( (335.00000008023494,0.7985241979701699) -- (337.50000007968254,0.7957565810672695),ccqqqq+linewidth(0.7));
draw( (337.50000007968254,0.7957565810672695) -- (340.00000007913013,0.7929737659886235),ccqqqq+linewidth(0.7));
draw( (340.00000007913013,0.7929737659886235) -- (342.5000000785777,0.7901758610692413),ccqqqq+linewidth(0.7));
draw( (342.5000000785777,0.7901758610692413) -- (345.0000000780253,0.7873629757959631),ccqqqq+linewidth(0.7));
draw( (345.0000000780253,0.7873629757959631) -- (347.5000000774729,0.7845352208005317),ccqqqq+linewidth(0.7));
draw( (347.5000000774729,0.7845352208005317) -- (350.0000000769205,0.7816927078521588),ccqqqq+linewidth(0.7));
draw( (350.0000000769205,0.7816927078521588) -- (352.5000000763681,0.7788355498496004),ccqqqq+linewidth(0.7));
draw( (352.5000000763681,0.7788355498496004) -- (355.0000000758157,0.7759638608127494),ccqqqq+linewidth(0.7));
draw( (355.0000000758157,0.7759638608127494) -- (357.5000000752633,0.7730777558737616),ccqqqq+linewidth(0.7));
draw( (357.5000000752633,0.7730777558737616) -- (360.0000000747109,0.7701773512677245),ccqqqq+linewidth(0.7));
draw( (360.0000000747109,0.7701773512677245) -- (362.5000000741585,0.7672627643228827),ccqqqq+linewidth(0.7));
draw( (362.5000000741585,0.7672627643228827) -- (365.0000000736061,0.7643341134504305),ccqqqq+linewidth(0.7));
draw( (365.0000000736061,0.7643341134504305) -- (367.5000000730537,0.7613915181338876),ccqqqq+linewidth(0.7));
draw( (367.5000000730537,0.7613915181338876) -- (370.0000000725013,0.7584350989180658),ccqqqq+linewidth(0.7));
draw( (370.0000000725013,0.7584350989180658) -- (372.5000000719489,0.7554649773976396),ccqqqq+linewidth(0.7));
draw( (372.5000000719489,0.7554649773976396) -- (375.00000007139647,0.752481276205338),ccqqqq+linewidth(0.7));
draw( (375.00000007139647,0.752481276205338) -- (377.50000007084407,0.7494841189997614),ccqqqq+linewidth(0.7));
draw( (377.50000007084407,0.7494841189997614) -- (380.00000007029166,0.7464736304528445),ccqqqq+linewidth(0.7));
draw( (380.00000007029166,0.7464736304528445) -- (382.50000006973926,0.7434499362369695),ccqqqq+linewidth(0.7));
draw( (382.50000006973926,0.7434499362369695) -- (385.00000006918685,0.7404131630117488),ccqqqq+linewidth(0.7));
draw( (385.00000006918685,0.7404131630117488) -- (387.50000006863445,0.7373634384104841),ccqqqq+linewidth(0.7));
draw( (387.50000006863445,0.7373634384104841) -- (390.00000006808204,0.7343008910263179),ccqqqq+linewidth(0.7));
draw( (390.00000006808204,0.7343008910263179) -- (392.50000006752964,0.731225650398086),ccqqqq+linewidth(0.7));
draw( (392.50000006752964,0.731225650398086) -- (395.00000006697724,0.7281378469958879),ccqqqq+linewidth(0.7));
draw( (395.00000006697724,0.7281378469958879) -- (397.50000006642483,0.725037612206383),ccqqqq+linewidth(0.7));
draw( (397.50000006642483,0.725037612206383) -- (400.0000000658724,0.7219250783178269),ccqqqq+linewidth(0.7));
draw( (400.0000000658724,0.7219250783178269) -- (402.50000006532,0.7188003785048608),ccqqqq+linewidth(0.7));
draw( (402.50000006532,0.7188003785048608) -- (405.0000000647676,0.7156636468130623),ccqqqq+linewidth(0.7));
draw( (405.0000000647676,0.7156636468130623) -- (407.5000000642152,0.712515018143277),ccqqqq+linewidth(0.7));
draw( (407.5000000642152,0.712515018143277) -- (410.0000000636628,0.7093546282357363),ccqqqq+linewidth(0.7));
draw( (410.0000000636628,0.7093546282357363) -- (412.5000000631104,0.7061826136539753),ccqqqq+linewidth(0.7));
draw( (412.5000000631104,0.7061826136539753) -- (415.000000062558,0.702999111768566),ccqqqq+linewidth(0.7));
draw( (415.000000062558,0.702999111768566) -- (417.5000000620056,0.6998042607406767),ccqqqq+linewidth(0.7));
draw( (417.5000000620056,0.6998042607406767) -- (420.0000000614532,0.6965981995054652),ccqqqq+linewidth(0.7));
draw( (420.0000000614532,0.6965981995054652) -- (422.5000000609008,0.6933810677553269),ccqqqq+linewidth(0.7));
draw( (422.5000000609008,0.6933810677553269) -- (425.0000000603484,0.6901530059230019),ccqqqq+linewidth(0.7));
draw( (425.0000000603484,0.6901530059230019) -- (427.500000059796,0.6869141551645573),ccqqqq+linewidth(0.7));
draw( (427.500000059796,0.6869141551645573) -- (430.0000000592436,0.6836646573422579),ccqqqq+linewidth(0.7));
draw( (430.0000000592436,0.6836646573422579) -- (432.50000005869117,0.6804046550073329),ccqqqq+linewidth(0.7));
draw( (432.50000005869117,0.6804046550073329) -- (435.00000005813877,0.6771342913826573),ccqqqq+linewidth(0.7));
draw( (435.00000005813877,0.6771342913826573) -- (437.50000005758636,0.6738537103453531),ccqqqq+linewidth(0.7));
draw( (437.50000005758636,0.6738537103453531) -- (440.00000005703396,0.6705630564093306),ccqqqq+linewidth(0.7));
draw( (440.00000005703396,0.6705630564093306) -- (442.50000005648155,0.6672624747077724),ccqqqq+linewidth(0.7));
draw( (442.50000005648155,0.6672624747077724) -- (445.00000005592915,0.6639521109755833),ccqqqq+linewidth(0.7));
draw( (445.00000005592915,0.6639521109755833) -- (447.50000005537674,0.6606321115318088),ccqqqq+linewidth(0.7));
draw( (447.50000005537674,0.6606321115318088) -- (450.00000005482434,0.6573026232620387),ccqqqq+linewidth(0.7));
draw( (450.00000005482434,0.6573026232620387) -- (452.50000005427194,0.6539637936008091),ccqqqq+linewidth(0.7));
draw( (452.50000005427194,0.6539637936008091) -- (455.00000005371953,0.6506157705140135),ccqqqq+linewidth(0.7));
draw( (455.00000005371953,0.6506157705140135) -- (457.5000000531671,0.6472587024813345),ccqqqq+linewidth(0.7));
draw( (457.5000000531671,0.6472587024813345) -- (460.0000000526147,0.6438927384787085),ccqqqq+linewidth(0.7));
draw( (460.0000000526147,0.6438927384787085) -- (462.5000000520623,0.6405180279608397),ccqqqq+linewidth(0.7));
draw( (462.5000000520623,0.6405180279608397) -- (465.0000000515099,0.6371347208437669),ccqqqq+linewidth(0.7));
draw( (465.0000000515099,0.6371347208437669) -- (467.5000000509575,0.6337429674875067),ccqqqq+linewidth(0.7));
draw( (467.5000000509575,0.6337429674875067) -- (470.0000000504051,0.6303429186787755),ccqqqq+linewidth(0.7));
draw( (470.0000000504051,0.6303429186787755) -- (472.5000000498527,0.6269347256138071),ccqqqq+linewidth(0.7));
draw( (472.5000000498527,0.6269347256138071) -- (475.0000000493003,0.623518539881279),ccqqqq+linewidth(0.7));
draw( (475.0000000493003,0.623518539881279) -- (477.5000000487479,0.6200945134453558),ccqqqq+linewidth(0.7));
draw( (477.5000000487479,0.6200945134453558) -- (480.0000000481955,0.616662798628866),ccqqqq+linewidth(0.7));
draw( (480.0000000481955,0.616662798628866) -- (482.5000000476431,0.6132235480966235),ccqqqq+linewidth(0.7));
draw( (482.5000000476431,0.6132235480966235) -- (485.0000000470907,0.6097769148389022),ccqqqq+linewidth(0.7));
draw( (485.0000000470907,0.6097769148389022) -- (487.5000000465383,0.6063230521550824),ccqqqq+linewidth(0.7));
draw( (487.5000000465383,0.6063230521550824) -- (490.00000004598587,0.6028621136374749),ccqqqq+linewidth(0.7));
draw( (490.00000004598587,0.6028621136374749) -- (492.50000004543347,0.5993942531553411),ccqqqq+linewidth(0.7));
draw( (492.50000004543347,0.5993942531553411) -- (495.00000004488106,0.5959196248391159),ccqqqq+linewidth(0.7));
draw( (495.00000004488106,0.5959196248391159) -- (497.50000004432866,0.5924383830648484),ccqqqq+linewidth(0.7));
draw( (497.50000004432866,0.5924383830648484) -- (500.00000004377625,0.5889506824388729),ccqqqq+linewidth(0.7));
draw( (500.00000004377625,0.5889506824388729) -- (502.50000004322385,0.5854566777827223),ccqqqq+linewidth(0.7));
draw( (502.50000004322385,0.5854566777827223) -- (505.00000004267145,0.5819565241182932),ccqqqq+linewidth(0.7));
draw( (505.00000004267145,0.5819565241182932) -- (507.50000004211904,0.5784503766532808),ccqqqq+linewidth(0.7));
draw( (507.50000004211904,0.5784503766532808) -- (510.00000004156664,0.5749383907668901),ccqqqq+linewidth(0.7));
draw( (510.00000004156664,0.5749383907668901) -- (512.5000000410142,0.5714207219958384),ccqqqq+linewidth(0.7));
draw( (512.5000000410142,0.5714207219958384) -- (515.0000000404618,0.5678975260206603),ccqqqq+linewidth(0.7));
draw( (515.0000000404618,0.5678975260206603) -- (517.5000000399094,0.564368958652332),ccqqqq+linewidth(0.7));
draw( (517.5000000399094,0.564368958652332) -- (520.000000039357,0.5608351758192159),ccqqqq+linewidth(0.7));
draw( (520.000000039357,0.5608351758192159) -- (522.5000000388046,0.5572963335543526),ccqqqq+linewidth(0.7));
draw( (522.5000000388046,0.5572963335543526) -- (525.0000000382522,0.5537525879831006),ccqqqq+linewidth(0.7));
draw( (525.0000000382522,0.5537525879831006) -- (527.5000000376998,0.5502040953111411),ccqqqq+linewidth(0.7));
draw( (527.5000000376998,0.5502040953111411) -- (530.0000000371474,0.5466510118128606),ccqqqq+linewidth(0.7));
draw( (530.0000000371474,0.5466510118128606) -- (532.500000036595,0.5430934938201211),ccqqqq+linewidth(0.7));
draw( (532.500000036595,0.5430934938201211) -- (535.0000000360426,0.5395316977114324),ccqqqq+linewidth(0.7));
draw( (535.0000000360426,0.5395316977114324) -- (537.5000000354902,0.5359657799015364),ccqqqq+linewidth(0.7));
draw( (537.5000000354902,0.5359657799015364) -- (540.0000000349378,0.5323958968314187),ccqqqq+linewidth(0.7));
draw( (540.0000000349378,0.5323958968314187) -- (542.5000000343854,0.5288222049587562),ccqqqq+linewidth(0.7));
draw( (542.5000000343854,0.5288222049587562) -- (545.000000033833,0.5252448607488176),ccqqqq+linewidth(0.7));
draw( (545.000000033833,0.5252448607488176) -- (547.5000000332806,0.521664020665822),ccqqqq+linewidth(0.7));
draw( (547.5000000332806,0.521664020665822) -- (550.0000000327282,0.5180798411647762),ccqqqq+linewidth(0.7));
draw( (550.0000000327282,0.5180798411647762) -- (552.5000000321758,0.5144924786837968),ccqqqq+linewidth(0.7));
draw( (552.5000000321758,0.5144924786837968) -- (555.0000000316234,0.5109020896369308),ccqqqq+linewidth(0.7));
draw( (555.0000000316234,0.5109020896369308) -- (557.500000031071,0.5073088304074904),ccqqqq+linewidth(0.7));
draw( (557.500000031071,0.5073088304074904) -- (560.0000000305185,0.503712857341907),ccqqqq+linewidth(0.7));
draw( (560.0000000305185,0.503712857341907) -- (562.5000000299661,0.5001143267441246),ccqqqq+linewidth(0.7));
draw( (562.5000000299661,0.5001143267441246) -- (565.0000000294137,0.4965133948705405),ccqqqq+linewidth(0.7));
draw( (565.0000000294137,0.4965133948705405) -- (567.5000000288613,0.49291021792550516),ccqqqq+linewidth(0.7));
draw( (567.5000000288613,0.49291021792550516) -- (570.0000000283089,0.48930495205739577),ccqqqq+linewidth(0.7));
draw( (570.0000000283089,0.48930495205739577) -- (572.5000000277565,0.4856977533552731),ccqqqq+linewidth(0.7));
draw( (572.5000000277565,0.4856977533552731) -- (575.0000000272041,0.48208877784613796),ccqqqq+linewidth(0.7));
draw( (575.0000000272041,0.48208877784613796) -- (577.5000000266517,0.4784781814927952),ccqqqq+linewidth(0.7));
draw( (577.5000000266517,0.4784781814927952) -- (580.0000000260993,0.47486612019233726),ccqqqq+linewidth(0.7));
draw( (580.0000000260993,0.47486612019233726) -- (582.5000000255469,0.47125274977526754),ccqqqq+linewidth(0.7));
draw( (582.5000000255469,0.47125274977526754) -- (585.0000000249945,0.467638226005263),ccqqqq+linewidth(0.7));
draw( (585.0000000249945,0.467638226005263) -- (587.5000000244421,0.4640227045795997),ccqqqq+linewidth(0.7));
draw( (587.5000000244421,0.4640227045795997) -- (590.0000000238897,0.46040634113024537),ccqqqq+linewidth(0.7));
draw( (590.0000000238897,0.46040634113024537) -- (592.5000000233373,0.45678929122563927),ccqqqq+linewidth(0.7));
draw( (592.5000000233373,0.45678929122563927) -- (595.0000000227849,0.45317171037316184),ccqqqq+linewidth(0.7));
draw( (595.0000000227849,0.45317171037316184) -- (597.5000000222325,0.4495537540223149),ccqqqq+linewidth(0.7));
draw( (597.5000000222325,0.4495537540223149) -- (600.0000000216801,0.4459355775686226),ccqqqq+linewidth(0.7));
draw( (600.0000000216801,0.4459355775686226) -- (602.5000000211277,0.44231733635825865),ccqqqq+linewidth(0.7));
draw( (602.5000000211277,0.44231733635825865) -- (605.0000000205753,0.4386991856934238),ccqqqq+linewidth(0.7));
draw( (605.0000000205753,0.4386991856934238) -- (607.5000000200229,0.43508128083847514),ccqqqq+linewidth(0.7));
draw( (607.5000000200229,0.43508128083847514) -- (610.0000000194705,0.43146377702682637),ccqqqq+linewidth(0.7));
draw( (610.0000000194705,0.43146377702682637) -- (612.5000000189181,0.42784682946862806),ccqqqq+linewidth(0.7));
draw( (612.5000000189181,0.42784682946862806) -- (615.0000000183657,0.4242305933592413),ccqqqq+linewidth(0.7));
draw( (615.0000000183657,0.4242305933592413) -- (617.5000000178132,0.4206152238885137),ccqqqq+linewidth(0.7));
draw( (617.5000000178132,0.4206152238885137) -- (620.0000000172608,0.41700087625087784),ccqqqq+linewidth(0.7));
draw( (620.0000000172608,0.41700087625087784) -- (622.5000000167084,0.41338770565627614),ccqqqq+linewidth(0.7));
draw( (622.5000000167084,0.41338770565627614) -- (625.000000016156,0.4097758673419286),ccqqqq+linewidth(0.7));
draw( (625.000000016156,0.4097758673419286) -- (627.5000000156036,0.4061655165849533),ccqqqq+linewidth(0.7));
draw( (627.5000000156036,0.4061655165849533) -- (630.0000000150512,0.40255680871585564),ccqqqq+linewidth(0.7));
draw( (630.0000000150512,0.40255680871585564) -- (632.5000000144988,0.39894989913289625),ccqqqq+linewidth(0.7));
draw( (632.5000000144988,0.39894989913289625) -- (635.0000000139464,0.3953449433173477),ccqqqq+linewidth(0.7));
draw( (635.0000000139464,0.3953449433173477) -- (637.500000013394,0.391742096849655),ccqqqq+linewidth(0.7));
draw( (637.500000013394,0.391742096849655) -- (640.0000000128416,0.3881415154265135),ccqqqq+linewidth(0.7));
draw( (640.0000000128416,0.3881415154265135) -- (642.5000000122892,0.38454335487887314),ccqqqq+linewidth(0.7));
draw( (642.5000000122892,0.38454335487887314) -- (645.0000000117368,0.38094777119088175),ccqqqq+linewidth(0.7));
draw( (645.0000000117368,0.38094777119088175) -- (647.5000000111844,0.37735492051978153),ccqqqq+linewidth(0.7));
draw( (647.5000000111844,0.37735492051978153) -- (650.000000010632,0.3737649592167702),ccqqqq+linewidth(0.7));
draw( (650.000000010632,0.3737649592167702) -- (652.5000000100796,0.37017804384883957),ccqqqq+linewidth(0.7));
draw( (652.5000000100796,0.37017804384883957) -- (655.0000000095272,0.3665943312215993),ccqqqq+linewidth(0.7));
draw( (655.0000000095272,0.3665943312215993) -- (657.5000000089748,0.3630139784031066),ccqqqq+linewidth(0.7));
draw( (657.5000000089748,0.3630139784031066) -- (660.0000000084224,0.3594371427487073),ccqqqq+linewidth(0.7));
draw( (660.0000000084224,0.3594371427487073) -- (662.50000000787,0.3558639819269016),ccqqqq+linewidth(0.7));
draw( (662.50000000787,0.3558639819269016) -- (665.0000000073176,0.3522946539462477),ccqqqq+linewidth(0.7));
draw( (665.0000000073176,0.3522946539462477) -- (667.5000000067652,0.3487293171833196),ccqqqq+linewidth(0.7));
draw( (667.5000000067652,0.3487293171833196) -- (670.0000000062128,0.3451681304117221),ccqqqq+linewidth(0.7));
draw( (670.0000000062128,0.3451681304117221) -- (672.5000000056604,0.3416112528321815),ccqqqq+linewidth(0.7));
draw( (672.5000000056604,0.3416112528321815) -- (675.000000005108,0.33805884410372694),ccqqqq+linewidth(0.7));
draw( (675.000000005108,0.33805884410372694) -- (677.5000000045555,0.33451106437596667),ccqqqq+linewidth(0.7));
draw( (677.5000000045555,0.33451106437596667) -- (680.0000000040031,0.33096807432247666),ccqqqq+linewidth(0.7));
draw( (680.0000000040031,0.33096807432247666) -- (682.5000000034507,0.3274300351753165),ccqqqq+linewidth(0.7));
draw( (682.5000000034507,0.3274300351753165) -- (685.0000000028983,0.3238971087606733),ccqqqq+linewidth(0.7));
draw( (685.0000000028983,0.3238971087606733) -- (687.5000000023459,0.3203694575356632),ccqqqq+linewidth(0.7));
draw( (687.5000000023459,0.3203694575356632) -- (690.0000000017935,0.3168472446262853),ccqqqq+linewidth(0.7));
draw( (690.0000000017935,0.3168472446262853) -- (692.5000000012411,0.31333063386655374),ccqqqq+linewidth(0.7));
draw( (692.5000000012411,0.31333063386655374) -- (695.0000000006887,0.3098197898388102),ccqqqq+linewidth(0.7));
draw( (695.0000000006887,0.3098197898388102) -- (697.5000000001363,0.3063148779152385),ccqqqq+linewidth(0.7));
draw( (697.5000000001363,0.3063148779152385) -- (699.9999999995839,0.30281606430058167),ccqqqq+linewidth(0.7));
draw( (699.9999999995839,0.30281606430058167) -- (702.4999999990315,0.29932351607609115),ccqqqq+linewidth(0.7));
draw( (702.4999999990315,0.29932351607609115) -- (704.9999999984791,0.29583740124469554),ccqqqq+linewidth(0.7));
draw( (704.9999999984791,0.29583740124469554) -- (707.4999999979267,0.2923578887774296),ccqqqq+linewidth(0.7));
draw( (707.4999999979267,0.2923578887774296) -- (709.9999999973743,0.28888514866111187),ccqqqq+linewidth(0.7));
draw( (709.9999999973743,0.28888514866111187) -- (712.4999999968219,0.2854193519472981),ccqqqq+linewidth(0.7));
draw( (712.4999999968219,0.2854193519472981) -- (714.9999999962695,0.2819606708025149),ccqqqq+linewidth(0.7));
draw( (714.9999999962695,0.2819606708025149) -- (717.4999999957171,0.2785092785597904),ccqqqq+linewidth(0.7));
draw( (717.4999999957171,0.2785092785597904) -- (719.9999999951647,0.2750653497714875),ccqqqq+linewidth(0.7));
draw( (719.9999999951647,0.2750653497714875) -- (722.4999999946123,0.27162906026346506),ccqqqq+linewidth(0.7));
draw( (722.4999999946123,0.27162906026346506) -- (724.9999999940599,0.2682005871905652),ccqqqq+linewidth(0.7));
draw( (724.9999999940599,0.2682005871905652) -- (727.4999999935075,0.2647801090934414),ccqqqq+linewidth(0.7));
draw( (727.4999999935075,0.2647801090934414) -- (729.999999992955,0.261367805956754),ccqqqq+linewidth(0.7));
draw( (729.999999992955,0.261367805956754) -- (732.4999999924026,0.25796385926872084),ccqqqq+linewidth(0.7));
draw( (732.4999999924026,0.25796385926872084) -- (734.9999999918502,0.25456845208205936),ccqqqq+linewidth(0.7));
draw( (734.9999999918502,0.25456845208205936) -- (737.4999999912978,0.2511817690763161),ccqqqq+linewidth(0.7));
draw( (737.4999999912978,0.2511817690763161) -- (739.9999999907454,0.24780399662160485),ccqqqq+linewidth(0.7));
draw( (739.9999999907454,0.24780399662160485) -- (742.499999990193,0.24443532284375813),ccqqqq+linewidth(0.7));
draw( (742.499999990193,0.24443532284375813) -- (744.9999999896406,0.24107593769091062),ccqqqq+linewidth(0.7));
draw( (744.9999999896406,0.24107593769091062) -- (747.4999999890882,0.23772603300153006),ccqqqq+linewidth(0.7));
draw( (747.4999999890882,0.23772603300153006) -- (749.9999999885358,0.23438580257388908),ccqqqq+linewidth(0.7));
draw( (749.9999999885358,0.23438580257388908) -- (752.4999999879834,0.23105544223701857),ccqqqq+linewidth(0.7));
draw( (752.4999999879834,0.23105544223701857) -- (754.999999987431,0.22773514992313026),ccqqqq+linewidth(0.7));
draw( (754.999999987431,0.22773514992313026) -- (757.4999999868786,0.22442512574153284),ccqqqq+linewidth(0.7));
draw( (757.4999999868786,0.22442512574153284) -- (759.9999999863262,0.22112557205405392),ccqqqq+linewidth(0.7));
draw( (759.9999999863262,0.22112557205405392) -- (762.4999999857738,0.21783669355196777),ccqqqq+linewidth(0.7));
draw( (762.4999999857738,0.21783669355196777) -- (764.9999999852214,0.2145586973344722),ccqqqq+linewidth(0.7));
draw( (764.9999999852214,0.2145586973344722) -- (767.499999984669,0.21129179298867584),ccqqqq+linewidth(0.7));
draw( (767.499999984669,0.21129179298867584) -- (769.9999999841166,0.20803619267116402),ccqqqq+linewidth(0.7));
draw( (769.9999999841166,0.20803619267116402) -- (772.4999999835642,0.20479211119111207),ccqqqq+linewidth(0.7));
draw( (772.4999999835642,0.20479211119111207) -- (774.9999999830118,0.2015597660949895),ccqqqq+linewidth(0.7));
draw( (774.9999999830118,0.2015597660949895) -- (777.4999999824594,0.19833937775283472),ccqqqq+linewidth(0.7));
draw( (777.4999999824594,0.19833937775283472) -- (779.999999981907,0.19513116944614795),ccqqqq+linewidth(0.7));
draw( (779.999999981907,0.19513116944614795) -- (782.4999999813546,0.19193536745739548),ccqqqq+linewidth(0.7));
draw( (782.4999999813546,0.19193536745739548) -- (784.9999999808022,0.18875220116112457),ccqqqq+linewidth(0.7));
draw( (784.9999999808022,0.18875220116112457) -- (787.4999999802498,0.1855819031167365),ccqqqq+linewidth(0.7));
draw( (787.4999999802498,0.1855819031167365) -- (789.9999999796973,0.18242470916289844),ccqqqq+linewidth(0.7));
draw( (789.9999999796973,0.18242470916289844) -- (792.499999979145,0.17928085851361375),ccqqqq+linewidth(0.7));
draw( (792.499999979145,0.17928085851361375) -- (794.9999999785925,0.17615059385598342),ccqqqq+linewidth(0.7));
draw( (794.9999999785925,0.17615059385598342) -- (797.4999999780401,0.17303416144963613),ccqqqq+linewidth(0.7));
draw( (797.4999999780401,0.17303416144963613) -- (799.9999999774877,0.16993181122787204),ccqqqq+linewidth(0.7));
draw( (799.9999999774877,0.16993181122787204) -- (802.4999999769353,0.16684379690050422),ccqqqq+linewidth(0.7));
draw( (802.4999999769353,0.16684379690050422) -- (804.9999999763829,0.16377037605843503),ccqqqq+linewidth(0.7));
draw( (804.9999999763829,0.16377037605843503) -- (807.4999999758305,0.1607118102799684),ccqqqq+linewidth(0.7));
draw( (807.4999999758305,0.1607118102799684) -- (809.9999999752781,0.1576683652388604),ccqqqq+linewidth(0.7));
draw( (809.9999999752781,0.1576683652388604) -- (812.4999999747257,0.1546403108141361),ccqqqq+linewidth(0.7));
draw( (812.4999999747257,0.1546403108141361) -- (814.9999999741733,0.1516279212016869),ccqqqq+linewidth(0.7));
draw( (814.9999999741733,0.1516279212016869) -- (817.4999999736209,0.14863147502763596),ccqqqq+linewidth(0.7));
draw( (817.4999999736209,0.14863147502763596) -- (819.9999999730685,0.1456512554635172),ccqqqq+linewidth(0.7));
draw( (819.9999999730685,0.1456512554635172) -- (822.4999999725161,0.14268755034325498),ccqqqq+linewidth(0.7));
draw( (822.4999999725161,0.14268755034325498) -- (824.9999999719637,0.13974065228195776),ccqqqq+linewidth(0.7));
draw( (824.9999999719637,0.13974065228195776) -- (827.4999999714113,0.13681085879657373),ccqqqq+linewidth(0.7));
draw( (827.4999999714113,0.13681085879657373) -- (829.9999999708589,0.1338984724283545),ccqqqq+linewidth(0.7));
draw( (829.9999999708589,0.1338984724283545) -- (832.4999999703065,0.1310038008672103),ccqqqq+linewidth(0.7));
draw( (832.4999999703065,0.1310038008672103) -- (834.9999999697541,0.12812715707791855),ccqqqq+linewidth(0.7));
draw( (834.9999999697541,0.12812715707791855) -- (837.4999999692017,0.12526885942820964),ccqqqq+linewidth(0.7));
draw( (837.4999999692017,0.12526885942820964) -- (839.9999999686493,0.12242923181877186),ccqqqq+linewidth(0.7));
draw( (839.9999999686493,0.12242923181877186) -- (842.4999999680969,0.11960860381514071),ccqqqq+linewidth(0.7));
draw( (842.4999999680969,0.11960860381514071) -- (844.9999999675445,0.11680731078151307),ccqqqq+linewidth(0.7));
draw( (844.9999999675445,0.11680731078151307) -- (847.499999966992,0.11402569401650631),ccqqqq+linewidth(0.7));
draw( (847.499999966992,0.11402569401650631) -- (849.9999999664396,0.11126410089084393),ccqqqq+linewidth(0.7));
draw( (849.9999999664396,0.11126410089084393) -- (852.4999999658872,0.10852288498699991),ccqqqq+linewidth(0.7));
draw( (852.4999999658872,0.10852288498699991) -- (854.9999999653348,0.10580240624083603),ccqqqq+linewidth(0.7));
draw( (854.9999999653348,0.10580240624083603) -- (857.4999999647824,0.10310303108518126),ccqqqq+linewidth(0.7));
draw( (857.4999999647824,0.10310303108518126) -- (859.99999996423,0.10042513259542996),ccqqqq+linewidth(0.7));
draw( (859.99999996423,0.10042513259542996) -- (862.4999999636776,0.09776909063714645),ccqqqq+linewidth(0.7));
draw( (862.4999999636776,0.09776909063714645) -- (864.9999999631252,0.09513529201567072),ccqqqq+linewidth(0.7));
draw( (864.9999999631252,0.09513529201567072) -- (867.4999999625728,0.0925241306277702),ccqqqq+linewidth(0.7));
draw( (867.4999999625728,0.0925241306277702) -- (869.9999999620204,0.08993600761532061),ccqqqq+linewidth(0.7));
draw( (869.9999999620204,0.08993600761532061) -- (872.499999961468,0.08737133152105447),ccqqqq+linewidth(0.7));
draw( (872.499999961468,0.08737133152105447) -- (874.9999999609156,0.08483051844636469),ccqqqq+linewidth(0.7));
draw( (874.9999999609156,0.08483051844636469) -- (877.4999999603632,0.08231399221118751),ccqqqq+linewidth(0.7));
draw( (877.4999999603632,0.08231399221118751) -- (879.9999999598108,0.07982218451599743),ccqqqq+linewidth(0.7));
draw( (879.9999999598108,0.07982218451599743) -- (882.4999999592584,0.07735553510586124),ccqqqq+linewidth(0.7));
draw( (882.4999999592584,0.07735553510586124) -- (884.999999958706,0.07491449193665956),ccqqqq+linewidth(0.7));
draw( (884.999999958706,0.07491449193665956) -- (887.4999999581536,0.07249951134340415),ccqqqq+linewidth(0.7));
draw( (887.4999999581536,0.07249951134340415) -- (889.9999999576012,0.07011105821069252),ccqqqq+linewidth(0.7));
draw( (889.9999999576012,0.07011105821069252) -- (892.4999999570488,0.06774960614533754),ccqqqq+linewidth(0.7));
draw( (892.4999999570488,0.06774960614533754) -- (894.9999999564964,0.06541563765114544),ccqqqq+linewidth(0.7));
draw( (894.9999999564964,0.06541563765114544) -- (897.499999955944,0.06310964430587473),ccqqqq+linewidth(0.7));
draw( (897.499999955944,0.06310964430587473) -- (899.9999999553916,0.0608321269403822),ccqqqq+linewidth(0.7));
draw( (899.9999999553916,0.0608321269403822) -- (902.4999999548392,0.05858359581996675),ccqqqq+linewidth(0.7));
draw( (902.4999999548392,0.05858359581996675) -- (904.9999999542867,0.05636457082794499),ccqqqq+linewidth(0.7));
draw( (904.9999999542867,0.05636457082794499) -- (907.4999999537343,0.054175581651423776),ccqqqq+linewidth(0.7));
draw( (907.4999999537343,0.054175581651423776) -- (909.9999999531819,0.05201716796934486),ccqqqq+linewidth(0.7));
draw( (909.9999999531819,0.05201716796934486) -- (912.4999999526295,0.049889879642741164),ccqqqq+linewidth(0.7));
draw( (912.4999999526295,0.049889879642741164) -- (914.9999999520771,0.047794276907292765),ccqqqq+linewidth(0.7));
draw( (914.9999999520771,0.047794276907292765) -- (917.4999999515247,0.045730930568126826),ccqqqq+linewidth(0.7));
draw( (917.4999999515247,0.045730930568126826) -- (919.9999999509723,0.043700422196931044),ccqqqq+linewidth(0.7));
draw( (919.9999999509723,0.043700422196931044) -- (922.4999999504199,0.04170334433134126),ccqqqq+linewidth(0.7));
draw( (922.4999999504199,0.04170334433134126) -- (924.9999999498675,0.039740300676673),ccqqqq+linewidth(0.7));
draw( (924.9999999498675,0.039740300676673) -- (927.4999999493151,0.0378119063099267),ccqqqq+linewidth(0.7));
draw( (927.4999999493151,0.0378119063099267) -- (929.9999999487627,0.035918787886189296),ccqqqq+linewidth(0.7));
draw( (929.9999999487627,0.035918787886189296) -- (932.4999999482103,0.03406158384733127),ccqqqq+linewidth(0.7));
draw( (932.4999999482103,0.03406158384733127) -- (934.9999999476579,0.03224094463310445),ccqqqq+linewidth(0.7));
draw( (934.9999999476579,0.03224094463310445) -- (937.4999999471055,0.030457532894581973),ccqqqq+linewidth(0.7));
draw( (937.4999999471055,0.030457532894581973) -- (939.9999999465531,0.028712023709998924),ccqqqq+linewidth(0.7));
draw( (939.9999999465531,0.028712023709998924) -- (942.4999999460007,0.02700510480299778),ccqqqq+linewidth(0.7));
draw( (942.4999999460007,0.02700510480299778) -- (944.9999999454483,0.025337476763257705),ccqqqq+linewidth(0.7));
draw( (944.9999999454483,0.025337476763257705) -- (947.4999999448959,0.0237098532695712),ccqqqq+linewidth(0.7));
draw( (947.4999999448959,0.0237098532695712) -- (949.9999999443435,0.022122961315336642),ccqqqq+linewidth(0.7));
draw( (949.9999999443435,0.022122961315336642) -- (952.4999999437911,0.02057754143652124),ccqqqq+linewidth(0.7));
draw( (952.4999999437911,0.02057754143652124) -- (954.9999999432387,0.019074347942039438),ccqqqq+linewidth(0.7));
draw( (954.9999999432387,0.019074347942039438) -- (957.4999999426863,0.01761414914665904),ccqqqq+linewidth(0.7));
draw( (957.4999999426863,0.01761414914665904) -- (959.9999999421339,0.016197727606329426),ccqqqq+linewidth(0.7));
draw( (959.9999999421339,0.016197727606329426) -- (962.4999999415815,0.014825880356061916),ccqqqq+linewidth(0.7));
draw( (962.4999999415815,0.014825880356061916) -- (964.999999941029,0.013499419150279746),ccqqqq+linewidth(0.7));
draw( (964.999999941029,0.013499419150279746) -- (967.4999999404766,0.012219170705690896),ccqqqq+linewidth(0.7));
draw( (967.4999999404766,0.012219170705690896) -- (969.9999999399242,0.010985976946718878),ccqqqq+linewidth(0.7));
draw( (969.9999999399242,0.010985976946718878) -- (972.4999999393718,0.009800695253447955),ccqqqq+linewidth(0.7));
draw( (972.4999999393718,0.009800695253447955) -- (974.9999999388194,0.008664198712136972),ccqqqq+linewidth(0.7));
draw( (974.9999999388194,0.008664198712136972) -- (977.499999938267,0.007577376368316013),ccqqqq+linewidth(0.7));
draw( (977.499999938267,0.007577376368316013) -- (979.9999999377146,0.006541133482421024),ccqqqq+linewidth(0.7));
draw( (979.9999999377146,0.006541133482421024) -- (982.4999999371622,0.005556391788081427),ccqqqq+linewidth(0.7));
draw( (982.4999999371622,0.005556391788081427) -- (984.9999999366098,0.004624089752962135),ccqqqq+linewidth(0.7));
draw( (984.9999999366098,0.004624089752962135) -- (987.4999999360574,0.003745182842252781),ccqqqq+linewidth(0.7));
draw( (987.4999999360574,0.003745182842252781) -- (989.999999935505,0.0029206437847726363),ccqqqq+linewidth(0.7));
draw( (989.999999935505,0.0029206437847726363) -- (992.4999999349526,0.0021514628417218518),ccqqqq+linewidth(0.7));
draw( (992.4999999349526,0.0021514628417218518) -- (994.9999999344002,0.0014386480780954614),ccqqqq+linewidth(0.7));
draw( (994.9999999344002,0.0014386480780954614) -- (997.4999999338478,0),ccqqqq+linewidth(0.7));
;
draw( (0,1.000479999971021) -- (0,1.000479999971021),qqwuqq+linewidth(0.7));
draw( (0,1.000479999971021) -- (2.000000153818027,1.00007268372521),qqwuqq+linewidth(0.7));
draw( (2.000000153818027,1.00007268372521) -- (4.000000153376089,0.9996015057689003),qqwuqq+linewidth(0.7));
draw( (4.000000153376089,0.9996015057689003) -- (6.000000152934151,0.9990655174202536),qqwuqq+linewidth(0.7));
draw( (6.000000152934151,0.9990655174202536) -- (8.000000152492213,0.9984638344089558),qqwuqq+linewidth(0.7));
draw( (8.000000152492213,0.9984638344089558) -- (10.000000152050275,0.9977956354709085),qqwuqq+linewidth(0.7));
draw( (10.000000152050275,0.9977956354709085) -- (12.000000151608337,0.9970601609606927),qqwuqq+linewidth(0.7));
draw( (12.000000151608337,0.9970601609606927) -- (14.000000151166399,0.9962567114816718),qqwuqq+linewidth(0.7));
draw( (14.000000151166399,0.9962567114816718) -- (16.00000015072446,0.9953846465336051),qqwuqq+linewidth(0.7));
draw( (16.00000015072446,0.9953846465336051) -- (18.00000015028252,0.9944433831776388),qqwuqq+linewidth(0.7));
draw( (18.00000015028252,0.9944433831776388) -- (20.00000014984058,0.9934323947185429),qqwuqq+linewidth(0.7));
draw( (20.00000014984058,0.9934323947185429) -- (22.00000014939864,0.9923512094040703),qqwuqq+linewidth(0.7));
draw( (22.00000014939864,0.9923512094040703) -- (24.0000001489567,0.9911994091412977),qqwuqq+linewidth(0.7));
draw( (24.0000001489567,0.9911994091412977) -- (26.00000014851476,0.989976628229831),qqwuqq+linewidth(0.7));
draw( (26.00000014851476,0.989976628229831) -- (28.00000014807282,0.9886825521117382),qqwuqq+linewidth(0.7));
draw( (28.00000014807282,0.9886825521117382) -- (30.00000014763088,0.9873169161380845),qqwuqq+linewidth(0.7));
draw( (30.00000014763088,0.9873169161380845) -- (32.00000014718894,0.9858795043519432),qqwuqq+linewidth(0.7));
draw( (32.00000014718894,0.9858795043519432) -- (34.000000146747006,0.9843701482877505),qqwuqq+linewidth(0.7));
draw( (34.000000146747006,0.9843701482877505) -- (36.00000014630507,0.9827887257868857),qqwuqq+linewidth(0.7));
draw( (36.00000014630507,0.9827887257868857) -- (38.00000014586313,0.9811351598293394),qqwuqq+linewidth(0.7));
draw( (38.00000014586313,0.9811351598293394) -- (40.0000001454212,0.9794094173813542),qqwuqq+linewidth(0.7));
draw( (40.0000001454212,0.9794094173813542) -- (42.00000014497926,0.9776115082589045),qqwuqq+linewidth(0.7));
draw( (42.00000014497926,0.9776115082589045) -- (44.000000144537324,0.9757414840068966),qqwuqq+linewidth(0.7));
draw( (44.000000144537324,0.9757414840068966) -- (46.00000014409539,0.9737994367939581),qqwuqq+linewidth(0.7));
draw( (46.00000014409539,0.9737994367939581) -- (48.00000014365345,0.9717854983226996),qqwuqq+linewidth(0.7));
draw( (48.00000014365345,0.9717854983226996) -- (50.000000143211516,0.9696998387553183),qqwuqq+linewidth(0.7));
draw( (50.000000143211516,0.9696998387553183) -- (52.00000014276958,0.9675426656544259),qqwuqq+linewidth(0.7));
draw( (52.00000014276958,0.9675426656544259) -- (54.00000014232764,0.9653142229389733),qqwuqq+linewidth(0.7));
draw( (54.00000014232764,0.9653142229389733) -- (56.00000014188571,0.9630147898551524),qqwuqq+linewidth(0.7));
draw( (56.00000014188571,0.9630147898551524) -- (58.00000014144377,0.9606446799621544),qqwuqq+linewidth(0.7));
draw( (58.00000014144377,0.9606446799621544) -- (60.000000141001834,0.9582042401326591),qqwuqq+linewidth(0.7));
draw( (60.000000141001834,0.9582042401326591) -- (62.0000001405599,0.9556938495679379),qqwuqq+linewidth(0.7));
draw( (62.0000001405599,0.9556938495679379) -- (64.00000014011796,0.9531139188274479),qqwuqq+linewidth(0.7));
draw( (64.00000014011796,0.9531139188274479) -- (66.00000013967602,0.9504648888727999),qqwuqq+linewidth(0.7));
draw( (66.00000013967602,0.9504648888727999) -- (68.00000013923407,0.9477472301259744),qqwuqq+linewidth(0.7));
draw( (68.00000013923407,0.9477472301259744) -- (70.00000013879213,0.944961441541676),qqwuqq+linewidth(0.7));
draw( (70.00000013879213,0.944961441541676) -- (72.00000013835019,0.9421080496936971),qqwuqq+linewidth(0.7));
draw( (72.00000013835019,0.9421080496936971) -- (74.00000013790824,0.9391876078751807),qqwuqq+linewidth(0.7));
draw( (74.00000013790824,0.9391876078751807) -- (76.0000001374663,0.936200695212662),qqwuqq+linewidth(0.7));
draw( (76.0000001374663,0.936200695212662) -- (78.00000013702436,0.9331479157937673),qqwuqq+linewidth(0.7));
draw( (78.00000013702436,0.9331479157937673) -- (80.00000013658241,0.9300298978084622),qqwuqq+linewidth(0.7));
draw( (80.00000013658241,0.9300298978084622) -- (82.00000013614047,0.9268472927037216),qqwuqq+linewidth(0.7));
draw( (82.00000013614047,0.9268472927037216) -- (84.00000013569853,0.9236007743515169),qqwuqq+linewidth(0.7));
draw( (84.00000013569853,0.9236007743515169) -- (86.00000013525658,0.9202910382299947),qqwuqq+linewidth(0.7));
draw( (86.00000013525658,0.9202910382299947) -- (88.00000013481464,0.91691880061774),qqwuqq+linewidth(0.7));
draw( (88.00000013481464,0.91691880061774) -- (90.0000001343727,0.9134847978010036),qqwuqq+linewidth(0.7));
draw( (90.0000001343727,0.9134847978010036) -- (92.00000013393075,0.909989785293782),qqwuqq+linewidth(0.7));
draw( (92.00000013393075,0.909989785293782) -- (94.00000013348881,0.9064345370706373),qqwuqq+linewidth(0.7));
draw( (94.00000013348881,0.9064345370706373) -- (96.00000013304687,0.9028198448121394),qqwuqq+linewidth(0.7));
draw( (96.00000013304687,0.9028198448121394) -- (98.00000013260492,0.8991465171628243),qqwuqq+linewidth(0.7));
draw( (98.00000013260492,0.8991465171628243) -- (100.00000013216298,0.8954153790015492),qqwuqq+linewidth(0.7));
draw( (100.00000013216298,0.8954153790015492) -- (102.00000013172104,0.8916272707241367),qqwuqq+linewidth(0.7));
draw( (102.00000013172104,0.8916272707241367) -- (104.0000001312791,0.8877830475381967),qqwuqq+linewidth(0.7));
draw( (104.0000001312791,0.8877830475381967) -- (106.00000013083715,0.8838835787700127),qqwuqq+linewidth(0.7));
draw( (106.00000013083715,0.8838835787700127) -- (108.00000013039521,0.8799297471833827),qqwuqq+linewidth(0.7));
draw( (108.00000013039521,0.8799297471833827) -- (110.00000012995326,0.8759224483103069),qqwuqq+linewidth(0.7));
draw( (110.00000012995326,0.8759224483103069) -- (112.00000012951132,0.8718625897934096),qqwuqq+linewidth(0.7));
draw( (112.00000012951132,0.8718625897934096) -- (114.00000012906938,0.8677510907399859),qqwuqq+linewidth(0.7));
draw( (114.00000012906938,0.8677510907399859) -- (116.00000012862743,0.8635888810875687),qqwuqq+linewidth(0.7));
draw( (116.00000012862743,0.8635888810875687) -- (118.00000012818549,0.859376900980902),qqwuqq+linewidth(0.7));
draw( (118.00000012818549,0.859376900980902) -- (120.00000012774355,0.855116100160216),qqwuqq+linewidth(0.7));
draw( (120.00000012774355,0.855116100160216) -- (122.0000001273016,0.8508074373606979),qqwuqq+linewidth(0.7));
draw( (122.0000001273016,0.8508074373606979) -- (124.00000012685966,0.8464518797230468),qqwuqq+linewidth(0.7));
draw( (124.00000012685966,0.8464518797230468) -- (126.00000012641772,0.8420504022150109),qqwuqq+linewidth(0.7));
draw( (126.00000012641772,0.8420504022150109) -- (128.0000001259758,0.8376039870637981),qqwuqq+linewidth(0.7));
draw( (128.0000001259758,0.8376039870637981) -- (130.00000012553386,0.8331136231992556),qqwuqq+linewidth(0.7));
draw( (130.00000012553386,0.8331136231992556) -- (132.00000012509193,0.8285803057077133),qqwuqq+linewidth(0.7));
draw( (132.00000012509193,0.8285803057077133) -- (134.00000012465,0.8240050352963852),qqwuqq+linewidth(0.7));
draw( (134.00000012465,0.8240050352963852) -- (136.00000012420807,0.8193888177682276),qqwuqq+linewidth(0.7));
draw( (136.00000012420807,0.8193888177682276) -- (138.00000012376614,0.8147326635071455),qqwuqq+linewidth(0.7));
draw( (138.00000012376614,0.8147326635071455) -- (140.0000001233242,0.810037586973449),qqwuqq+linewidth(0.7));
draw( (140.0000001233242,0.810037586973449) -- (142.00000012288228,0.8053046062094525),qqwuqq+linewidth(0.7));
draw( (142.00000012288228,0.8053046062094525) -- (144.00000012244035,0.8005347423551179),qqwuqq+linewidth(0.7));
draw( (144.00000012244035,0.8005347423551179) -- (146.00000012199843,0.7957290191736373),qqwuqq+linewidth(0.7));
draw( (146.00000012199843,0.7957290191736373) -- (148.0000001215565,0.7908884625868533),qqwuqq+linewidth(0.7));
draw( (148.0000001215565,0.7908884625868533) -- (150.00000012111457,0.7860141002204205),qqwuqq+linewidth(0.7));
draw( (150.00000012111457,0.7860141002204205) -- (152.00000012067264,0.7811069609586009),qqwuqq+linewidth(0.7));
draw( (152.00000012067264,0.7811069609586009) -- (154.0000001202307,0.7761680745085967),qqwuqq+linewidth(0.7));
draw( (154.0000001202307,0.7761680745085967) -- (156.00000011978878,0.7711984709743228),qqwuqq+linewidth(0.7));
draw( (156.00000011978878,0.7711984709743228) -- (158.00000011934685,0.7661991804395147),qqwuqq+linewidth(0.7));
draw( (158.00000011934685,0.7661991804395147) -- (160.00000011890492,0.761171232560077),qqwuqq+linewidth(0.7));
draw( (160.00000011890492,0.761171232560077) -- (162.000000118463,0.7561156561655723),qqwuqq+linewidth(0.7));
draw( (162.000000118463,0.7561156561655723) -- (164.00000011802106,0.7510334788697535),qqwuqq+linewidth(0.7));
draw( (164.00000011802106,0.7510334788697535) -- (166.00000011757913,0.7459257266900418),qqwuqq+linewidth(0.7));
draw( (166.00000011757913,0.7459257266900418) -- (168.0000001171372,0.7407934236758517),qqwuqq+linewidth(0.7));
draw( (168.0000001171372,0.7407934236758517) -- (170.00000011669528,0.7356375915456713),qqwuqq+linewidth(0.7));
draw( (170.00000011669528,0.7356375915456713) -- (172.00000011625335,0.7304592493327948),qqwuqq+linewidth(0.7));
draw( (172.00000011625335,0.7304592493327948) -- (174.00000011581142,0.7252594130396186),qqwuqq+linewidth(0.7));
draw( (174.00000011581142,0.7252594130396186) -- (176.0000001153695,0.7200390953003993),qqwuqq+linewidth(0.7));
draw( (176.0000001153695,0.7200390953003993) -- (178.00000011492756,0.7147993050523859),qqwuqq+linewidth(0.7));
draw( (178.00000011492756,0.7147993050523859) -- (180.00000011448563,0.7095410472152246),qqwuqq+linewidth(0.7));
draw( (180.00000011448563,0.7095410472152246) -- (182.0000001140437,0.7042653223785484),qqwuqq+linewidth(0.7));
draw( (182.0000001140437,0.7042653223785484) -- (184.00000011360177,0.6989731264976541),qqwuqq+linewidth(0.7));
draw( (184.00000011360177,0.6989731264976541) -- (186.00000011315984,0.6936654505971764),qqwuqq+linewidth(0.7));
draw( (186.00000011315984,0.6936654505971764) -- (188.0000001127179,0.6883432804826651),qqwuqq+linewidth(0.7));
draw( (188.0000001127179,0.6883432804826651) -- (190.00000011227598,0.6830075964599736),qqwuqq+linewidth(0.7));
draw( (190.00000011227598,0.6830075964599736) -- (192.00000011183405,0.6776593730623676),qqwuqq+linewidth(0.7));
draw( (192.00000011183405,0.6776593730623676) -- (194.00000011139213,0.6722995787852633),qqwuqq+linewidth(0.7));
draw( (194.00000011139213,0.6722995787852633) -- (196.0000001109502,0.6669291758285032),qqwuqq+linewidth(0.7));
draw( (196.0000001109502,0.6669291758285032) -- (198.00000011050827,0.6615491198460798),qqwuqq+linewidth(0.7));
draw( (198.00000011050827,0.6615491198460798) -- (200.00000011006634,0.6561603597032173),qqwuqq+linewidth(0.7));
draw( (200.00000011006634,0.6561603597032173) -- (202.0000001096244,0.6507638372407241),qqwuqq+linewidth(0.7));
draw( (202.0000001096244,0.6507638372407241) -- (204.00000010918248,0.6453604870465228),qqwuqq+linewidth(0.7));
draw( (204.00000010918248,0.6453604870465228) -- (206.00000010874055,0.639951236234274),qqwuqq+linewidth(0.7));
draw( (206.00000010874055,0.639951236234274) -- (208.00000010829862,0.6345370042290028),qqwuqq+linewidth(0.7));
draw( (208.00000010829862,0.6345370042290028) -- (210.0000001078567,0.6291187025596422),qqwuqq+linewidth(0.7));
draw( (210.0000001078567,0.6291187025596422) -- (212.00000010741476,0.6236972346584053),qqwuqq+linewidth(0.7));
draw( (212.00000010741476,0.6236972346584053) -- (214.00000010697283,0.6182734956669),qqwuqq+linewidth(0.7));
draw( (214.00000010697283,0.6182734956669) -- (216.0000001065309,0.6128483722489015),qqwuqq+linewidth(0.7));
draw( (216.0000001065309,0.6128483722489015) -- (218.00000010608898,0.6074227424096942),qqwuqq+linewidth(0.7));
draw( (218.00000010608898,0.6074227424096942) -- (220.00000010564705,0.6019974753218994),qqwuqq+linewidth(0.7));
draw( (220.00000010564705,0.6019974753218994) -- (222.00000010520512,0.5965734311577057),qqwuqq+linewidth(0.7));
draw( (222.00000010520512,0.5965734311577057) -- (224.0000001047632,0.5911514609274126),qqwuqq+linewidth(0.7));
draw( (224.0000001047632,0.5911514609274126) -- (226.00000010432126,0.5857324063242101),qqwuqq+linewidth(0.7));
draw( (226.00000010432126,0.5857324063242101) -- (228.00000010387933,0.5803170995751059),qqwuqq+linewidth(0.7));
draw( (228.00000010387933,0.5803170995751059) -- (230.0000001034374,0.5749063632979188),qqwuqq+linewidth(0.7));
draw( (230.0000001034374,0.5749063632979188) -- (232.00000010299547,0.5695010103642555),qqwuqq+linewidth(0.7));
draw( (232.00000010299547,0.5695010103642555) -- (234.00000010255354,0.5641018437683919),qqwuqq+linewidth(0.7));
draw( (234.00000010255354,0.5641018437683919) -- (236.0000001021116,0.5587096565019697),qqwuqq+linewidth(0.7));
draw( (236.0000001021116,0.5587096565019697) -- (238.00000010166968,0.5533252314344362),qqwuqq+linewidth(0.7));
draw( (238.00000010166968,0.5533252314344362) -- (240.00000010122776,0.5479493411991408),qqwuqq+linewidth(0.7));
draw( (240.00000010122776,0.5479493411991408) -- (242.00000010078583,0.5425827480850096),qqwuqq+linewidth(0.7));
draw( (242.00000010078583,0.5425827480850096) -- (244.0000001003439,0.5372262039337157),qqwuqq+linewidth(0.7));
draw( (244.0000001003439,0.5372262039337157) -- (246.00000009990197,0.5318804500422739),qqwuqq+linewidth(0.7));
draw( (246.00000009990197,0.5318804500422739) -- (248.00000009946004,0.5265462170709703),qqwuqq+linewidth(0.7));
draw( (248.00000009946004,0.5265462170709703) -- (250.0000000990181,0.5212242249565547),qqwuqq+linewidth(0.7));
draw( (250.0000000990181,0.5212242249565547) -- (252.00000009857618,0.5159151828306184),qqwuqq+linewidth(0.7));
draw( (252.00000009857618,0.5159151828306184) -- (254.00000009813425,0.5106197889430755),qqwuqq+linewidth(0.7));
draw( (254.00000009813425,0.5106197889430755) -- (256.0000000976923,0.5053387305906752),qqwuqq+linewidth(0.7));
draw( (256.0000000976923,0.5053387305906752) -- (258.00000009725034,0.5000726840504636),qqwuqq+linewidth(0.7));
draw( (258.00000009725034,0.5000726840504636) -- (260.0000000968084,0.49482231451812414),qqwuqq+linewidth(0.7));
draw( (260.0000000968084,0.49482231451812414) -- (262.0000000963664,0.48958827605111804),qqwuqq+linewidth(0.7));
draw( (262.0000000963664,0.48958827605111804) -- (264.00000009592446,0.4843712115165489),qqwuqq+linewidth(0.7));
draw( (264.00000009592446,0.4843712115165489) -- (266.0000000954825,0.47917175254367667),qqwuqq+linewidth(0.7));
draw( (266.0000000954825,0.47917175254367667) -- (268.00000009504055,0.47399051948101123),qqwuqq+linewidth(0.7));
draw( (268.00000009504055,0.47399051948101123) -- (270.0000000945986,0.4688281213579016),qqwuqq+linewidth(0.7));
draw( (270.0000000945986,0.4688281213579016) -- (272.00000009415663,0.46368515585056164),qqwuqq+linewidth(0.7));
draw( (272.00000009415663,0.46368515585056164) -- (274.0000000937147,0.4585622092524444),qqwuqq+linewidth(0.7));
draw( (274.0000000937147,0.4585622092524444) -- (276.0000000932727,0.45345985644890174),qqwuqq+linewidth(0.7));
draw( (276.0000000932727,0.45345985644890174) -- (278.00000009283076,0.44837866089605694),qqwuqq+linewidth(0.7));
draw( (278.00000009283076,0.44837866089605694) -- (280.0000000923888,0.4433191746038127),qqwuqq+linewidth(0.7));
draw( (280.0000000923888,0.4433191746038127) -- (282.00000009194684,0.438281938122929),qqwuqq+linewidth(0.7));
draw( (282.00000009194684,0.438281938122929) -- (284.0000000915049,0.43326748053609937),qqwuqq+linewidth(0.7));
draw( (284.0000000915049,0.43326748053609937) -- (286.00000009106293,0.42827631945295),qqwuqq+linewidth(0.7));
draw( (286.00000009106293,0.42827631945295) -- (288.000000090621,0.4233089610089024),qqwuqq+linewidth(0.7));
draw( (288.000000090621,0.4233089610089024) -- (290.000000090179,0.4183658998678153),qqwuqq+linewidth(0.7));
draw( (290.000000090179,0.4183658998678153) -- (292.00000008973706,0.4134476192283541),qqwuqq+linewidth(0.7));
draw( (292.00000008973706,0.4134476192283541) -- (294.0000000892951,0.40855459083400164),qqwuqq+linewidth(0.7));
draw( (294.0000000892951,0.40855459083400164) -- (296.00000008885314,0.4036872749866609),qqwuqq+linewidth(0.7));
draw( (296.00000008885314,0.4036872749866609) -- (298.0000000884112,0.3988461205637623),qqwuqq+linewidth(0.7));
draw( (298.0000000884112,0.3988461205637623) -- (300.0000000879692,0.39403156503882475),qqwuqq+linewidth(0.7));
draw( (300.0000000879692,0.39403156503882475) -- (302.00000008752727,0.3892440345053983),qqwuqq+linewidth(0.7));
draw( (302.00000008752727,0.3892440345053983) -- (304.0000000870853,0.38448394370431616),qqwuqq+linewidth(0.7));
draw( (304.0000000870853,0.38448394370431616) -- (306.00000008664335,0.37975169605419706),qqwuqq+linewidth(0.7));
draw( (306.00000008664335,0.37975169605419706) -- (308.0000000862014,0.37504768368513586),qqwuqq+linewidth(0.7));
draw( (308.0000000862014,0.37504768368513586) -- (310.00000008575944,0.37037228747550766),qqwuqq+linewidth(0.7));
draw( (310.00000008575944,0.37037228747550766) -- (312.0000000853175,0.3657258770918253),qqwuqq+linewidth(0.7));
draw( (312.0000000853175,0.3657258770918253) -- (314.0000000848755,0.36110881103159215),qqwuqq+linewidth(0.7));
draw( (314.0000000848755,0.36110881103159215) -- (316.00000008443357,0.35652143666907693),qqwuqq+linewidth(0.7));
draw( (316.00000008443357,0.35652143666907693) -- (318.0000000839916,0.35196409030395404),qqwuqq+linewidth(0.7));
draw( (318.0000000839916,0.35196409030395404) -- (320.00000008354965,0.3474370972127454),qqwuqq+linewidth(0.7));
draw( (320.00000008354965,0.3474370972127454) -- (322.0000000831077,0.34294077170299847),qqwuqq+linewidth(0.7));
draw( (322.0000000831077,0.34294077170299847) -- (324.00000008266574,0.3384754171701457),qqwuqq+linewidth(0.7));
draw( (324.00000008266574,0.3384754171701457) -- (326.0000000822238,0.3340413261569743),qqwuqq+linewidth(0.7));
draw( (326.0000000822238,0.3340413261569743) -- (328.0000000817818,0.32963878041565803),qqwuqq+linewidth(0.7));
draw( (328.0000000817818,0.32963878041565803) -- (330.00000008133986,0.3252680509722758),qqwuqq+linewidth(0.7));
draw( (330.00000008133986,0.3252680509722758) -- (332.0000000808979,0.320929398193773),qqwuqq+linewidth(0.7));
draw( (332.0000000808979,0.320929398193773) -- (334.00000008045595,0.3166230718572903),qqwuqq+linewidth(0.7));
draw( (334.00000008045595,0.3166230718572903) -- (336.000000080014,0.31234931122181586),qqwuqq+linewidth(0.7));
draw( (336.000000080014,0.31234931122181586) -- (338.00000007957203,0.3081083451020909),qqwuqq+linewidth(0.7));
draw( (338.00000007957203,0.3081083451020909) -- (340.0000000791301,0.30390039194471713),qqwuqq+linewidth(0.7));
draw( (340.0000000791301,0.30390039194471713) -- (342.0000000786881,0.29972565990640704),qqwuqq+linewidth(0.7));
draw( (342.0000000786881,0.29972565990640704) -- (344.00000007824616,0.29558434693431385),qqwuqq+linewidth(0.7));
draw( (344.00000007824616,0.29558434693431385) -- (346.0000000778042,0.2914766408483974),qqwuqq+linewidth(0.7));
draw( (346.0000000778042,0.2914766408483974) -- (348.00000007736224,0.2874027194257564),qqwuqq+linewidth(0.7));
draw( (348.00000007736224,0.2874027194257564) -- (350.0000000769203,0.2833627504868787),qqwuqq+linewidth(0.7));
draw( (350.0000000769203,0.2833627504868787) -- (352.00000007647833,0.2793568919837533),qqwuqq+linewidth(0.7));
draw( (352.00000007647833,0.2793568919837533) -- (354.0000000760364,0.2753852920897855),qqwuqq+linewidth(0.7));
draw( (354.0000000760364,0.2753852920897855) -- (356.0000000755944,0.27144808929146413),qqwuqq+linewidth(0.7));
draw( (356.0000000755944,0.27144808929146413) -- (358.00000007515246,0.267545412481731),qqwuqq+linewidth(0.7));
draw( (358.00000007515246,0.267545412481731) -- (360.0000000747105,0.2636773810549847),qqwuqq+linewidth(0.7));
draw( (360.0000000747105,0.2636773810549847) -- (362.00000007426854,0.25984410500368554),qqwuqq+linewidth(0.7));
draw( (362.00000007426854,0.25984410500368554) -- (364.0000000738266,0.25604568501648856),qqwuqq+linewidth(0.7));
draw( (364.0000000738266,0.25604568501648856) -- (366.0000000733846,0.25228221257787137),qqwuqq+linewidth(0.7));
draw( (366.0000000733846,0.25228221257787137) -- (368.00000007294267,0.2485537700691868),qqwuqq+linewidth(0.7));
draw( (368.00000007294267,0.2485537700691868) -- (370.0000000725007,0.24486043087111037),qqwuqq+linewidth(0.7));
draw( (370.0000000725007,0.24486043087111037) -- (372.00000007205875,0.2412022594674046),qqwuqq+linewidth(0.7));
draw( (372.00000007205875,0.2412022594674046) -- (374.0000000716168,0.23757931154997985),qqwuqq+linewidth(0.7));
draw( (374.0000000716168,0.23757931154997985) -- (376.00000007117484,0.23399163412517257),qqwuqq+linewidth(0.7));
draw( (376.00000007117484,0.23399163412517257) -- (378.0000000707329,0.23043926562122005),qqwuqq+linewidth(0.7));
draw( (378.0000000707329,0.23043926562122005) -- (380.0000000702909,0.22692223599685551),qqwuqq+linewidth(0.7));
draw( (380.0000000702909,0.22692223599685551) -- (382.00000006984897,0.2234405668509953),qqwuqq+linewidth(0.7));
draw( (382.00000006984897,0.2234405668509953) -- (384.000000069407,0.21999427153345694),qqwuqq+linewidth(0.7));
draw( (384.000000069407,0.21999427153345694) -- (386.00000006896505,0.21658335525667458),qqwuqq+linewidth(0.7));
draw( (386.00000006896505,0.21658335525667458) -- (388.0000000685231,0.2132078152083454),qqwuqq+linewidth(0.7));
draw( (388.0000000685231,0.2132078152083454) -- (390.00000006808114,0.2098676406649748),qqwuqq+linewidth(0.7));
draw( (390.00000006808114,0.2098676406649748) -- (392.0000000676392,0.20656281310627492),qqwuqq+linewidth(0.7));
draw( (392.0000000676392,0.20656281310627492) -- (394.0000000671972,0.20329330633035697),qqwuqq+linewidth(0.7));
draw( (394.0000000671972,0.20329330633035697) -- (396.00000006675526,0.20005908656968707),qqwuqq+linewidth(0.7));
draw( (396.00000006675526,0.20005908656968707) -- (398.0000000663133,0.19686011260774844),qqwuqq+linewidth(0.7));
draw( (398.0000000663133,0.19686011260774844) -- (400.00000006587135,0.19369633589637764),qqwuqq+linewidth(0.7));
draw( (400.00000006587135,0.19369633589637764) -- (402.0000000654294,0.1905677006737211),qqwuqq+linewidth(0.7));
draw( (402.0000000654294,0.1905677006737211) -- (404.00000006498743,0.18747414408276758),qqwuqq+linewidth(0.7));
draw( (404.00000006498743,0.18747414408276758) -- (406.0000000645455,0.18441559629042703),qqwuqq+linewidth(0.7));
draw( (406.0000000645455,0.18441559629042703) -- (408.0000000641035,0.18139198060708622),qqwuqq+linewidth(0.7));
draw( (408.0000000641035,0.18139198060708622) -- (410.00000006366156,0.17840321360664174),qqwuqq+linewidth(0.7));
draw( (410.00000006366156,0.17840321360664174) -- (412.0000000632196,0.17544920524692106),qqwuqq+linewidth(0.7));
draw( (412.0000000632196,0.17544920524692106) -- (414.00000006277764,0.1725298589904741),qqwuqq+linewidth(0.7));
draw( (414.00000006277764,0.1725298589904741) -- (416.0000000623357,0.16964507192570974),qqwuqq+linewidth(0.7));
draw( (416.0000000623357,0.16964507192570974) -- (418.00000006189373,0.16679473488829943),qqwuqq+linewidth(0.7));
draw( (418.00000006189373,0.16679473488829943) -- (420.0000000614518,0.1639787325828376),qqwuqq+linewidth(0.7));
draw( (420.0000000614518,0.1639787325828376) -- (422.0000000610098,0.16119694370470736),qqwuqq+linewidth(0.7));
draw( (422.0000000610098,0.16119694370470736) -- (424.00000006056786,0.15844924106212244),qqwuqq+linewidth(0.7));
draw( (424.00000006056786,0.15844924106212244) -- (426.0000000601259,0.155735491698287),qqwuqq+linewidth(0.7));
draw( (426.0000000601259,0.155735491698287) -- (428.00000005968394,0.15305555701364598),qqwuqq+linewidth(0.7));
draw( (428.00000005968394,0.15305555701364598) -- (430.000000059242,0.1504092928882139),qqwuqq+linewidth(0.7));
draw( (430.000000059242,0.1504092928882139) -- (432.0000000588,0.14779654980388646),qqwuqq+linewidth(0.7));
draw( (432.0000000588,0.14779654980388646) -- (434.00000005835807,0.14521717296675546),qqwuqq+linewidth(0.7));
draw( (434.00000005835807,0.14521717296675546) -- (436.0000000579161,0.14267100242935793),qqwuqq+linewidth(0.7));
draw( (436.0000000579161,0.14267100242935793) -- (438.00000005747415,0.14015787321284087),qqwuqq+linewidth(0.7));
draw( (438.00000005747415,0.14015787321284087) -- (440.0000000570322,0.1376776154289839),qqwuqq+linewidth(0.7));
draw( (440.0000000570322,0.1376776154289839) -- (442.00000005659024,0.13523005440207359),qqwuqq+linewidth(0.7));
draw( (442.00000005659024,0.13523005440207359) -- (444.0000000561483,0.1328150107905693),qqwuqq+linewidth(0.7));
draw( (444.0000000561483,0.1328150107905693) -- (446.0000000557063,0.13043230070855122),qqwuqq+linewidth(0.7));
draw( (446.0000000557063,0.13043230070855122) -- (448.00000005526437,0.12808173584688615),qqwuqq+linewidth(0.7));
draw( (448.00000005526437,0.12808173584688615) -- (450.0000000548224,0.12576312359410646),qqwuqq+linewidth(0.7));
draw( (450.0000000548224,0.12576312359410646) -- (452.00000005438045,0.12347626715695885),qqwuqq+linewidth(0.7));
draw( (452.00000005438045,0.12347626715695885) -- (454.0000000539385,0.12122096568057592),qqwuqq+linewidth(0.7));
draw( (454.0000000539385,0.12122096568057592) -- (456.00000005349654,0.11899701436827131),qqwuqq+linewidth(0.7));
draw( (456.00000005349654,0.11899701436827131) -- (458.0000000530546,0.11680420460090168),qqwuqq+linewidth(0.7));
draw( (458.0000000530546,0.11680420460090168) -- (460.0000000526126,0.11464232405574742),qqwuqq+linewidth(0.7));
draw( (460.0000000526126,0.11464232405574742) -- (462.00000005217066,0.11251115682495907),qqwuqq+linewidth(0.7));
draw( (462.00000005217066,0.11251115682495907) -- (464.0000000517287,0.11041048353342543),qqwuqq+linewidth(0.7));
draw( (464.0000000517287,0.11041048353342543) -- (466.00000005128675,0.10834008145612999),qqwuqq+linewidth(0.7));
draw( (466.00000005128675,0.10834008145612999) -- (468.0000000508448,0.10629972463493498),qqwuqq+linewidth(0.7));
draw( (468.0000000508448,0.10629972463493498) -- (470.00000005040283,0.1042891839947145),qqwuqq+linewidth(0.7));
draw( (470.00000005040283,0.1042891839947145) -- (472.0000000499609,0.1023082274589267),qqwuqq+linewidth(0.7));
draw( (472.0000000499609,0.1023082274589267) -- (474.0000000495189,0.10035662006444743),qqwuqq+linewidth(0.7));
draw( (474.0000000495189,0.10035662006444743) -- (476.00000004907696,0.09843412407577472),qqwuqq+linewidth(0.7));
draw( (476.00000004907696,0.09843412407577472) -- (478.000000048635,0.09654049909848506),qqwuqq+linewidth(0.7));
draw( (478.000000048635,0.09654049909848506) -- (480.00000004819304,0.09467550219195306),qqwuqq+linewidth(0.7));
draw( (480.00000004819304,0.09467550219195306) -- (482.0000000477511,0.09283888798131518),qqwuqq+linewidth(0.7));
draw( (482.0000000477511,0.09283888798131518) -- (484.00000004730913,0.09103040876862428),qqwuqq+linewidth(0.7));
draw( (484.00000004730913,0.09103040876862428) -- (486.00000004686717,0.08924981464319207),qqwuqq+linewidth(0.7));
draw( (486.00000004686717,0.08924981464319207) -- (488.0000000464252,0.08749685359109421),qqwuqq+linewidth(0.7));
draw( (488.0000000464252,0.08749685359109421) -- (490.00000004598326,0.08577127160378892),qqwuqq+linewidth(0.7));
draw( (490.00000004598326,0.08577127160378892) -- (492.0000000455413,0.08407281278587728),qqwuqq+linewidth(0.7));
draw( (492.0000000455413,0.08407281278587728) -- (494.00000004509934,0.08240121946190643),qqwuqq+linewidth(0.7));
draw( (494.00000004509934,0.08240121946190643) -- (496.0000000446574,0.08075623228227236),qqwuqq+linewidth(0.7));
draw( (496.0000000446574,0.08075623228227236) -- (498.0000000442154,0.07913759032814982),qqwuqq+linewidth(0.7));
draw( (498.0000000442154,0.07913759032814982) -- (500.00000004377347,0.07754503121542666),qqwuqq+linewidth(0.7));
draw( (500.00000004377347,0.07754503121542666) -- (502.0000000433315,0.0759782911976685),qqwuqq+linewidth(0.7));
draw( (502.0000000433315,0.0759782911976685) -- (504.00000004288955,0.07443710526801142),qqwuqq+linewidth(0.7));
draw( (504.00000004288955,0.07443710526801142) -- (506.0000000424476,0.07292120726006351),qqwuqq+linewidth(0.7));
draw( (506.0000000424476,0.07292120726006351) -- (508.00000004200564,0.07143032994770035),qqwuqq+linewidth(0.7));
draw( (508.00000004200564,0.07143032994770035) -- (510.0000000415637,0.06996420514380808),qqwuqq+linewidth(0.7));
draw( (510.0000000415637,0.06996420514380808) -- (512.0000000411218,0.06852256379787838),qqwuqq+linewidth(0.7));
draw( (512.0000000411218,0.06852256379787838) -- (514.0000000406799,0.06710513609254468),qqwuqq+linewidth(0.7));
draw( (514.0000000406799,0.06710513609254468) -- (516.000000040238,0.06571165153891143),qqwuqq+linewidth(0.7));
draw( (516.000000040238,0.06571165153891143) -- (518.0000000397961,0.06434183907078264),qqwuqq+linewidth(0.7));
draw( (518.0000000397961,0.06434183907078264) -- (520.0000000393542,0.06299542713767065),qqwuqq+linewidth(0.7));
draw( (520.0000000393542,0.06299542713767065) -- (522.0000000389123,0.06167214379664249),qqwuqq+linewidth(0.7));
draw( (522.0000000389123,0.06167214379664249) -- (524.0000000384704,0.06037171680293049),qqwuqq+linewidth(0.7));
draw( (524.0000000384704,0.06037171680293049) -- (526.0000000380285,0.0590938736993501),qqwuqq+linewidth(0.7));
draw( (526.0000000380285,0.0590938736993501) -- (528.0000000375866,0.05783834190445747),qqwuqq+linewidth(0.7));
draw( (528.0000000375866,0.05783834190445747) -- (530.0000000371447,0.05660484879943106),qqwuqq+linewidth(0.7));
draw( (530.0000000371447,0.05660484879943106) -- (532.0000000367028,0.05539312181373446),qqwuqq+linewidth(0.7));
draw( (532.0000000367028,0.05539312181373446) -- (534.0000000362609,0.054202888509448266),qqwuqq+linewidth(0.7));
draw( (534.0000000362609,0.054202888509448266) -- (536.000000035819,0.05303387666432535),qqwuqq+linewidth(0.7));
draw( (536.000000035819,0.05303387666432535) -- (538.0000000353771,0.05188581435353268),qqwuqq+linewidth(0.7));
draw( (538.0000000353771,0.05188581435353268) -- (540.0000000349352,0.05075843003005853),qqwuqq+linewidth(0.7));
draw( (540.0000000349352,0.05075843003005853) -- (542.0000000344933,0.04965145260379156),qqwuqq+linewidth(0.7));
draw( (542.0000000344933,0.04965145260379156) -- (544.0000000340514,0.04856461151926367),qqwuqq+linewidth(0.7));
draw( (544.0000000340514,0.04856461151926367) -- (546.0000000336095,0.047497636831992485),qqwuqq+linewidth(0.7));
draw( (546.0000000336095,0.047497636831992485) -- (548.0000000331676,0.04645025928350932),qqwuqq+linewidth(0.7));
draw( (548.0000000331676,0.04645025928350932) -- (550.0000000327257,0.04542221037493632),qqwuqq+linewidth(0.7));
draw( (550.0000000327257,0.04542221037493632) -- (552.0000000322838,0.04441322243925183),qqwuqq+linewidth(0.7));
draw( (552.0000000322838,0.04441322243925183) -- (554.0000000318419,0.043423028712087974),qqwuqq+linewidth(0.7));
draw( (554.0000000318419,0.043423028712087974) -- (556.0000000314,0.042451363401121545),qqwuqq+linewidth(0.7));
draw( (556.0000000314,0.042451363401121545) -- (558.0000000309581,0.041497961754110746),qqwuqq+linewidth(0.7));
draw( (558.0000000309581,0.041497961754110746) -- (560.0000000305162,0.04056256012541626),qqwuqq+linewidth(0.7));
draw( (560.0000000305162,0.04056256012541626) -- (562.0000000300743,0.03964489604115268),qqwuqq+linewidth(0.7));
draw( (562.0000000300743,0.03964489604115268) -- (564.0000000296324,0.03874470826286397),qqwuqq+linewidth(0.7));
draw( (564.0000000296324,0.03874470826286397) -- (566.0000000291905,0.03786173684975369),qqwuqq+linewidth(0.7));
draw( (566.0000000291905,0.03786173684975369) -- (568.0000000287486,0.036995723219442556),qqwuqq+linewidth(0.7));
draw( (568.0000000287486,0.036995723219442556) -- (570.0000000283067,0.036146410207313684),qqwuqq+linewidth(0.7));
draw( (570.0000000283067,0.036146410207313684) -- (572.0000000278648,0.03531354212435289),qqwuqq+linewidth(0.7));
draw( (572.0000000278648,0.03531354212435289) -- (574.0000000274229,0.03449686481348868),qqwuqq+linewidth(0.7));
draw( (574.0000000274229,0.03449686481348868) -- (576.000000026981,0.03369612570450986),qqwuqq+linewidth(0.7));
draw( (576.000000026981,0.03369612570450986) -- (578.000000026539,0.032911073867453955),qqwuqq+linewidth(0.7));
draw( (578.000000026539,0.032911073867453955) -- (580.0000000260972,0.032141460064563865),qqwuqq+linewidth(0.7));
draw( (580.0000000260972,0.032141460064563865) -- (582.0000000256553,0.03138703680069055),qqwuqq+linewidth(0.7));
draw( (582.0000000256553,0.03138703680069055) -- (584.0000000252134,0.03064755837223687),qqwuqq+linewidth(0.7));
draw( (584.0000000252134,0.03064755837223687) -- (586.0000000247715,0.029922780914599434),qqwuqq+linewidth(0.7));
draw( (586.0000000247715,0.029922780914599434) -- (588.0000000243296,0.02921246244814657),qqwuqq+linewidth(0.7));
draw( (588.0000000243296,0.02921246244814657) -- (590.0000000238877,0.028516362922624827),qqwuqq+linewidth(0.7));
draw( (590.0000000238877,0.028516362922624827) -- (592.0000000234458,0.02783424426010872),qqwuqq+linewidth(0.7));
draw( (592.0000000234458,0.02783424426010872) -- (594.0000000230038,0.027165870396502756),qqwuqq+linewidth(0.7));
draw( (594.0000000230038,0.027165870396502756) -- (596.000000022562,0.02651100732142808),qqwuqq+linewidth(0.7));
draw( (596.000000022562,0.02651100732142808) -- (598.00000002212,0.025869423116701262),qqwuqq+linewidth(0.7));
draw( (598.00000002212,0.025869423116701262) -- (600.0000000216781,0.025240887993261074),qqwuqq+linewidth(0.7));
draw( (600.0000000216781,0.025240887993261074) -- (602.0000000212362,0.024625174326640353),qqwuqq+linewidth(0.7));
draw( (602.0000000212362,0.024625174326640353) -- (604.0000000207943,0.024022056690905513),qqwuqq+linewidth(0.7));
draw( (604.0000000207943,0.024022056690905513) -- (606.0000000203524,0.023431311891103368),qqwuqq+linewidth(0.7));
draw( (606.0000000203524,0.023431311891103368) -- (608.0000000199105,0.022852718994260557),qqwuqq+linewidth(0.7));
draw( (608.0000000199105,0.022852718994260557) -- (610.0000000194686,0.022286059358827326),qqwuqq+linewidth(0.7));
draw( (610.0000000194686,0.022286059358827326) -- (612.0000000190267,0.021731116662649264),qqwuqq+linewidth(0.7));
draw( (612.0000000190267,0.021731116662649264) -- (614.0000000185848,0.021187676929531496),qqwuqq+linewidth(0.7));
draw( (614.0000000185848,0.021187676929531496) -- (616.0000000181429,0.020655528554247904),qqwuqq+linewidth(0.7));
draw( (616.0000000181429,0.020655528554247904) -- (618.000000017701,0.02013446232604854),qqwuqq+linewidth(0.7));
draw( (618.000000017701,0.02013446232604854) -- (620.0000000172591,0.019624271450811914),qqwuqq+linewidth(0.7));
draw( (620.0000000172591,0.019624271450811914) -- (622.0000000168172,0.0191247515716656),qqwuqq+linewidth(0.7));
draw( (622.0000000168172,0.0191247515716656) -- (624.0000000163753,0.018635700788121934),qqwuqq+linewidth(0.7));
draw( (624.0000000163753,0.018635700788121934) -- (626.0000000159334,0.018156919673822047),qqwuqq+linewidth(0.7));
draw( (626.0000000159334,0.018156919673822047) -- (628.0000000154915,0.017688211292822276),qqwuqq+linewidth(0.7));
draw( (628.0000000154915,0.017688211292822276) -- (630.0000000150496,0.017229381214442174),qqwuqq+linewidth(0.7));
draw( (630.0000000150496,0.017229381214442174) -- (632.0000000146077,0.016780237526643926),qqwuqq+linewidth(0.7));
draw( (632.0000000146077,0.016780237526643926) -- (634.0000000141658,0.01634059084811712),qqwuqq+linewidth(0.7));
draw( (634.0000000141658,0.01634059084811712) -- (636.0000000137239,0.015910254338763696),qqwuqq+linewidth(0.7));
draw( (636.0000000137239,0.015910254338763696) -- (638.000000013282,0.015489043708970529),qqwuqq+linewidth(0.7));
draw( (638.000000013282,0.015489043708970529) -- (640.0000000128401,0.01507677722739209),qqwuqq+linewidth(0.7));
draw( (640.0000000128401,0.01507677722739209) -- (642.0000000123982,0.014673275727348),qqwuqq+linewidth(0.7));
draw( (642.0000000123982,0.014673275727348) -- (644.0000000119563,0.014278362611896078),qqwuqq+linewidth(0.7));
draw( (644.0000000119563,0.014278362611896078) -- (646.0000000115144,0.01389186385750718),qqwuqq+linewidth(0.7));
draw( (646.0000000115144,0.01389186385750718) -- (648.0000000110725,0.013513608016460399),qqwuqq+linewidth(0.7));
draw( (648.0000000110725,0.013513608016460399) -- (650.0000000106306,0.013143426217778087),qqwuqq+linewidth(0.7));
draw( (650.0000000106306,0.013143426217778087) -- (652.0000000101887,0.012781152167002041),qqwuqq+linewidth(0.7));
draw( (652.0000000101887,0.012781152167002041) -- (654.0000000097468,0.01242662214449286),qqwuqq+linewidth(0.7));
draw( (654.0000000097468,0.01242662214449286) -- (656.0000000093049,0.01207967500256446),qqwuqq+linewidth(0.7));
draw( (656.0000000093049,0.01207967500256446) -- (658.000000008863,0.011740152161277884),qqwuqq+linewidth(0.7));
draw( (658.000000008863,0.011740152161277884) -- (660.0000000084211,0.011407897602980777),qqwuqq+linewidth(0.7));
draw( (660.0000000084211,0.011407897602980777) -- (662.0000000079792,0.01108275786556323),qqwuqq+linewidth(0.7));
draw( (662.0000000079792,0.01108275786556323) -- (664.0000000075373,0.010764582034535897),qqwuqq+linewidth(0.7));
draw( (664.0000000075373,0.010764582034535897) -- (666.0000000070954,0.010453221733810247),qqwuqq+linewidth(0.7));
draw( (666.0000000070954,0.010453221733810247) -- (668.0000000066535,0.010148531115386827),qqwuqq+linewidth(0.7));
draw( (668.0000000066535,0.010148531115386827) -- (670.0000000062116,0.009850366847712788),qqwuqq+linewidth(0.7));
draw( (670.0000000062116,0.009850366847712788) -- (672.0000000057697,0.009558588102940524),qqwuqq+linewidth(0.7));
draw( (672.0000000057697,0.009558588102940524) -- (674.0000000053278,0.009273056543076974),qqwuqq+linewidth(0.7));
draw( (674.0000000053278,0.009273056543076974) -- (676.0000000048859,0.008993636304890362),qqwuqq+linewidth(0.7));
draw( (676.0000000048859,0.008993636304890362) -- (678.000000004444,0.008720193983785984),qqwuqq+linewidth(0.7));
draw( (678.000000004444,0.008720193983785984) -- (680.0000000040021,0.00845259861655201),qqwuqq+linewidth(0.7));
draw( (680.0000000040021,0.00845259861655201) -- (682.0000000035602,0.008190721662966194),qqwuqq+linewidth(0.7));
draw( (682.0000000035602,0.008190721662966194) -- (684.0000000031183,0.007934436986499094),qqwuqq+linewidth(0.7));
draw( (684.0000000031183,0.007934436986499094) -- (686.0000000026764,0.0076836208338164624),qqwuqq+linewidth(0.7));
draw( (686.0000000026764,0.0076836208338164624) -- (688.0000000022345,0.007438151813234928),qqwuqq+linewidth(0.7));
draw( (688.0000000022345,0.007438151813234928) -- (690.0000000017926,0.007197910872453139),qqwuqq+linewidth(0.7));
draw( (690.0000000017926,0.007197910872453139) -- (692.0000000013507,0.006962781274926666),qqwuqq+linewidth(0.7));
draw( (692.0000000013507,0.006962781274926666) -- (694.0000000009088,0.0067326485756260546),qqwuqq+linewidth(0.7));
draw( (694.0000000009088,0.0067326485756260546) -- (696.0000000004669,0.006507400595634039),qqwuqq+linewidth(0.7));
draw( (696.0000000004669,0.006507400595634039) -- (698.000000000025,0.006286927395851016),qqwuqq+linewidth(0.7));
draw( (698.000000000025,0.006286927395851016) -- (699.9999999995831,0.0060711212500530465),qqwuqq+linewidth(0.7));
draw( (699.9999999995831,0.0060711212500530465) -- (701.9999999991412,0.005859876616716608),qqwuqq+linewidth(0.7));
draw( (701.9999999991412,0.005859876616716608) -- (703.9999999986993,0.005653090110320669),qqwuqq+linewidth(0.7));
draw( (703.9999999986993,0.005653090110320669) -- (705.9999999982574,0.005450660471629121),qqwuqq+linewidth(0.7));
draw( (705.9999999982574,0.005450660471629121) -- (707.9999999978155,0.005252488537321298),qqwuqq+linewidth(0.7));
draw( (707.9999999978155,0.005252488537321298) -- (709.9999999973736,0.005058477208775614),qqwuqq+linewidth(0.7));
draw( (709.9999999973736,0.005058477208775614) -- (711.9999999969317,0.00486853142011201),qqwuqq+linewidth(0.7));
draw( (711.9999999969317,0.00486853142011201) -- (713.9999999964898,0.004682558105624235),qqwuqq+linewidth(0.7));
draw( (713.9999999964898,0.004682558105624235) -- (715.9999999960479,0.004500466166403427),qqwuqq+linewidth(0.7));
draw( (715.9999999960479,0.004500466166403427) -- (717.999999995606,0.004322166436335539),qqwuqq+linewidth(0.7));
draw( (717.999999995606,0.004322166436335539) -- (719.9999999951641,0.004147571647588055),qqwuqq+linewidth(0.7));
draw( (719.9999999951641,0.004147571647588055) -- (721.9999999947222,0.003976596395273813),qqwuqq+linewidth(0.7));
draw( (721.9999999947222,0.003976596395273813) -- (723.9999999942803,0.0038091571017924153),qqwuqq+linewidth(0.7));
draw( (723.9999999942803,0.0038091571017924153) -- (725.9999999938384,0.003645171980407369),qqwuqq+linewidth(0.7));
draw( (725.9999999938384,0.003645171980407369) -- (727.9999999933965,0.0034845609984923698),qqwuqq+linewidth(0.7));
draw( (727.9999999933965,0.0034845609984923698) -- (729.9999999929546,0.003327245840170967),qqwuqq+linewidth(0.7));
draw( (729.9999999929546,0.003327245840170967) -- (731.9999999925127,0.0031731498685028114),qqwuqq+linewidth(0.7));
draw( (731.9999999925127,0.0031731498685028114) -- (733.9999999920708,0.0030221980874092225),qqwuqq+linewidth(0.7));
draw( (733.9999999920708,0.0030221980874092225) -- (735.9999999916289,0.0028743171029939063),qqwuqq+linewidth(0.7));
draw( (735.9999999916289,0.0028743171029939063) -- (737.999999991187,0.002729435084621201),qqwuqq+linewidth(0.7));
draw( (737.999999991187,0.002729435084621201) -- (739.9999999907451,0.0025874817257034444),qqwuqq+linewidth(0.7));
draw( (739.9999999907451,0.0025874817257034444) -- (741.9999999903032,0.002448388204055796),qqwuqq+linewidth(0.7));
draw( (741.9999999903032,0.002448388204055796) -- (743.9999999898613,0.0023120871421404843),qqwuqq+linewidth(0.7));
draw( (743.9999999898613,0.0023120871421404843) -- (745.9999999894194,0.00217851256710766),qqwuqq+linewidth(0.7));
draw( (745.9999999894194,0.00217851256710766) -- (747.9999999889775,0.0020475998703526344),qqwuqq+linewidth(0.7));
draw( (747.9999999889775,0.0020475998703526344) -- (749.9999999885356,0.0019192857673226982),qqwuqq+linewidth(0.7));
draw( (749.9999999885356,0.0019192857673226982) -- (751.9999999880937,0.0017935082569575655),qqwuqq+linewidth(0.7));
draw( (751.9999999880937,0.0017935082569575655) -- (753.9999999876518,0.0016702065809823807),qqwuqq+linewidth(0.7));
draw( (753.9999999876518,0.0016702065809823807) -- (755.9999999872099,0.0015493211834582965),qqwuqq+linewidth(0.7));
draw( (755.9999999872099,0.0015493211834582965) -- (757.999999986768,0.0014307936699111679),qqwuqq+linewidth(0.7));
draw( (757.999999986768,0.0014307936699111679) -- (759.9999999863261,0.0013145667669611782),qqwuqq+linewidth(0.7));
draw( (759.9999999863261,0.0013145667669611782) -- (761.9999999858842,0.0012005842814213352),qqwuqq+linewidth(0.7));
draw( (761.9999999858842,0.0012005842814213352) -- (763.9999999854423,0.001088791060078087),qqwuqq+linewidth(0.7));
draw( (763.9999999854423,0.001088791060078087) -- (765.9999999850004,0),qqwuqq+linewidth(0.7));
draw( (765.9999999850004,0) -- (767.9999999845585,0),qqwuqq+linewidth(0.7));
draw( (767.9999999845585,0) -- (769.9999999841166,0),qqwuqq+linewidth(0.7));
draw( (769.9999999841166,0) -- (771.9999999836747,0),qqwuqq+linewidth(0.7));
draw( (771.9999999836747,0) -- (773.9999999832328,0),qqwuqq+linewidth(0.7));
draw( (773.9999999832328,0) -- (775.9999999827909,0),qqwuqq+linewidth(0.7));
draw( (775.9999999827909,0) -- (777.999999982349,0),qqwuqq+linewidth(0.7));
draw( (777.999999982349,0) -- (779.9999999819071,0),qqwuqq+linewidth(0.7));
draw( (779.9999999819071,0) -- (781.9999999814652,0),qqwuqq+linewidth(0.7));
draw( (781.9999999814652,0) -- (783.9999999810233,0),qqwuqq+linewidth(0.7));
draw( (783.9999999810233,0) -- (785.9999999805814,0),qqwuqq+linewidth(0.7));
draw( (785.9999999805814,0) -- (787.9999999801395,0),qqwuqq+linewidth(0.7));
draw( (787.9999999801395,0) -- (789.9999999796976,0),qqwuqq+linewidth(0.7));
draw( (789.9999999796976,0) -- (791.9999999792557,0),qqwuqq+linewidth(0.7));
draw( (791.9999999792557,0) -- (793.9999999788138,0),qqwuqq+linewidth(0.7));
draw( (793.9999999788138,0) -- (795.9999999783719,0),qqwuqq+linewidth(0.7));
draw( (795.9999999783719,0) -- (797.99999997793,0),qqwuqq+linewidth(0.7));
;
draw( (800.0000000573743,0) -- (800.0000000573743,0),qqwuqq+linewidth(0.7));
draw( (800.0000000573743,0) -- (800.5000000435914,0),qqwuqq+linewidth(0.7));
draw( (800.5000000435914,0) -- (801.0000000298085,0),qqwuqq+linewidth(0.7));
draw( (801.0000000298085,0) -- (801.5000000160255,0),qqwuqq+linewidth(0.7));
draw( (801.5000000160255,0) -- (802.0000000022426,0),qqwuqq+linewidth(0.7));
draw( (802.0000000022426,0) -- (802.4999999884596,0),qqwuqq+linewidth(0.7));
draw( (802.4999999884596,0) -- (802.9999999746767,0),qqwuqq+linewidth(0.7));
draw( (802.9999999746767,0) -- (803.4999999608938,0),qqwuqq+linewidth(0.7));
draw( (803.4999999608938,0) -- (803.9999999471108,0),qqwuqq+linewidth(0.7));
draw( (803.9999999471108,0) -- (804.4999999333279,0),qqwuqq+linewidth(0.7));
draw( (804.4999999333279,0) -- (804.999999919545,0),qqwuqq+linewidth(0.7));
draw( (804.999999919545,0) -- (805.499999905762,0),qqwuqq+linewidth(0.7));
draw( (805.499999905762,0) -- (805.9999998919791,0),qqwuqq+linewidth(0.7));
draw( (805.9999998919791,0) -- (806.4999998781961,0),qqwuqq+linewidth(0.7));
draw( (806.4999998781961,0) -- (806.9999998644132,0),qqwuqq+linewidth(0.7));
draw( (806.9999998644132,0) -- (807.4999998506303,0),qqwuqq+linewidth(0.7));
draw( (807.4999998506303,0) -- (807.9999998368473,0),qqwuqq+linewidth(0.7));
draw( (807.9999998368473,0) -- (808.4999998230644,0),qqwuqq+linewidth(0.7));
draw( (808.4999998230644,0) -- (808.9999998092815,0),qqwuqq+linewidth(0.7));
draw( (808.9999998092815,0) -- (809.4999997954985,0),qqwuqq+linewidth(0.7));
draw( (809.4999997954985,0) -- (809.9999997817156,0),qqwuqq+linewidth(0.7));
draw( (809.9999997817156,0) -- (810.4999997679326,0),qqwuqq+linewidth(0.7));
draw( (810.4999997679326,0) -- (810.9999997541497,0),qqwuqq+linewidth(0.7));
draw( (810.9999997541497,0) -- (811.4999997403668,0),qqwuqq+linewidth(0.7));
draw( (811.4999997403668,0) -- (811.9999997265838,0),qqwuqq+linewidth(0.7));
draw( (811.9999997265838,0) -- (812.4999997128009,0),qqwuqq+linewidth(0.7));
draw( (812.4999997128009,0) -- (812.999999699018,0),qqwuqq+linewidth(0.7));
draw( (812.999999699018,0) -- (813.499999685235,0),qqwuqq+linewidth(0.7));
draw( (813.499999685235,0) -- (813.9999996714521,0),qqwuqq+linewidth(0.7));
draw( (813.9999996714521,0) -- (814.4999996576692,0),qqwuqq+linewidth(0.7));
draw( (814.4999996576692,0) -- (814.9999996438862,0),qqwuqq+linewidth(0.7));
draw( (814.9999996438862,0) -- (815.4999996301033,0),qqwuqq+linewidth(0.7));
draw( (815.4999996301033,0) -- (815.9999996163203,0),qqwuqq+linewidth(0.7));
draw( (815.9999996163203,0) -- (816.4999996025374,0),qqwuqq+linewidth(0.7));
draw( (816.4999996025374,0) -- (816.9999995887545,0),qqwuqq+linewidth(0.7));
draw( (816.9999995887545,0) -- (817.4999995749715,0),qqwuqq+linewidth(0.7));
draw( (817.4999995749715,0) -- (817.9999995611886,0),qqwuqq+linewidth(0.7));
draw( (817.9999995611886,0) -- (818.4999995474057,0),qqwuqq+linewidth(0.7));
draw( (818.4999995474057,0) -- (818.9999995336227,0),qqwuqq+linewidth(0.7));
draw( (818.9999995336227,0) -- (819.4999995198398,0),qqwuqq+linewidth(0.7));
draw( (819.4999995198398,0) -- (819.9999995060568,0),qqwuqq+linewidth(0.7));
draw( (819.9999995060568,0) -- (820.4999994922739,0),qqwuqq+linewidth(0.7));
draw( (820.4999994922739,0) -- (820.999999478491,0),qqwuqq+linewidth(0.7));
draw( (820.999999478491,0) -- (821.499999464708,0),qqwuqq+linewidth(0.7));
draw( (821.499999464708,0) -- (821.9999994509251,0),qqwuqq+linewidth(0.7));
draw( (821.9999994509251,0) -- (822.4999994371422,0),qqwuqq+linewidth(0.7));
draw( (822.4999994371422,0) -- (822.9999994233592,0),qqwuqq+linewidth(0.7));
draw( (822.9999994233592,0) -- (823.4999994095763,0),qqwuqq+linewidth(0.7));
draw( (823.4999994095763,0) -- (823.9999993957933,0),qqwuqq+linewidth(0.7));
draw( (823.9999993957933,0) -- (824.4999993820104,0),qqwuqq+linewidth(0.7));
draw( (824.4999993820104,0) -- (824.9999993682275,0),qqwuqq+linewidth(0.7));
draw( (824.9999993682275,0) -- (825.4999993544445,0),qqwuqq+linewidth(0.7));
draw( (825.4999993544445,0) -- (825.9999993406616,0),qqwuqq+linewidth(0.7));
draw( (825.9999993406616,0) -- (826.4999993268787,0),qqwuqq+linewidth(0.7));
draw( (826.4999993268787,0) -- (826.9999993130957,0),qqwuqq+linewidth(0.7));
draw( (826.9999993130957,0) -- (827.4999992993128,0),qqwuqq+linewidth(0.7));
draw( (827.4999992993128,0) -- (827.9999992855298,0),qqwuqq+linewidth(0.7));
draw( (827.9999992855298,0) -- (828.4999992717469,0),qqwuqq+linewidth(0.7));
draw( (828.4999992717469,0) -- (828.999999257964,0),qqwuqq+linewidth(0.7));
draw( (828.999999257964,0) -- (829.499999244181,0),qqwuqq+linewidth(0.7));
draw( (829.499999244181,0) -- (829.9999992303981,0),qqwuqq+linewidth(0.7));
draw( (829.9999992303981,0) -- (830.4999992166152,0),qqwuqq+linewidth(0.7));
draw( (830.4999992166152,0) -- (830.9999992028322,0),qqwuqq+linewidth(0.7));
draw( (830.9999992028322,0) -- (831.4999991890493,0),qqwuqq+linewidth(0.7));
draw( (831.4999991890493,0) -- (831.9999991752663,0),qqwuqq+linewidth(0.7));
draw( (831.9999991752663,0) -- (832.4999991614834,0),qqwuqq+linewidth(0.7));
draw( (832.4999991614834,0) -- (832.9999991477005,0),qqwuqq+linewidth(0.7));
draw( (832.9999991477005,0) -- (833.4999991339175,0),qqwuqq+linewidth(0.7));
draw( (833.4999991339175,0) -- (833.9999991201346,0),qqwuqq+linewidth(0.7));
draw( (833.9999991201346,0) -- (834.4999991063517,0),qqwuqq+linewidth(0.7));
draw( (834.4999991063517,0) -- (834.9999990925687,0),qqwuqq+linewidth(0.7));
draw( (834.9999990925687,0) -- (835.4999990787858,0),qqwuqq+linewidth(0.7));
draw( (835.4999990787858,0) -- (835.9999990650028,0),qqwuqq+linewidth(0.7));
draw( (835.9999990650028,0) -- (836.4999990512199,0),qqwuqq+linewidth(0.7));
draw( (836.4999990512199,0) -- (836.999999037437,0),qqwuqq+linewidth(0.7));
draw( (836.999999037437,0) -- (837.499999023654,0),qqwuqq+linewidth(0.7));
draw( (837.499999023654,0) -- (837.9999990098711,0),qqwuqq+linewidth(0.7));
draw( (837.9999990098711,0) -- (838.4999989960882,0),qqwuqq+linewidth(0.7));
draw( (838.4999989960882,0) -- (838.9999989823052,0),qqwuqq+linewidth(0.7));
draw( (838.9999989823052,0) -- (839.4999989685223,0),qqwuqq+linewidth(0.7));
draw( (839.4999989685223,0) -- (839.9999989547393,0),qqwuqq+linewidth(0.7));
draw( (839.9999989547393,0) -- (840.4999989409564,0),qqwuqq+linewidth(0.7));
draw( (840.4999989409564,0) -- (840.9999989271735,0),qqwuqq+linewidth(0.7));
draw( (840.9999989271735,0) -- (841.4999989133905,0),qqwuqq+linewidth(0.7));
draw( (841.4999989133905,0) -- (841.9999988996076,0),qqwuqq+linewidth(0.7));
draw( (841.9999988996076,0) -- (842.4999988858247,0),qqwuqq+linewidth(0.7));
draw( (842.4999988858247,0) -- (842.9999988720417,0),qqwuqq+linewidth(0.7));
draw( (842.9999988720417,0) -- (843.4999988582588,0),qqwuqq+linewidth(0.7));
draw( (843.4999988582588,0) -- (843.9999988444758,0),qqwuqq+linewidth(0.7));
draw( (843.9999988444758,0) -- (844.4999988306929,0),qqwuqq+linewidth(0.7));
draw( (844.4999988306929,0) -- (844.99999881691,0),qqwuqq+linewidth(0.7));
draw( (844.99999881691,0) -- (845.499998803127,0),qqwuqq+linewidth(0.7));
draw( (845.499998803127,0) -- (845.9999987893441,0),qqwuqq+linewidth(0.7));
draw( (845.9999987893441,0) -- (846.4999987755612,0),qqwuqq+linewidth(0.7));
draw( (846.4999987755612,0) -- (846.9999987617782,0),qqwuqq+linewidth(0.7));
draw( (846.9999987617782,0) -- (847.4999987479953,0),qqwuqq+linewidth(0.7));
draw( (847.4999987479953,0) -- (847.9999987342123,0),qqwuqq+linewidth(0.7));
draw( (847.9999987342123,0) -- (848.4999987204294,0),qqwuqq+linewidth(0.7));
draw( (848.4999987204294,0) -- (848.9999987066465,0),qqwuqq+linewidth(0.7));
draw( (848.9999987066465,0) -- (849.4999986928635,0),qqwuqq+linewidth(0.7));
draw( (849.4999986928635,0) -- (849.9999986790806,0),qqwuqq+linewidth(0.7));
draw( (849.9999986790806,0) -- (850.4999986652977,0),qqwuqq+linewidth(0.7));
draw( (850.4999986652977,0) -- (850.9999986515147,0),qqwuqq+linewidth(0.7));
draw( (850.9999986515147,0) -- (851.4999986377318,0),qqwuqq+linewidth(0.7));
draw( (851.4999986377318,0) -- (851.9999986239488,0),qqwuqq+linewidth(0.7));
draw( (851.9999986239488,0) -- (852.4999986101659,0),qqwuqq+linewidth(0.7));
draw( (852.4999986101659,0) -- (852.999998596383,0),qqwuqq+linewidth(0.7));
draw( (852.999998596383,0) -- (853.4999985826,0),qqwuqq+linewidth(0.7));
draw( (853.4999985826,0) -- (853.9999985688171,0),qqwuqq+linewidth(0.7));
draw( (853.9999985688171,0) -- (854.4999985550342,0),qqwuqq+linewidth(0.7));
draw( (854.4999985550342,0) -- (854.9999985412512,0),qqwuqq+linewidth(0.7));
draw( (854.9999985412512,0) -- (855.4999985274683,0),qqwuqq+linewidth(0.7));
draw( (855.4999985274683,0) -- (855.9999985136853,0),qqwuqq+linewidth(0.7));
draw( (855.9999985136853,0) -- (856.4999984999024,0),qqwuqq+linewidth(0.7));
draw( (856.4999984999024,0) -- (856.9999984861195,0),qqwuqq+linewidth(0.7));
draw( (856.9999984861195,0) -- (857.4999984723365,0),qqwuqq+linewidth(0.7));
draw( (857.4999984723365,0) -- (857.9999984585536,0),qqwuqq+linewidth(0.7));
draw( (857.9999984585536,0) -- (858.4999984447707,0),qqwuqq+linewidth(0.7));
draw( (858.4999984447707,0) -- (858.9999984309877,0),qqwuqq+linewidth(0.7));
draw( (858.9999984309877,0) -- (859.4999984172048,0),qqwuqq+linewidth(0.7));
draw( (859.4999984172048,0) -- (859.9999984034218,0),qqwuqq+linewidth(0.7));
draw( (859.9999984034218,0) -- (860.4999983896389,0),qqwuqq+linewidth(0.7));
draw( (860.4999983896389,0) -- (860.999998375856,0),qqwuqq+linewidth(0.7));
draw( (860.999998375856,0) -- (861.499998362073,0),qqwuqq+linewidth(0.7));
draw( (861.499998362073,0) -- (861.9999983482901,0),qqwuqq+linewidth(0.7));
draw( (861.9999983482901,0) -- (862.4999983345072,0),qqwuqq+linewidth(0.7));
draw( (862.4999983345072,0) -- (862.9999983207242,0),qqwuqq+linewidth(0.7));
draw( (862.9999983207242,0) -- (863.4999983069413,0),qqwuqq+linewidth(0.7));
draw( (863.4999983069413,0) -- (863.9999982931583,0),qqwuqq+linewidth(0.7));
draw( (863.9999982931583,0) -- (864.4999982793754,0),qqwuqq+linewidth(0.7));
draw( (864.4999982793754,0) -- (864.9999982655925,0),qqwuqq+linewidth(0.7));
draw( (864.9999982655925,0) -- (865.4999982518095,0),qqwuqq+linewidth(0.7));
draw( (865.4999982518095,0) -- (865.9999982380266,0),qqwuqq+linewidth(0.7));
draw( (865.9999982380266,0) -- (866.4999982242437,0),qqwuqq+linewidth(0.7));
draw( (866.4999982242437,0) -- (866.9999982104607,0),qqwuqq+linewidth(0.7));
draw( (866.9999982104607,0) -- (867.4999981966778,0),qqwuqq+linewidth(0.7));
draw( (867.4999981966778,0) -- (867.9999981828948,0),qqwuqq+linewidth(0.7));
draw( (867.9999981828948,0) -- (868.4999981691119,0),qqwuqq+linewidth(0.7));
draw( (868.4999981691119,0) -- (868.999998155329,0),qqwuqq+linewidth(0.7));
draw( (868.999998155329,0) -- (869.499998141546,0),qqwuqq+linewidth(0.7));
draw( (869.499998141546,0) -- (869.9999981277631,0),qqwuqq+linewidth(0.7));
draw( (869.9999981277631,0) -- (870.4999981139802,0),qqwuqq+linewidth(0.7));
draw( (870.4999981139802,0) -- (870.9999981001972,0),qqwuqq+linewidth(0.7));
draw( (870.9999981001972,0) -- (871.4999980864143,0),qqwuqq+linewidth(0.7));
draw( (871.4999980864143,0) -- (871.9999980726313,0),qqwuqq+linewidth(0.7));
draw( (871.9999980726313,0) -- (872.4999980588484,0),qqwuqq+linewidth(0.7));
draw( (872.4999980588484,0) -- (872.9999980450655,0),qqwuqq+linewidth(0.7));
draw( (872.9999980450655,0) -- (873.4999980312825,0),qqwuqq+linewidth(0.7));
draw( (873.4999980312825,0) -- (873.9999980174996,0),qqwuqq+linewidth(0.7));
draw( (873.9999980174996,0) -- (874.4999980037167,0),qqwuqq+linewidth(0.7));
draw( (874.4999980037167,0) -- (874.9999979899337,0),qqwuqq+linewidth(0.7));
draw( (874.9999979899337,0) -- (875.4999979761508,0),qqwuqq+linewidth(0.7));
draw( (875.4999979761508,0) -- (875.9999979623678,0),qqwuqq+linewidth(0.7));
draw( (875.9999979623678,0) -- (876.4999979485849,0),qqwuqq+linewidth(0.7));
draw( (876.4999979485849,0) -- (876.999997934802,0),qqwuqq+linewidth(0.7));
draw( (876.999997934802,0) -- (877.499997921019,0),qqwuqq+linewidth(0.7));
draw( (877.499997921019,0) -- (877.9999979072361,0),qqwuqq+linewidth(0.7));
draw( (877.9999979072361,0) -- (878.4999978934532,0),qqwuqq+linewidth(0.7));
draw( (878.4999978934532,0) -- (878.9999978796702,0),qqwuqq+linewidth(0.7));
draw( (878.9999978796702,0) -- (879.4999978658873,0),qqwuqq+linewidth(0.7));
draw( (879.4999978658873,0) -- (879.9999978521043,0),qqwuqq+linewidth(0.7));
draw( (879.9999978521043,0) -- (880.4999978383214,0),qqwuqq+linewidth(0.7));
draw( (880.4999978383214,0) -- (880.9999978245385,0),qqwuqq+linewidth(0.7));
draw( (880.9999978245385,0) -- (881.4999978107555,0),qqwuqq+linewidth(0.7));
draw( (881.4999978107555,0) -- (881.9999977969726,0),qqwuqq+linewidth(0.7));
draw( (881.9999977969726,0) -- (882.4999977831897,0),qqwuqq+linewidth(0.7));
draw( (882.4999977831897,0) -- (882.9999977694067,0),qqwuqq+linewidth(0.7));
draw( (882.9999977694067,0) -- (883.4999977556238,0),qqwuqq+linewidth(0.7));
draw( (883.4999977556238,0) -- (883.9999977418408,0),qqwuqq+linewidth(0.7));
draw( (883.9999977418408,0) -- (884.4999977280579,0),qqwuqq+linewidth(0.7));
draw( (884.4999977280579,0) -- (884.999997714275,0),qqwuqq+linewidth(0.7));
draw( (884.999997714275,0) -- (885.499997700492,0),qqwuqq+linewidth(0.7));
draw( (885.499997700492,0) -- (885.9999976867091,0),qqwuqq+linewidth(0.7));
draw( (885.9999976867091,0) -- (886.4999976729262,0),qqwuqq+linewidth(0.7));
draw( (886.4999976729262,0) -- (886.9999976591432,0),qqwuqq+linewidth(0.7));
draw( (886.9999976591432,0) -- (887.4999976453603,0),qqwuqq+linewidth(0.7));
draw( (887.4999976453603,0) -- (887.9999976315773,0),qqwuqq+linewidth(0.7));
draw( (887.9999976315773,0) -- (888.4999976177944,0),qqwuqq+linewidth(0.7));
draw( (888.4999976177944,0) -- (888.9999976040115,0),qqwuqq+linewidth(0.7));
draw( (888.9999976040115,0) -- (889.4999975902285,0),qqwuqq+linewidth(0.7));
draw( (889.4999975902285,0) -- (889.9999975764456,0),qqwuqq+linewidth(0.7));
draw( (889.9999975764456,0) -- (890.4999975626627,0),qqwuqq+linewidth(0.7));
draw( (890.4999975626627,0) -- (890.9999975488797,0),qqwuqq+linewidth(0.7));
draw( (890.9999975488797,0) -- (891.4999975350968,0),qqwuqq+linewidth(0.7));
draw( (891.4999975350968,0) -- (891.9999975213138,0),qqwuqq+linewidth(0.7));
draw( (891.9999975213138,0) -- (892.4999975075309,0),qqwuqq+linewidth(0.7));
draw( (892.4999975075309,0) -- (892.999997493748,0),qqwuqq+linewidth(0.7));
draw( (892.999997493748,0) -- (893.499997479965,0),qqwuqq+linewidth(0.7));
draw( (893.499997479965,0) -- (893.9999974661821,0),qqwuqq+linewidth(0.7));
draw( (893.9999974661821,0) -- (894.4999974523992,0),qqwuqq+linewidth(0.7));
draw( (894.4999974523992,0) -- (894.9999974386162,0),qqwuqq+linewidth(0.7));
draw( (894.9999974386162,0) -- (895.4999974248333,0),qqwuqq+linewidth(0.7));
draw( (895.4999974248333,0) -- (895.9999974110503,0),qqwuqq+linewidth(0.7));
draw( (895.9999974110503,0) -- (896.4999973972674,0),qqwuqq+linewidth(0.7));
draw( (896.4999973972674,0) -- (896.9999973834845,0),qqwuqq+linewidth(0.7));
draw( (896.9999973834845,0) -- (897.4999973697015,0),qqwuqq+linewidth(0.7));
draw( (897.4999973697015,0) -- (897.9999973559186,0),qqwuqq+linewidth(0.7));
draw( (897.9999973559186,0) -- (898.4999973421357,0),qqwuqq+linewidth(0.7));
draw( (898.4999973421357,0) -- (898.9999973283527,0),qqwuqq+linewidth(0.7));
draw( (898.9999973283527,0) -- (899.4999973145698,0),qqwuqq+linewidth(0.7));
draw( (899.4999973145698,0) -- (899.9999973007868,0),qqwuqq+linewidth(0.7));
draw( (899.9999973007868,0) -- (900.4999972870039,0),qqwuqq+linewidth(0.7));
draw( (900.4999972870039,0) -- (900.999997273221,0),qqwuqq+linewidth(0.7));
draw( (900.999997273221,0) -- (901.499997259438,0),qqwuqq+linewidth(0.7));
draw( (901.499997259438,0) -- (901.9999972456551,0),qqwuqq+linewidth(0.7));
draw( (901.9999972456551,0) -- (902.4999972318722,0),qqwuqq+linewidth(0.7));
draw( (902.4999972318722,0) -- (902.9999972180892,0),qqwuqq+linewidth(0.7));
draw( (902.9999972180892,0) -- (903.4999972043063,0),qqwuqq+linewidth(0.7));
draw( (903.4999972043063,0) -- (903.9999971905233,0),qqwuqq+linewidth(0.7));
draw( (903.9999971905233,0) -- (904.4999971767404,0),qqwuqq+linewidth(0.7));
draw( (904.4999971767404,0) -- (904.9999971629575,0),qqwuqq+linewidth(0.7));
draw( (904.9999971629575,0) -- (905.4999971491745,0),qqwuqq+linewidth(0.7));
draw( (905.4999971491745,0) -- (905.9999971353916,0),qqwuqq+linewidth(0.7));
draw( (905.9999971353916,0) -- (906.4999971216087,0),qqwuqq+linewidth(0.7));
draw( (906.4999971216087,0) -- (906.9999971078257,0),qqwuqq+linewidth(0.7));
draw( (906.9999971078257,0) -- (907.4999970940428,0),qqwuqq+linewidth(0.7));
draw( (907.4999970940428,0) -- (907.9999970802598,0),qqwuqq+linewidth(0.7));
draw( (907.9999970802598,0) -- (908.4999970664769,0),qqwuqq+linewidth(0.7));
draw( (908.4999970664769,0) -- (908.999997052694,0),qqwuqq+linewidth(0.7));
draw( (908.999997052694,0) -- (909.499997038911,0),qqwuqq+linewidth(0.7));
draw( (909.499997038911,0) -- (909.9999970251281,0),qqwuqq+linewidth(0.7));
draw( (909.9999970251281,0) -- (910.4999970113452,0),qqwuqq+linewidth(0.7));
draw( (910.4999970113452,0) -- (910.9999969975622,0),qqwuqq+linewidth(0.7));
draw( (910.9999969975622,0) -- (911.4999969837793,0),qqwuqq+linewidth(0.7));
draw( (911.4999969837793,0) -- (911.9999969699963,0),qqwuqq+linewidth(0.7));
draw( (911.9999969699963,0) -- (912.4999969562134,0),qqwuqq+linewidth(0.7));
draw( (912.4999969562134,0) -- (912.9999969424305,0),qqwuqq+linewidth(0.7));
draw( (912.9999969424305,0) -- (913.4999969286475,0),qqwuqq+linewidth(0.7));
draw( (913.4999969286475,0) -- (913.9999969148646,0),qqwuqq+linewidth(0.7));
draw( (913.9999969148646,0) -- (914.4999969010817,0),qqwuqq+linewidth(0.7));
draw( (914.4999969010817,0) -- (914.9999968872987,0),qqwuqq+linewidth(0.7));
draw( (914.9999968872987,0) -- (915.4999968735158,0),qqwuqq+linewidth(0.7));
draw( (915.4999968735158,0) -- (915.9999968597328,0),qqwuqq+linewidth(0.7));
draw( (915.9999968597328,0) -- (916.4999968459499,0),qqwuqq+linewidth(0.7));
draw( (916.4999968459499,0) -- (916.999996832167,0),qqwuqq+linewidth(0.7));
draw( (916.999996832167,0) -- (917.499996818384,0),qqwuqq+linewidth(0.7));
draw( (917.499996818384,0) -- (917.9999968046011,0),qqwuqq+linewidth(0.7));
draw( (917.9999968046011,0) -- (918.4999967908182,0),qqwuqq+linewidth(0.7));
draw( (918.4999967908182,0) -- (918.9999967770352,0),qqwuqq+linewidth(0.7));
draw( (918.9999967770352,0) -- (919.4999967632523,0),qqwuqq+linewidth(0.7));
draw( (919.4999967632523,0) -- (919.9999967494693,0),qqwuqq+linewidth(0.7));
draw( (919.9999967494693,0) -- (920.4999967356864,0),qqwuqq+linewidth(0.7));
draw( (920.4999967356864,0) -- (920.9999967219035,0),qqwuqq+linewidth(0.7));
draw( (920.9999967219035,0) -- (921.4999967081205,0),qqwuqq+linewidth(0.7));
draw( (921.4999967081205,0) -- (921.9999966943376,0),qqwuqq+linewidth(0.7));
draw( (921.9999966943376,0) -- (922.4999966805547,0),qqwuqq+linewidth(0.7));
draw( (922.4999966805547,0) -- (922.9999966667717,0),qqwuqq+linewidth(0.7));
draw( (922.9999966667717,0) -- (923.4999966529888,0),qqwuqq+linewidth(0.7));
draw( (923.4999966529888,0) -- (923.9999966392058,0),qqwuqq+linewidth(0.7));
draw( (923.9999966392058,0) -- (924.4999966254229,0),qqwuqq+linewidth(0.7));
draw( (924.4999966254229,0) -- (924.99999661164,0),qqwuqq+linewidth(0.7));
draw( (924.99999661164,0) -- (925.499996597857,0),qqwuqq+linewidth(0.7));
draw( (925.499996597857,0) -- (925.9999965840741,0),qqwuqq+linewidth(0.7));
draw( (925.9999965840741,0) -- (926.4999965702912,0),qqwuqq+linewidth(0.7));
draw( (926.4999965702912,0) -- (926.9999965565082,0),qqwuqq+linewidth(0.7));
draw( (926.9999965565082,0) -- (927.4999965427253,0),qqwuqq+linewidth(0.7));
draw( (927.4999965427253,0) -- (927.9999965289423,0),qqwuqq+linewidth(0.7));
draw( (927.9999965289423,0) -- (928.4999965151594,0),qqwuqq+linewidth(0.7));
draw( (928.4999965151594,0) -- (928.9999965013765,0),qqwuqq+linewidth(0.7));
draw( (928.9999965013765,0) -- (929.4999964875935,0),qqwuqq+linewidth(0.7));
draw( (929.4999964875935,0) -- (929.9999964738106,0),qqwuqq+linewidth(0.7));
draw( (929.9999964738106,0) -- (930.4999964600277,0),qqwuqq+linewidth(0.7));
draw( (930.4999964600277,0) -- (930.9999964462447,0),qqwuqq+linewidth(0.7));
draw( (930.9999964462447,0) -- (931.4999964324618,0),qqwuqq+linewidth(0.7));
draw( (931.4999964324618,0) -- (931.9999964186788,0),qqwuqq+linewidth(0.7));
draw( (931.9999964186788,0) -- (932.4999964048959,0),qqwuqq+linewidth(0.7));
draw( (932.4999964048959,0) -- (932.999996391113,0),qqwuqq+linewidth(0.7));
draw( (932.999996391113,0) -- (933.49999637733,0),qqwuqq+linewidth(0.7));
draw( (933.49999637733,0) -- (933.9999963635471,0),qqwuqq+linewidth(0.7));
draw( (933.9999963635471,0) -- (934.4999963497642,0),qqwuqq+linewidth(0.7));
draw( (934.4999963497642,0) -- (934.9999963359812,0),qqwuqq+linewidth(0.7));
draw( (934.9999963359812,0) -- (935.4999963221983,0),qqwuqq+linewidth(0.7));
draw( (935.4999963221983,0) -- (935.9999963084153,0),qqwuqq+linewidth(0.7));
draw( (935.9999963084153,0) -- (936.4999962946324,0),qqwuqq+linewidth(0.7));
draw( (936.4999962946324,0) -- (936.9999962808495,0),qqwuqq+linewidth(0.7));
draw( (936.9999962808495,0) -- (937.4999962670665,0),qqwuqq+linewidth(0.7));
draw( (937.4999962670665,0) -- (937.9999962532836,0),qqwuqq+linewidth(0.7));
draw( (937.9999962532836,0) -- (938.4999962395007,0),qqwuqq+linewidth(0.7));
draw( (938.4999962395007,0) -- (938.9999962257177,0),qqwuqq+linewidth(0.7));
draw( (938.9999962257177,0) -- (939.4999962119348,0),qqwuqq+linewidth(0.7));
draw( (939.4999962119348,0) -- (939.9999961981518,0),qqwuqq+linewidth(0.7));
draw( (939.9999961981518,0) -- (940.4999961843689,0),qqwuqq+linewidth(0.7));
draw( (940.4999961843689,0) -- (940.999996170586,0),qqwuqq+linewidth(0.7));
draw( (940.999996170586,0) -- (941.499996156803,0),qqwuqq+linewidth(0.7));
draw( (941.499996156803,0) -- (941.9999961430201,0),qqwuqq+linewidth(0.7));
draw( (941.9999961430201,0) -- (942.4999961292372,0),qqwuqq+linewidth(0.7));
draw( (942.4999961292372,0) -- (942.9999961154542,0),qqwuqq+linewidth(0.7));
draw( (942.9999961154542,0) -- (943.4999961016713,0),qqwuqq+linewidth(0.7));
draw( (943.4999961016713,0) -- (943.9999960878883,0),qqwuqq+linewidth(0.7));
draw( (943.9999960878883,0) -- (944.4999960741054,0),qqwuqq+linewidth(0.7));
draw( (944.4999960741054,0) -- (944.9999960603225,0),qqwuqq+linewidth(0.7));
draw( (944.9999960603225,0) -- (945.4999960465395,0),qqwuqq+linewidth(0.7));
draw( (945.4999960465395,0) -- (945.9999960327566,0),qqwuqq+linewidth(0.7));
draw( (945.9999960327566,0) -- (946.4999960189737,0),qqwuqq+linewidth(0.7));
draw( (946.4999960189737,0) -- (946.9999960051907,0),qqwuqq+linewidth(0.7));
draw( (946.9999960051907,0) -- (947.4999959914078,0),qqwuqq+linewidth(0.7));
draw( (947.4999959914078,0) -- (947.9999959776248,0),qqwuqq+linewidth(0.7));
draw( (947.9999959776248,0) -- (948.4999959638419,0),qqwuqq+linewidth(0.7));
draw( (948.4999959638419,0) -- (948.999995950059,0),qqwuqq+linewidth(0.7));
draw( (948.999995950059,0) -- (949.499995936276,0),qqwuqq+linewidth(0.7));
draw( (949.499995936276,0) -- (949.9999959224931,0),qqwuqq+linewidth(0.7));
draw( (949.9999959224931,0) -- (950.4999959087102,0),qqwuqq+linewidth(0.7));
draw( (950.4999959087102,0) -- (950.9999958949272,0),qqwuqq+linewidth(0.7));
draw( (950.9999958949272,0) -- (951.4999958811443,0),qqwuqq+linewidth(0.7));
draw( (951.4999958811443,0) -- (951.9999958673613,0),qqwuqq+linewidth(0.7));
draw( (951.9999958673613,0) -- (952.4999958535784,0),qqwuqq+linewidth(0.7));
draw( (952.4999958535784,0) -- (952.9999958397955,0),qqwuqq+linewidth(0.7));
draw( (952.9999958397955,0) -- (953.4999958260125,0),qqwuqq+linewidth(0.7));
draw( (953.4999958260125,0) -- (953.9999958122296,0),qqwuqq+linewidth(0.7));
draw( (953.9999958122296,0) -- (954.4999957984467,0),qqwuqq+linewidth(0.7));
draw( (954.4999957984467,0) -- (954.9999957846637,0),qqwuqq+linewidth(0.7));
draw( (954.9999957846637,0) -- (955.4999957708808,0),qqwuqq+linewidth(0.7));
draw( (955.4999957708808,0) -- (955.9999957570978,0),qqwuqq+linewidth(0.7));
draw( (955.9999957570978,0) -- (956.4999957433149,0),qqwuqq+linewidth(0.7));
draw( (956.4999957433149,0) -- (956.999995729532,0),qqwuqq+linewidth(0.7));
draw( (956.999995729532,0) -- (957.499995715749,0),qqwuqq+linewidth(0.7));
draw( (957.499995715749,0) -- (957.9999957019661,0),qqwuqq+linewidth(0.7));
draw( (957.9999957019661,0) -- (958.4999956881832,0),qqwuqq+linewidth(0.7));
draw( (958.4999956881832,0) -- (958.9999956744002,0),qqwuqq+linewidth(0.7));
draw( (958.9999956744002,0) -- (959.4999956606173,0),qqwuqq+linewidth(0.7));
draw( (959.4999956606173,0) -- (959.9999956468343,0),qqwuqq+linewidth(0.7));
draw( (959.9999956468343,0) -- (960.4999956330514,0),qqwuqq+linewidth(0.7));
draw( (960.4999956330514,0) -- (960.9999956192685,0),qqwuqq+linewidth(0.7));
draw( (960.9999956192685,0) -- (961.4999956054855,0),qqwuqq+linewidth(0.7));
draw( (961.4999956054855,0) -- (961.9999955917026,0),qqwuqq+linewidth(0.7));
draw( (961.9999955917026,0) -- (962.4999955779197,0),qqwuqq+linewidth(0.7));
draw( (962.4999955779197,0) -- (962.9999955641367,0),qqwuqq+linewidth(0.7));
draw( (962.9999955641367,0) -- (963.4999955503538,0),qqwuqq+linewidth(0.7));
draw( (963.4999955503538,0) -- (963.9999955365709,0),qqwuqq+linewidth(0.7));
draw( (963.9999955365709,0) -- (964.4999955227879,0),qqwuqq+linewidth(0.7));
draw( (964.4999955227879,0) -- (964.999995509005,0),qqwuqq+linewidth(0.7));
draw( (964.999995509005,0) -- (965.499995495222,0),qqwuqq+linewidth(0.7));
draw( (965.499995495222,0) -- (965.9999954814391,0),qqwuqq+linewidth(0.7));
draw( (965.9999954814391,0) -- (966.4999954676562,0),qqwuqq+linewidth(0.7));
draw( (966.4999954676562,0) -- (966.9999954538732,0),qqwuqq+linewidth(0.7));
draw( (966.9999954538732,0) -- (967.4999954400903,0),qqwuqq+linewidth(0.7));
draw( (967.4999954400903,0) -- (967.9999954263074,0),qqwuqq+linewidth(0.7));
draw( (967.9999954263074,0) -- (968.4999954125244,0),qqwuqq+linewidth(0.7));
draw( (968.4999954125244,0) -- (968.9999953987415,0),qqwuqq+linewidth(0.7));
draw( (968.9999953987415,0) -- (969.4999953849585,0),qqwuqq+linewidth(0.7));
draw( (969.4999953849585,0) -- (969.9999953711756,0),qqwuqq+linewidth(0.7));
draw( (969.9999953711756,0) -- (970.4999953573927,0),qqwuqq+linewidth(0.7));
draw( (970.4999953573927,0) -- (970.9999953436097,0),qqwuqq+linewidth(0.7));
draw( (970.9999953436097,0) -- (971.4999953298268,0),qqwuqq+linewidth(0.7));
draw( (971.4999953298268,0) -- (971.9999953160439,0),qqwuqq+linewidth(0.7));
draw( (971.9999953160439,0) -- (972.4999953022609,0),qqwuqq+linewidth(0.7));
draw( (972.4999953022609,0) -- (972.999995288478,0),qqwuqq+linewidth(0.7));
draw( (972.999995288478,0) -- (973.499995274695,0),qqwuqq+linewidth(0.7));
draw( (973.499995274695,0) -- (973.9999952609121,0),qqwuqq+linewidth(0.7));
draw( (973.9999952609121,0) -- (974.4999952471292,0),qqwuqq+linewidth(0.7));
draw( (974.4999952471292,0) -- (974.9999952333462,0),qqwuqq+linewidth(0.7));
draw( (974.9999952333462,0) -- (975.4999952195633,0),qqwuqq+linewidth(0.7));
draw( (975.4999952195633,0) -- (975.9999952057804,0),qqwuqq+linewidth(0.7));
draw( (975.9999952057804,0) -- (976.4999951919974,0),qqwuqq+linewidth(0.7));
draw( (976.4999951919974,0) -- (976.9999951782145,0),qqwuqq+linewidth(0.7));
draw( (976.9999951782145,0) -- (977.4999951644315,0),qqwuqq+linewidth(0.7));
draw( (977.4999951644315,0) -- (977.9999951506486,0),qqwuqq+linewidth(0.7));
draw( (977.9999951506486,0) -- (978.4999951368657,0),qqwuqq+linewidth(0.7));
draw( (978.4999951368657,0) -- (978.9999951230827,0),qqwuqq+linewidth(0.7));
draw( (978.9999951230827,0) -- (979.4999951092998,0),qqwuqq+linewidth(0.7));
draw( (979.4999951092998,0) -- (979.9999950955169,0),qqwuqq+linewidth(0.7));
draw( (979.9999950955169,0) -- (980.4999950817339,0),qqwuqq+linewidth(0.7));
draw( (980.4999950817339,0) -- (980.999995067951,0),qqwuqq+linewidth(0.7));
draw( (980.999995067951,0) -- (981.499995054168,0),qqwuqq+linewidth(0.7));
draw( (981.499995054168,0) -- (981.9999950403851,0),qqwuqq+linewidth(0.7));
draw( (981.9999950403851,0) -- (982.4999950266022,0),qqwuqq+linewidth(0.7));
draw( (982.4999950266022,0) -- (982.9999950128192,0),qqwuqq+linewidth(0.7));
draw( (982.9999950128192,0) -- (983.4999949990363,0),qqwuqq+linewidth(0.7));
draw( (983.4999949990363,0) -- (983.9999949852534,0),qqwuqq+linewidth(0.7));
draw( (983.9999949852534,0) -- (984.4999949714704,0),qqwuqq+linewidth(0.7));
draw( (984.4999949714704,0) -- (984.9999949576875,0),qqwuqq+linewidth(0.7));
draw( (984.9999949576875,0) -- (985.4999949439045,0),qqwuqq+linewidth(0.7));
draw( (985.4999949439045,0) -- (985.9999949301216,0),qqwuqq+linewidth(0.7));
draw( (985.9999949301216,0) -- (986.4999949163387,0),qqwuqq+linewidth(0.7));
draw( (986.4999949163387,0) -- (986.9999949025557,0),qqwuqq+linewidth(0.7));
draw( (986.9999949025557,0) -- (987.4999948887728,0),qqwuqq+linewidth(0.7));
draw( (987.4999948887728,0) -- (987.9999948749899,0),qqwuqq+linewidth(0.7));
draw( (987.9999948749899,0) -- (988.4999948612069,0),qqwuqq+linewidth(0.7));
draw( (988.4999948612069,0) -- (988.999994847424,0),qqwuqq+linewidth(0.7));
draw( (988.999994847424,0) -- (989.499994833641,0),qqwuqq+linewidth(0.7));
draw( (989.499994833641,0) -- (989.9999948198581,0),qqwuqq+linewidth(0.7));
draw( (989.9999948198581,0) -- (990.4999948060752,0),qqwuqq+linewidth(0.7));
draw( (990.4999948060752,0) -- (990.9999947922922,0),qqwuqq+linewidth(0.7));
draw( (990.9999947922922,0) -- (991.4999947785093,0),qqwuqq+linewidth(0.7));
draw( (991.4999947785093,0) -- (991.9999947647264,0),qqwuqq+linewidth(0.7));
draw( (991.9999947647264,0) -- (992.4999947509434,0),qqwuqq+linewidth(0.7));
draw( (992.4999947509434,0) -- (992.9999947371605,0),qqwuqq+linewidth(0.7));
draw( (992.9999947371605,0) -- (993.4999947233775,0),qqwuqq+linewidth(0.7));
draw( (993.4999947233775,0) -- (993.9999947095946,0),qqwuqq+linewidth(0.7));
draw( (993.9999947095946,0) -- (994.4999946958117,0),qqwuqq+linewidth(0.7));
draw( (994.4999946958117,0) -- (994.9999946820287,0),qqwuqq+linewidth(0.7));
draw( (994.9999946820287,0) -- (995.4999946682458,0),qqwuqq+linewidth(0.7));
draw( (995.4999946682458,0) -- (995.9999946544629,0),qqwuqq+linewidth(0.7));
draw( (995.9999946544629,0) -- (996.4999946406799,0),qqwuqq+linewidth(0.7));
draw( (996.4999946406799,0) -- (996.999994626897,0),qqwuqq+linewidth(0.7));
draw( (996.999994626897,0) -- (997.499994613114,0),qqwuqq+linewidth(0.7));
draw( (997.499994613114,0) -- (997.9999945993311,0),qqwuqq+linewidth(0.7));
draw( (997.9999945993311,0) -- (998.4999945855482,0),qqwuqq+linewidth(0.7));
draw( (998.4999945855482,0) -- (998.9999945717652,0),qqwuqq+linewidth(0.7));
draw( (998.9999945717652,0) -- (999.4999945579823,0),qqwuqq+linewidth(0.7));
draw( (999.4999945579823,0) -- (999.9999945441994,0),qqwuqq+linewidth(0.7));

 /* dots and labels */
clip((xmin,ymin)--(xmin,ymax)--(xmax,ymax)--(xmax,ymin)--cycle); 
label("$1$", (0,1), W*2);
label("$T$", (1000,0), S*2);
 /* end of picture */
\end{asy}
\end{center}


where the $y$-axis is $\ov{\alpha}_t$ (i.e., the proportion of the original image that we are preserving), $x$-axis is time, and the green and red schedules are linear and cosine schedules, respectively. The cosine schedule is more gradual and has been shown to generally produce better results for smaller image sizes, e.g., $32x32$ pixels. 

\subsection{Going Backwards}

The more difficult step in diffusion models is figuring out how to generate data in the complex data space (i.e., an image) from pure noise. When $\beta_t$ is small, $q(x_{t-1}|x_t)$ is essentially a gaussian, so we can attempt to learn $p_{\theta}(x_{t-1}|x_t) = \Norm(x_{t-1}; \mu_{\theta}(x_t,t), \Sigma_{\theta}(x_t,t))$. 

Unfortunately, $q(x_{t-1}|x_t)$ is intractable, but $q(x_{t-1}|x_t,x_0)$ can be calculated by flipping everything to the forwards direction with Bayes. Per the normal distribution for multivariate distributions, we want something that looks like this: 
\[q(x_{t-1}|x_t,x_0) = \frac{1}{\sqrt{(2\pi)^k \det \Sigma_{\theta}(x_t,t)}}\exp\left(-\frac{\lVert x_{t-1} - \mu_{\theta}(x_t,t) \rVert^2}{2\det \Sigma_{\theta}(x_t,t)}\right),\] 
where we implicitly assume $\Sigma_{\theta}(x_t,t) = \tilde{\beta}_tI$ for some derived value of $\tilde{\beta}_t$ (our noise is independent). Recall that $q(x_t|x_{t-1}) = \Norm(x_t; \sqrt{\alpha_t}x_{t-1}, \beta_t I)$ and $q(x_{t}|x_{0})\sim \Norm(x_t; \sqrt{\ov{\alpha}_t}x_0, \sqrt{1-\ov{\alpha}_t}I)$, so we have 
\begin{align*}
	&q(x_{t-1}|x_t,x_0) = q(x_t|x_{t-1},x_0)\frac{q(x_{t-1}|x_0)}{q(x_t|x_0)}\\
	&\quad= \frac{1}{\sqrt{(2\pi)^k\beta_t^k}}\frac{\sqrt{(2\pi)^k(1-\ov{\alpha}_t)^k}}{\sqrt{(2\pi)^k(1-\sqrt{\ov{\alpha}_{t-1}})^k}}\exp\left(-\frac{\lVert x_{t-1} - \mu_{\theta}(x_t,t)\rVert^2)}{2\Sigma_{\theta}(x_t,t)}\right).
\end{align*}
Therefore,
\[\tilde{\beta} = \frac{1-\sqrt{\ov{\alpha}_{t-1}}}{1-\sqrt{\ov{\alpha}_t}}\beta_t.\]
To compute the mean, we continue expanding inside of the exp: 
\begin{align*}
	&q(x_{t-1}|x_t,x_0) = q(x_t|x_{t-1},x_0)\frac{q(x_{t-1}|x_0)}{q(x_t|x_0)} \\
	&\quad = \frac{1}{\sqrt{(2\pi\tilde{\beta})^k}}\exp\left(-\frac{1}{2}\left(\frac{(x_t - \sqrt{\alpha}_tx_{t-1})^2}{\beta_t} + \frac{(x_{t-1} - \sqrt{\ov{\alpha}_{t-1}}x_0)^2}{1-\ov{\alpha}_{t-1}} - \frac{(x_t-\sqrt{\ov{\alpha}_t}x_0)^2}{(1-\ov{\alpha}_t)}\right)\right) \\
	&\quad= \frac{1}{\sqrt{(2\pi\tilde{\beta})^k}}\exp\left(-\frac{1}{2}\left(x_{t-1}^2\left(\frac{\alpha_t}{\beta_t} + \frac{1}{1-\ov{\alpha}_{t-1}}\right)\right.\right. \\
	&\qquad\qquad\qquad \left.\left. -x_{t-1}\left(\frac{2\sqrt{\alpha_t}}{\beta_t}x_t + \frac{2\sqrt{\ov{\alpha}_{t-1}}}{1-\ov{\alpha}_{t-1}}x_0\right) + \frac{x_t^2}{\beta_t} + \frac{\ov{\alpha}_{t-1}x_0^2}{1-\ov{\alpha}_{t-1}} - \frac{(x_t-\sqrt{\ov{\alpha}_t}x_0)^2}{1-\ov{\alpha}_t}\right)\right),
\end{align*}
where matrix multiplications are handwaved. We can verify 
\[\frac{\alpha_t}{\beta_t} + \frac{1}{1-\ov{\alpha}_{t-1}} = \frac{\alpha_t-\ov{\alpha}_t + 1-\alpha_t}{\beta_t(1-\ov{\alpha}_{t-1})} = \frac{1}{\tilde{\beta_t}},\]
so we have
\begin{align*}
q(x_{t-1}|x_t,x_0) &= \frac{1}{\sqrt{(2\pi\tilde{\beta}^k)}}\exp\left(-\frac{\lVert x_{t-1} - \mu_{\theta}(x_t,t)\rVert^2}{2\tilde{\beta}_t^k}\right). 
\end{align*}
(the exponents don't quite match up because we are handwaving away all the matrix logic). Comparing coefficients gives 
\begin{align*}
	\mu_{\theta}(x_t,t) &= \left(\frac{\sqrt{\alpha}_t}{\beta_t}x_t + \frac{\sqrt{\ov{\alpha}_{t-1}}}{1-\ov{\alpha}_{t-1}}x_0\right)\tilde{\beta} \\
											&= \frac{\sqrt{\alpha_t}(1-\ov{\alpha}_{t-1})}{1-\ov{\alpha}_t}x_t + \frac{\sqrt{\ov{\alpha}_{t-1}}\beta_t}{1-\ov{\alpha}_t}x_0 \\
											&= \frac{\sqrt{\alpha_t}(1-\ov{\alpha}_{t-1})}{1-\ov{\alpha}_t}x_t + \frac{\sqrt{\ov{\alpha}_{t-1}}\beta_t}{1-\ov{\alpha}_t}\frac{1}{\sqrt{\ov{\alpha}_t}}(x_t - \sqrt{1-\ov{\alpha}_t}\varepsilon_t) \\
											&= \frac{1}{\sqrt{\alpha_t}}\frac{\alpha_t-\ov{\alpha}_t}{1-\ov{\alpha}_t}x_t + \frac{1}{\sqrt{\alpha_t}}\frac{1-\alpha_t}{1-\ov{\alpha}_t}(x_t - \sqrt{1-\ov{\alpha}_t}\varepsilon_t) \\
											&= \frac{1}{\sqrt{\alpha_t}}\left(x_t - \frac{1-\alpha_t}{\sqrt{1-\ov{\alpha}_t}}\varepsilon_t\right). 
\end{align*}

\subsection{Theoretical Loss}

We can now explicitly compute $q(x_{t-1}|x_t,x_0)\sim \Norm(x_{t-1};\mu_{\theta}(x_t,t), \Sigma_{\theta}(x_t,t))$, so we can compute the backwards diffusion step at any timestep conditioned on knowing $x_0$. To find the best possible $x_0$, we can optimize log likelihood use the same ELBO lower bound used to optimize VAEs and gaussian mixtures: 

\begin{align*}
	\log p_{\theta}(x_0) \geq \EE_q\left(\frac{p_{\theta}(x_0,x_{1:T})}{q(x_{1:T}|x_0)}\right) = \EE_q\left(\frac{p_{\theta}(x_{0:T})}{q(x_{1:T}|x_0)}\right). 
\end{align*}
To turn this into a minimization problem, we optimize $L_{VLB} = -\log p_{\theta}(x_0)$, where VLB stands for variational lower bound. With a bit of algebra, we can simplify this expression:
\begin{align*}
	L_{VLB} &= \EE_q\left(\log \frac{q(x_{1:T}|x_0)}{p_{\theta}(x_{0:T})}\right) \\
					&= \EE_q\left(\log \frac{\prod_{t=1}^T q(x_t|x_{t-1})}{p_{\theta}(x_T)\prod_{t=1}^Tp_{\theta}(x_{t-1}|x_t)}\right) \\
					&= \EE_q\left(-\log p_{\theta}(x_T) + \left(\sum_{t=1}^T \log \frac{q(x_t|x_{t-1})}{p_{\theta}(x_{t-1}|x_t)}\right) + \log \frac{q(x_1|x_0)}{p_{\theta}(x_0|x_1)}\right) \\
					&= \EE_q \left(-\log p_{\theta}(x_T) + \left(\sum_{t=2}^T \log \frac{q(x_{t-1}|x_t,x_0)q(x_t|x_0)}{p_{\theta}(x_{t-1}|x_t)q(x_{t-1}|x_0)}\right) + \log \frac{q(x_1|x_0)}{p_{\theta}(x_0|x_1)}\right) \\
					&= \EE_q \left(-\log p_{\theta}(x_T) + \left(\sum_{t=2}^T \log \frac{q(x_{t-1}|x_t,x_0)}{p_{\theta}(x_{t-1}|x_t)}\right) + \log \frac{q(x_T|x_0)}{q(x_1|x_0)} + \log \frac{q(x_1|x_0)}{p_{\theta}(x_0|x_1)}\right) \\
					&= \EE_q \left(\log \frac{q(x_T|x_0)}{p_{\theta}(x_T)} + \left(\sum_{t=2}^T \log \frac{q(x_{t-1}|x_t,x_0)}{p_{\theta}(x_{t-1}|x_t)}\right) - \log p_{\theta}(x_0|x_1)\right) \\
					&= D_{KL}(q(x_T|x_0)\Vert p_{\theta}(x_T)) + \sum_{t=2}^T D_{KL}(q(x_{t-1}|x_t,x_0)\Vert p_{\theta}(x_{t-1}|x_t)) - \log p_{\theta}(x_0|x_1) \\
					&= L_T + \sum_{t=2}^T L_{t-1} + L_0.
\end{align*}

\subsection{Training Loss}

In practice, $L_T$ can be ignored, since $x_T$ is always pure gaussian noise. There are some things we can do for $L_0$ that are not that important here. Therefore, we care about optimizing $L_t$ for $t=1$ to $T-1$.  

The \href{https://en.wikipedia.org/wiki/Multivariate_normal_distribution#Kullback%E2%80%93Leibler_divergence}{KL divergence for multivariate normal distributions} is given by 
	\[D_{KL}(\Norm_0 || \Norm_1) = \frac{1}{2}\left(\Tr(\Sigma_1^{-1}\Sigma_0) + \frac{1}{\det \Sigma_1}\lVert \mu_1-\mu_0\rVert^2 - k + \ln \frac{\det \Sigma_1}{\det \Sigma_0}\right).\]
To minimize $L_{t}$, we only care about terms that depend on $\theta$. Our expression for posterior variance 
\[\tilde{\beta_t} = \frac{1-\ov{\alpha}_{t-1}}{1-\ov{\alpha}_t}\cdot \beta_t\]
does not depend on $\theta$, since all of our $\alpha$,$\beta$ are predetermined, so the only terms we care about in the expression for KL divergence are the terms with $\mu_0,\mu_1$, i.e., 
\[\frac{1}{\det \Sigma_1}\lVert \mu_1 - \mu_0\rVert^2,\]
in the notation of the above expression. We know $q(x_{t-1}|x_t,x_0) = \Norm(x_{t-1}; \tilde{\mu}(x_t,x_0), \tilde{\beta}_tI)$, and we want our model $p_{\theta}(x_{t-1}|x_t)$ to learn $\Norm(x_{t-1}; \mu_{\theta}(x_t,t), \Sigma_{\theta}(x_t,t))$, where $\mu_{\theta}(x_t,t)\approx \tilde{\mu}(x_t,x_0)$ and $\Sigma_{\theta}(x_t,t)\approx \tilde{\beta}_t I$. Since we are ultimately trying to minimize log loss across all possible $x_0$, we minimize the EV across $x_0$. Also, like the reparamaterization trick for VAEs, we don't want to sample directly from each gaussian when computing values like $\tilde{\mu}_t(x_t,x_0)$; instead, we first sample $\varepsilon_t\sim \Norm(0,1)$, and let $\tilde{\mu}$ take it as a parameter, so that we can backprop through the function properly. In sum, our goal is to now minimize EV across both $x_0$ and $\varepsilon_t$, which gives us
\[L_{t} - C = \EE_{x_0, \varepsilon_t}\left(\frac{1}{2\lVert\Sigma_{\theta}(x_t,t)\rVert_2^2}\lVert \tilde{\mu}_t(x_t, x_0) - \mu_{\theta}(x_t,t)\rVert^2\right)\] 
with $C$ independent of $\theta$. Plugging in known values, this simplifies: \begin{align*}
	L_t - C &= \EE_{x_0,\varepsilon_t}\left(\frac{1}{2\lVert\Sigma_{\theta}(x_t,t)\rVert_2^2}\left\lVert\frac{1}{\sqrt{\alpha_t}}\left(x_t - \frac{1-\alpha_t}{\sqrt{1-\ov{\alpha}_t}}\varepsilon_t\right) - \frac{1}{\sqrt{\alpha_t}}\left(x_t - \frac{1-\alpha_t}{\sqrt{1-\ov{\alpha}_t}}\varepsilon_{\theta}(x_t,t)\right)\right\rVert^2\right) \\
					&= \EE_{x_0,\varepsilon_t}\left(\frac{(1-\alpha_t)^2}{2\alpha_t(1-\ov{\alpha}_t) \lVert\Sigma_{\theta}(x_t,t)\rVert^2}\lVert\varepsilon_t - \varepsilon_{\theta}(\sqrt{\ov{\alpha}_t}x_0 + \sqrt{1-\ov{\alpha}_t}\varepsilon_t, t)\rVert^2\right). 
\end{align*} 
In practice, it has been shown that total loss $L$ given by  
\[L - C = \EE_{t,x_0,\varepsilon_t}(\lVert\varepsilon_t - \varepsilon_{\theta}(\sqrt{\ov{\alpha}_t}x_0 + \sqrt{1-\ov{\alpha}_t}\varepsilon_t,t)\rVert^2)\] 
gives better results (note that we have added $t$ to the expected value, so this represents a total loss objective over all timesteps). This is really convenient for training, because this is just the expected mean squared error between the actual noise added to an image at time $t$, $\varepsilon_t$, and the error predicted by the backwards diffusion process. We can therefully usually train by randomly selecting a batch of images, timesteps, and noise to add to each image, and then incurring loss equal to the MSE of the noise predicted by the unet and the actual noise that was added to each image. 

\subsection{Turning diffusion models into classifiers}

In addition to normal diffusion models that learn $p_{\theta}(x_{0:T})$, there are also conditional diffusion models that learn $p_{\theta}(x_{0:T}|c_i)$. The only difference between class conditional models and normal models is that the neural network learned during the backwards diffusion process takes the class as an additional parameter. It turns out that we can also use these models as classifiers. 

\[p_{\theta}(c_i|x) = \frac{p(c_i)p_{\theta}(x|c_i)}{\sum_j p(c_j)p_{\theta}(x|c_j)}.\]
The learned UNet produces $\mathcal{L}_{\theta}(x|c_i)\leq \log p_{\theta}(x|c_i)$. In principle, our learned ELBO should be the same as the actual log likelihood, so we may say $\mathcal{L}_{\theta}(x|c_i)\approx \log p_{\theta}(x|c_i)$. Assuming uniform priors on all of the classes, we thus have 
\[p_{\theta}(c_i|x) = \frac{\exp(-\EE_{t,\varepsilon}\lVert \varepsilon - \varepsilon_{\theta}(x_t,c_i) \rVert^2)}{\sum_j \exp(-\EE_{t,\varepsilon}\lVert \varepsilon-\varepsilon_{\theta}(x_t,c_j)\rVert^2)}.\] 
ELBO values can be approximated via Monte Carlo by sampling $(t,\varepsilon)$ pairs, using the trained UNet to predict $\varepsilon_{\theta}(x_i,c_i)$, and computing the expected mean squared loss of the errors over all samples. This is an overkill but interesting way to classify things that has shown decent results. 
