\section{April 25, 2023}

\subsection{Bootstrapping}

\begin{definition}
\deflabel

\ac{Bootstrapping} is a simulation based method to assess the variability of any estimator. 
\end{definition}

Consider some 
\[\hat{\theta}_n^{(1)}, \hat{\theta}_n^{(2)}, \hdots, \hat{\theta}_n^{(k)}\sim P.\]
A reasonable estimator for the mean: $\hat{\theta}_n^* = \sum_{i=1}^k\hat{\theta}_n^{(i)}/k$. By the strong law of large numbers, this is an unbiased estimator, i.e., it converges almost surely to $\EE[\hat{\theta}_n]$. A reasonable estimator for variance: $v_{boot} = \sum_{i=1}^k(\hat{\theta}_n^{(i)}-\hat{\theta}_n^*)^2$. By the strong law of large numbers, this also converges to $\var[\hat{\theta}_n]$.

Main idea: split the data into some number of blocks, and then compute effectively i.i.d. estimators:
\begin{align*}
    X^{(1)} &=\{X_1^{(1)}, \hdots, X_n^{(1)}\} \rightarrow \hat{\theta}_n^{(1)}\\
    X^{(2)} &=\{X_1^{(2)}, \hdots, X_n^{(2)}\} \rightarrow \hat{\theta}_n^{(2)}\\
    &\vdots\\
\end{align*}

To construct confidence intervals:

\begin{example}
\exlabel

Constructing normal confidence intervals from bootstrapped samples.
\end{example}

Given $\hat{\theta}$ asymptotically normal, i.e., $\sqrt{n}(\hat{\theta}-\theta_n)\cd \Norm(0,V(\hat{\theta}_n))$, then 
 \[C.I._{boot} = \left[\hat{\theta}_n-Z_{\alpha/2}\sqrt{v_{boot}}, \hat{\theta}_n+Z_{\alpha/2}\sqrt{v_{boot}}\right].\]

\begin{example}
\exlabel

Constructing \ac{pivotal intervals} from bootstrapped samples. 
\end{example}

This method is generally more popular than the first method. The goal is to find $t$ such that 
\[\PP[-t\leq \hat{\theta}-\theta \leq t]\approx 1-\alpha,\]
so that $[\hat{\theta}-t, \hat{\theta}+t]$ is a $(1-\alpha)$-level confidence interval. 

We can compute some estimator for this value $t_{\alpha}$ s.t.
\[\frac{1}{B}\sum_{i=1}^B \mathbbm{1}(-t_{\alpha}\leq \hat{\theta}_n-\hat{\theta}^{(i)}\leq t_{\alpha})\approx 1-\alpha.\]
Then, our interval is 
\[C.I._{boot} = [\hat{\theta}_n - t_{\alpha}, \hat{\theta}+t_{\alpha}].\]

\begin{example}
\exlabel

Histogram based confidence intervals. 
\end{example}

Compute $\hat{\theta}_n^{(1)}, \hdots, \hat{\theta}_n^{(B)}$. Then, construct empirical quantiles by plotting these values in a histogram, and take 
\[C.I._{boot} = [q_{\alpha/2}^*, q_{1-\alpha/2}^*].\]