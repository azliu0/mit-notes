\section{April 6, 2023}

Intractability II. 

\subsection{Circuit-SAT}

The input to Circuit-SAT:
\begin{itemize}
    \item Boolean circuit $C$ on variables $x_1, \hdots, x_n$
\end{itemize}
Output: 
\begin{itemize}
    \item YES if $C$ is satisfiable, and NO otherwise. 
\end{itemize}

\begin{definition}
\deflabel

A \ac{boolean circuit} is a directed acyclic graph. Each node represents a boolean gate, e.g., AND, OR, NOT, that takes in the appropriate number of inputs and outputs any number of outputs which all have the value of the operation on the inputs. One of the nodes is designated as the output node. The input nodes $x_1, \hdots, x_n$ are fed through the graph, and the output of the graph is said to be the output value of the output node. 
\end{definition}

Boolean circuits can be evaluated in linear time. In particular, because they are DAGs, there is a topological ordering of the vertices. Since the input nodes $x_1, \hdots, x_n$ are sources, and the output node is a sink, we can use DP to compute the value of each node in linear time. 

Circuit-SAT asks whether $C$ is satisfiable. $C$ is said to be \ac{satisfiable} if there exists an initial assignment to $x_1, \hdots, x_n$ which produces an output value of $1$. The fastest known algorithm is to brute force all inputs, $2^n$. 

\begin{theorem}
\thmlabelname{Cook-Levin} 

Circuit-SAT is NP-complete. 
\end{theorem}

For any problem $\pi$, there are two steps to show that $\pi$ is NP-complete:
\begin{itemize}
    \item prove that $\pi\in NP$. 
    \item prove that there exists a reduction $Q\leq_p \pi$ from any problem $Q\in NP$. 
\end{itemize}

\begin{theorem}
\claimlabel

Circuit-SAT $\in$ NP.
\end{theorem}

\begin{proof}
Verifying Circuit-SAT is equivalent to evaluating a boolean circuit in linear time, which we discussed above is possible with dynamic programming.
\end{proof}

To prove that Circuit-SAT is NP-hard, we will make use of the following theorem:

\begin{theorem}
\thmlabel

Let $Q$ be some decision algorithm, and $A$ an algorithm solving $Q$ in $p(n)$ time on inputs of size $n$. Then, for every fixed $n$, there is a Boolean circuit $C_n$ of size $\text{poly}(p(n))$ s.t. for every $n$-length input $x$, $Q(x) = C_n(x)$. And, $C_n$ can be constructed in $\text{poly}(p(n))$ time. 
\end{theorem}

Assumption of the model:
\begin{itemize}
    \item Let $L_i$ be the state in memory at step $i$. 
    \item Between states $L_i$ and $L_{i+1}$, a single operation is performed transforming $L_i$ into $L_{i+1}$. 
    \item Then, there is a fixed boolean circuit $M$ that takes $L_i$ to $L_{i+1}$, which has size poly($p(n)$), and can be written down in poly($p(n)$). This is a reasonable assumption because this is how processors work. 
\end{itemize}

\begin{proof}
Look at a run of $A$ on size $n$ inputs. Each operation can be modelled by a change in state from $L_i$ to $L_{i+1}$ on whatever machine is running our algorithm. Therefore, given the sequence of states $L_0$, $\hdots$, $L_{p(n)}$ which runs $A$, there is a sequential circuit taking $L_0$ to $L_{p(n)}$ with size poly($p(n)$), so we are done.
\end{proof}

\begin{theorem}
\thmlabel

Circuit-SAT is NP hard. 
\end{theorem}

\begin{proof}
Let $V_Q$ be the verifier algorithm for $Q$. Let $x$ be an instance of $Q$ of size $n$. $V_Q(x,y)$ where $\vert x\vert = n$ and $\vert y\vert = p(n)\in$ poly($n$). Let $N = \vert x\vert + \vert y \vert \in$ poly($n$). By the previous theorem, there exists a boolean circuit $C_N$ such that $C_N(x,y) = V_Q(x,y)$ for all $x,y$ which is constructable in poly($n$) time.

Now we have a circuit $C_N$ which takes $x,y$ as input and produces $V_Q(x,y)$ in poly($n$) time. We want to reduce $x$ itself to a boolean circuit problem. Since $x$ has a binary representation, hard wire its input in $C_N$, which produces a circuit $C_{N,x}$ whose only input is a certificate $y$. 

Finally, $Q(x)$ has a solution if and only if there exists some certificate $y$ such that $V_Q(x,y)=1$. Since the circuit $C_{N,x}$ takes in any certificate and produces $V_Q(x,y)$ in polynomial time, finding $Q(x)$ is equivalent to seeing whether there exists an input to $C_{N,x}$ which produces a positive output value, which is Circuit-SAT, and we are done.
\end{proof}

\subsection{CNF-SAT, 3-SAT}

CNF: Conjunctive Normal Form. 

\begin{definition}
\deflabel

$F$ is a CNF formula on $x_1, \hdots, x_n$ if it is of the form
\[C_1\wedge C_2\wedge \hdots \wedge C_m,\]
where each $C_i$ is a \ac{clause} of the form $(l_1\vee \hdots \vee l_{k_i})$, and each \ac{literal} $l_i$ is either $x_r$ or $\neg x_r$ for some $r$. 
\end{definition}

$F$ is a 3-CNF formula if every clause has three literals. The 3SAT problem takes in some $3$-CNF formula $F$ on $n$ variables and outputs whether there is a boolean assignment on which $F$ evaluates to true. 

\begin{theorem}
\thmlabel

$3$SAT is NP-complete. 
\end{theorem}

\begin{proof}
$3$SAT is in NP because it is trivial to check whether a boolean assignment $y$ satisfies $F$. 

Now, we show that $3$SAT is NP-hard by showing that there exists a reduction from Circuit-SAT to $3$-SAT. Take any instance of Circuit-SAT $C$ on $n$ variables $x_1, \hdots, x_n$, with $t$ gates and $m$ wires. Since we may assume everything is connected, all sizes are $O(m)$. At each gate, define a variable $g_i$. Now, construct the following ``clauses'' on $g_1, \hdots, g_t$ (these are not necessarily proper CNF clauses):
\begin{itemize}
    \item The value of the output gate is $1$
    \item For every $i$, $g_i$ computes the value of the gate. For example, the clause corresponding to $g_2$ would be $(g_2\iff \neg x_3)$ for a NOT gate which takes $x_3$ as input. 
\end{itemize}
It is possible to construct all of these ``clauses'' in linear time $O(m)$ by iterating through the circuit in topological order. 

To finish, rewrite each ``clause'' with proper CNF clauses. For example, $(g_i\iff \neg a)$ is equivalent to $(\neg g_i\vee \neg a)\wedge (g_i\vee a)$. By writing out the full normal form of each possible gate, it is possible to show that we can replace each ``clause'' with at most $3$ CNF clauses, hence our final expression is a $3$-CNF formula and we are done.
\end{proof}

