\section{September 21, 2022}

\subsection{Conjugation in $S_n$}

Given $\pi\in S_n$, the conjugate of any elt wrt $\pi$ is given by 
\[\pi (a_1 a_2 \hdots a_k)\pi^{-1} = (\pi(a_1) \pi(a_2) \hdots \pi(a_k)).\]

This can be proved by considering the movement of any element. Suppose for some $a_i$ that $\pi^{-1}(a_i) = a_j$. Then the left hand side maps $a_i\mapsto a_j\mapsto a_{j+1}\mapsto \pi(a_{j+1})$. On the other hand, since $\pi(a_j)=a_i$, the right hand side maps $a_i\mapsto \pi(a_{j+1})$, so they map to the same thing. 

\subsection{Fields and Vector Spaces}

We started with group theory, now we'll do some linear algebra before combining them again later. 

\begin{definition}
\deflabel 

A \ac{field} is a set $F$ with binary operators $+, \cdot$ such that 
\begin{enumerate}
    \item [1.] $(F, +)$ is an abelian group
    \item [2.] $(F\backslash\{0\}, \cdot)$ is an abelian group
    \item [3.] $x(y+z) = xy+xz$
\end{enumerate}
\end{definition}

We need the third condition (distributive law) in order to define how our two binary operators interact. 

\begin{example}
\exlabel

Here are some examples of fields. 
\end{example}

$\RR$, $\QQ$, and $\CC$ are all fields. Note the exclusion of $0$ when we're dealing with multiplication, which makes them all valid abelian groups. $\ZZ/p\ZZ = \{0, 1, \hdots, p-1\}$ is also a field, where we define $+,\cdot$ modulo $p$. 

\begin{example}
\exlabel

Here are some non-examples of fields. 
\end{example}

$\ZZ$ is not a field, since we can't do division (i.e., inverses) in $\ZZ$. In general, if you can't do division, the set that you're working with can't be a field. Unlike the previous example with prime order, $\ZZ/6\ZZ$ is not a field, since $2$ and $3$ do not have multiplicative inverses. 

\begin{theorem}
\lemlabel

Multiplicative inverses for elements in $\ZZ/p\ZZ^x$ exist.
\end{theorem}

\begin{proof}
Given $a\not\equiv 0\pmod{p}$, we want to show the existence of some $x$ such that $ax\equiv 1\pmod{p}$. Consider the subgroup of $(\ZZ, +)$ generated by $a,p$: 
\[a\ZZ+p\ZZ = \{ax+py: x,y\in \ZZ\} = d\ZZ,\]
since every subgroup in $\ZZ$ has a single generator. $a$, $p$ must be multiples of $d$, but $\gcd(a,p)=1\implies d=\pm 1$. Taking $d=1$, we conclude that $ax+py=1$ for some $x,y\in \ZZ$, so $ax\equiv 1\pmod{p}$ has a solution. 
\end{proof}

\begin{definition}
\deflabel

A \ac{vector space} $V$ over field $F$ is defined as a set $V$, a binary operator $+$, and a scalar multiplication $F\times V\rightarrow V$ with $(\lambda, v)\mapsto \lambda v$. These operators must satisfy the following: 
\begin{itemize}
    \item $(V,+)$ is an abelian group
    \item $1v=v\quad \forall v\in V$
    \item $\lambda(\mu v) = \mu(\lambda v)\quad \forall \mu, \lambda\in F$
    \item $\lambda(v+w) = \lambda v+\lambda w$\newline
    $(\lambda+\mu)v = \lambda v + \mu v\quad\forall v,w\in V, \mu, \lambda\in F$
\end{itemize}
\end{definition}

Here are some common examples of vector spaces. 

\begin{example}
\exlabel

Examples of vector spaces.
\end{example}

\begin{itemize}
    \item $F^n$, the set of $n$-dimensional column vectors with elements in field $F$, is an $F$-vector space. 
    \item Given $A\in F^{m\times n}$, $\{x\in F^n : Ax=0\}$, or the set of solutions to homogenous linear equations, is an $F$-vector space. This is also a \ac{subspace} of $F^n$. 
    \item $\CC$ is an $\RR$-vector space (and a $\CC$-vector space)
    \item $\RR$ is a $\QQ$-vector space which is infinite dimensional
    \item $\{\text{cont. functions from }\RR\text{ to }\RR\}$ is an $\RR$-vector space which is also infinite dimensional
    \item $\{\text{solns to }y''=-y\}$ is an $\RR$-vector space which is $2$-dimensional, since all solutions are a linear combination of $\cos{t}$ and $\sin {t}$ (equivalently, $e^{it}$ and $e^{-it}$). This vector space is also a subspace of the previous example.
    \item polynomials with real coefficients of degree $<n$ is an $n$-dimensional vector space over $\RR$
\end{itemize}