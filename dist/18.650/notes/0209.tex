\section{February 9, 2023}

\subsection{Delta Method}

Last time, we learned about Slutsky's Theorem. Why is it the case that 
\[\begin{rcases*}Y_n\cd Y \\
Z_n\cd Z\end{rcases*}\centernot\implies Y_n+Z_n\cd Y+Z?\]
Consider the following counterexample:
\[Y_n = \sqrt{n}\frac{\ov{X_n} - \mu}{\sigma}\cd \Norm(0,1) \quad\text{and}\quad Z_n=-Y_n\cd \Norm(0,1).\]
In this case, $Y_n+Z_n\centernot\rightarrow Y+Z = \Norm(0,2)$.

\begin{theorem}
\thmlabelname{The Delta Method}

Suppose
\begin{itemize}
    \item $\sqrt{n}(Y_n-\theta) \cd \Norm(0, \sigma^2)$
    \item $g$ is continuously differentiable at $\theta$
\end{itemize}
Then, 
\[\sqrt{n}(g(Y_n) - g(\theta))\cd \Norm(0, g'(\theta)^2\sigma^2).\]
\end{theorem}

Note the similarities to the CLT. If we treat $Y_n = \sum_{i=1}^n X_i$, then our $\theta$ is $\EE[X_i]$. The only difference here is that we're calculating the distribution of $g(Y_n)$ instead of $Y_n$ itself. 

\begin{theorem}
\lemlabel

If $\vert Y_n - X_n\vert \cp 0$ and $X_n\cd X$, then $Y_n\cd X$. 
\end{theorem}

\begin{proof}
% It suffices to verify that 
% \[\EE[f(X_n)]\cinf \EE[f(X)]\]
% for all functions $f$. 
Let $Z_n = Y_n-X_n$. By Slutsky's theorem, $(X_n, Z_n)\cd (0, X)$, which implies $X_n+Z_n = Y_n \cd X$.
\end{proof}

\subsection{Kissing Experiment (confidence intervals)}

Let $p$ be the proportion of couples that turn their head to the right when kissing. Observe $n=124$ couples kissing. It turns out that $80$ couples turned to the right. Therefore, we can estimate $p$ with the estimator
\[\phat = \frac{80}{124} = 64.5\%.\]
It seems intuitively true that there is a preference for couples to turn to the right, since $65.5\% > 50\%$. On the other hand, if we observed $n=3$ couples, and found that $2$ of them turned to the right, we would be less convinced that this is necessarily the case. At what threshold are we actually convinced that $p > 50\%$?

\hrulebar

Define a sequence of random variables $\{R_i\}_{1\leq i\leq n}$, where $R_i=1$ if the $i$th couple turns to the right, and $R_i = 0$ otherwise. For the sake of our model, we assume: 
\begin{itemize}
\setlength \itemsep{0cm}
\item $R_i\sim \Bern(p)$. Modelling each $R_i$ as a r.v. is how we deal with the lack of other information. If we knew more, we could use psychology or physics to deduce whether there is a natural tendency to lean right while kissing (in other words, we would not need to use statistics).    
    \item $R_1, \hdots, R_n$ are mutually independent. This is reasonable since the behavior of one couple does not interfere with the behavior of another.
\end{itemize}

Now, by the strong law of large numbers,
\[\phat = \ov{R_n} \cas p.\]
How do we quantify how confident we are with our estimate when $n$ is not infinitely large? By the CLT, 
\[\PP\left(\sqrt{n}\frac{\ov{R_n} - p}{\sqrt{p(1-p)}}\leq x\right) \cinf \PP(\Norm(0,1)\leq x),\]
for all \ac{quantiles} $x$. In other words, for large $n$, we may say 
\[\sqrt{n}(\ov{R_n}-p)\approx \Norm(0, \sigma^2) = \Norm(0, p(1-p)),\]
which implies
\[\PP[\vert \ov{R_n}-p\vert \geq a/\sqrt{n}]\approx \PP[\vert \Norm(0,1)\vert \geq a/\sigma] = 2-2\Phi\left(\frac{a}{\sigma}\right).\]
Let $q_{\alpha/2}$ be the $(1-\alpha/2)$ quantile of $\Norm(0,1)$, i.e.,
\[1-\alpha/2 = \Phi\left(q_{\alpha/2}\right).\]
Then, 
\[\PP[\vert \Norm(0,1)\vert \leq q_{\alpha/2}] = 1-\alpha.\]
Per the equation earlier, we thus have that, with probability $1-\alpha$,
\[\vert \ov{R_n} - p\vert \leq \frac{q_{\alpha/2}\sigma}{\sqrt{n}}\leq \frac{q_{\alpha/2}}{2\sqrt{n}},\]
following from the fact that $\sigma = \sqrt{p(1-p)}\leq 1/2$.

\begin{definition}
\deflabel

The interval given by 
\[\left[\ov{R_n} - \frac{\vert q_{\alpha/2}\vert}{2\sqrt{n}}, \ov{R_n} + \frac{\vert q_{\alpha/2}\vert}{2\sqrt{n}}\right]\]
is called the \ac{$1-\alpha$ Confidence Interval} (C.I.) for $p$. 
\end{definition}

Intuition checks:
\begin{itemize}
\setlength \itemsep{0cm}
    \item This naming makes sense, because the probability that $p$ lies in this interval is $1-\alpha$ (by rearranging the equation we had earlier). 
    \item When $\alpha=1$, we are $0\%$ confident that $p$ lies in the interval. Indeed, $q_{1/2} = 0$, so the interval has length $0$.
    \item When $\alpha=0$, we are $100\%$ confident that $p$ lies in this interval. Indeed, $\vert q_{0}\vert = \infty$, so the interval spans all possible values for $p$.
\end{itemize}