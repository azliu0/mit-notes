\section{January 10, 2023}

\subsection{Basic Counting Principles}

\begin{theorem}
\lemlabelname{Addition Principle}

For $n$ disjoint sets $S_1, S_2, \hdots, S_n$, the cardinality of their sum 
\[\norm{S_1 + S_2 + \hdots + S_n} = \norm{S_1} + \hdots + \norm{S_n}.\]
\end{theorem}

\begin{definition}
\deflabel

Let $S_1, \hdots, S_n$ be finite sets. Their \ac{Cartesian Product} is defined 
\[S_1\times \hdots \times S_n = \{(s_1, \hdots, s_n)\vert s_1\in S_1, \hdots, s_n\in S_n\}.\]
\end{definition}

\begin{theorem}
\lemlabelname{Multiplication Principle}

For $n$ disjoint sets $S_1, S_2, \hdots, S_n$, the cardinality of their cartesian product 
\[\norm{S_1 \times \hdots \times S_n} = \norm{S_1} \cdot \hdots \cdot \norm{S_n}.\]
\end{theorem}

\begin{definition}
\deflabel

A function $f: A\rightarrow B$ between finite sets $A$ and $B$ is called a \ac{bijection} if $f(a)=f(b)\implies a=b$ (injectivity) and for all $b\in B$, there exists $a\in A$ such that $f(a) = b$ (surjectivity).
\end{definition}

\begin{theorem}
\thmlabelname{Pascal's identity}

For nonnegative integers $n,k$, 
\[\binom{n+1}{k+1} = \binom{n}{k} + \binom{n}{k+1}.\]
\end{theorem}

\begin{proof}
    The left hand side represents the number of ways to choose $k+1$ elements from a set of size $n+1$. Another way to count this is to consider whether or not to include the last element in the set. If this element is included, this contributes $\binom{n}{k}$. If this element is not included, this contributes $\binom{n}{k+1}$. Together, this forms the right hand side, so we are done. 
\end{proof}

\subsection{PIE}

\begin{theorem}
\lemlabelname{PIE}

Let $A_1, A_2, \hdots, A_n$ be finite sets. Then 
\[\norm{A_1\cup \hdots \cup A_n} = \sum_{j=1}^n(-1)^{j-1}\sum_{\{i_1, \hdots, i_j\}\subseteq [n]} \norm{A_{i_1}\cap \hdots \cap A_{i_n}}.\]
\end{theorem}

\begin{example}
\exlabelname{Derangements}

A permutation on $n$ elements $\pi_n$ is called a \ac{derangement} if $\pi(i)\neq i$ for all $i\in [n]$. Let $D(n)$ be the number of derangements in $S_n$. Then 
\[D(n) = \sum_{k=0}^{n}(-1)^k\frac{n!}{k!}.\]
\end{example}

\begin{proof}
$D(n)$ is equal to $n!$ minus the total number of permutations that have at least one fixed point. Let $A_i$ be the set of all permutations which fixes $i$. Then 
\[D(n) = n! - \vert A_1\cup \hdots \cup A_n\vert.\]

For any set of $k$ points which are fixed, there are $(n-k)!$ ways to permutate the remaining $(n-k)$ elements. Therefore, by PIE,
\begin{align*}
    \vert A_1\cup \hdots \cup A_n\vert &= \sum_{j=1}^n (-1)^{j-1} \sum_{\{i_1, \hdots, i_j\}\subseteq [n]} \vert A_{i_1}\cap \hdots \cap A_{i_n}\vert \\
    &= \sum_{j=1}^n (-1)^{j-1} \binom{n}{j}(n-j)! = \sum_{j=1}^n (-1)^{j-1} \frac{n!}{j!}.
\end{align*}

Substituting this into our expression for $D(n)$ gives us the desired result. 
\end{proof}

Using the taylor series for $e$, it can be proven that $D(n) = \lfloor n!/e + 1/2\rfloor$.

\begin{example}
    \exlabelname{Euler Totient}

    For any positive integer $m$, the euler totient $\varphi(m)$ is defined as the number of positive integers between $1$ and $m$ inclusive that are coprime to $n$. If the prime factorization of $m$ is $p_1^{a_1}p_2^{a_2}\hdots p_k^{a_k}$, then 
    \[\varphi(m) = m\prod_{i=1}^k\left(1-\frac{1}{p_i}\right).\]
\end{example}

\begin{proof}
    Consider instead the number of positive integers that are not coprime to $n$. These positive integers have at least $1$ prime power in common with $m$. Let $A_{i}$ denote the set of positive integers $\leq m$ with prime power $p_i$. Then 
    \begin{align*}
        \varphi(m) &= m - \vert A_{1}\cup \hdots \cup A_{k}\vert \\
        &= m - \left(\sum_{j=1}^k (-1)^{j-1} \sum_{\{i_1, \hdots, i_j\}\subseteq [m]}\frac{m}{p_{i_1}\cdot \hdots \cdot p_{i_j}}\right) \\ 
        &= m\left(1 + \sum_{j=1}^k \sum_{\{i_1, \hdots, i_j\}\subseteq [m]}\left(\frac{-1}{p_{i_1}}\right)\left(\frac{-1}{p_{i_2}}\right)\hdots \left(\frac{-1}{p_{i_j}}\right)\right) \\ 
        &= m\prod_{i=1}^k \left(1 - \frac{1}{p_i}\right),
    \end{align*}
    where the last line follows by polynomial expansion.
\end{proof}







